\_codedecl \skiptorecto {Macros used for typesetting Franz Kafka - Metamorphosis, 2022-06-09}

    \_doc -----------------------------
    First I define which font to use, the size, and leading.    
    If you are using the default f-ebgaramond.opm, 
    make sure you have the Octavio Pardo version of EBGaramond.
    This is the version you’ll find in the \LaTeX\ font catalogue,
    and on Google fonts.
    \_cod -----------------------------

\fontfam[ebgaramond]
\typosize[11/14]

    \_doc -----------------------------
    \`\parskip` is set to 0pt instead of the default 0pt plus 1pt.
    We want, as far as possible, to typeset on a grid. 
    I.e., lines must be exactly \`\baselineskip` apart,
    and all pages must be vertically aligned with regard to lines.
    \_cod -----------------------------

\parskip=0pt

    \_doc -----------------------------
    Setting \`\parindent` to something so the indentation is vaguely
    square-like. 
    \_cod -----------------------------

\parindent=\baselineskip

    \_doc -----------------------------
    Widows and orphans can be tricky puzzles to solve. 
    Bringhurst and Tschichold both recommend adding and removing lines
    of the page as a method, and thanks to code copied from
    {\em Tips and tricks \#81} there’s an easy way to do that.
    Therefore I allow widows and orphans with \`\widowpenalty` and 
    \`\clubpenalty`, with  and plan to fix these
    as the very last step before everything is done. More about this later.
    \_cod -----------------------------

\widowpenalty=0
\clubpenalty=0

    \_doc -----------------------------
    As this is the first project, I'm keeping things simple by setting
    values for \`\hsize`, \`vsize`, and \`margins` to get something
    that looks at least similar to what I imagined.
    \_cod -----------------------------

\hsize=10cm
\vsize=\topskip \advance\vsize by 32\baselineskip % 33 lines

\margins/2 a5 (2,,1.5,)cm

    \_doc -----------------------------
    The title page and copyright page should not have page numbers, so
    I modify \`\footline` to have a conditional, which I have set to true 
    as default. Since “outer margin” is different for recto and verso pages,
    the footline reverses the order of commands for even numbered pages.
    The \`\footlinedist` must take into account that I might add a line
    to the page as previously described regarding widows and orphans.
    \_cod -----------------------------

\newif\iffootline
\footlinetrue

\footline={
    \iffootline
        \ifodd\pageno
            \rmfixed \hss \folio \quad
        \else
            \rmfixed \quad \folio \hss 
        \fi
    \fi
}

\footlinedist=2\baselineskip

    \_doc -----------------------------
    \`\skiptorecto` is a simple page break which skips an extra page
    if needed to land on a recto page.
    \_cod -----------------------------

\def\skiptorecto{%
    \vfil\break
    \ifodd\pageno\else\leavevmode\vfil\break\fi
}

    \_doc -----------------------------
    I was unsure if I should call the three parts “chapters”, 
    or “sections”, so I used the macro \`\act` in the text source,
    and then to synonym it to something later. I also decided that I wanted
    to use the macro like this “\`\act`\ `II`”, which actually sets the name 
    of the chapter to “II” instead of relying on the automatic indexing. 
    Since there are only three chapters/acts, this was deemed acceptable 
    because it reduced the amount of code to change. 
    To be honest, the simplest solution would have been to define \`\act` 
    outside the \OpTeX\ sectioning system, but I wanted to get to know 
    the default commands better.
    \_cod -----------------------------

\let\act=\secc

    \_doc -----------------------------
    Eventually I decided that \`\act` could be a \`\secc`, and copied
    the relevant code from sections.opm as a starting point:
    \begtt
    \_def \_seccfont {\_scalemain\_typoscale[\_magstep1/\_magstep1]\_boldify}

    ...

    \_def\_printsecc#1{\_par
        \_abovetitle{\_penalty-200}{\_medskip\_smallskip}
        {\_seccfont \_noindent \_raggedright \_printrefnum[@\_quad]#1\_nbpar}%
        \_nobreak \_belowtitle{\_medskip}%
        \_firstnoindent
    }
    \endtt
    I make it text size, regular weight, centered, and make vertical 
    skips multiples of \`\baselineskip` to stay on the grid.
    \_cod -----------------------------

\_def \_seccfont {\_scalemain\_typoscale[\_magstep0/\_magstep0]\rm}
 
\_def\_printsecc#1{\_par
   \_abovetitle{\_penalty-200}{\vskip2\baselineskip}
   {\_seccfont \centerline{#1}}%
   \_nobreak \_belowtitle{\vskip2\baselineskip}%
   \_firstnoindent
}

    \_doc -----------------------------
    Now for the orphans and widows. The following code is copied from 
    {\em Tips and tricks \#81}, see the following link for explanation: 
    (\url{https://petr.olsak.net/optex/optex-tricks.html\#widows}).
    \_cod -----------------------------    

\newdimen\vsizedefault
\def\extravsizes#1{%
   \foreach #1\do ##1:##2{%
      \tmpnum=##1
      \sxdef{vs:\the\tmpnum}{##2}
      \advance\tmpnum by \ifodd\tmpnum -\fi 1
      \sxdef{vs:\the\tmpnum}{##2}
   }
   \global\vsizedefault=\vsize \corrvsize  % for page 1
   \global\pgbottomskip=0pt plus1fil minus\baselineskip
}
\addto\_begoutput{\global\vsize=\vsizedefault}
\addto\_endoutput{\corrvsize}

\def\corrvsize{
   \global\vsize=\vsizedefault
   \ifcsname vs:\the\pageno\endcsname
      \global\advance \vsize by \csname vs:\the\pageno\endcsname \baselineskip
   \fi
}

    \_doc -----------------------------
    Tl;dr: pages 40 and 41 both get -1 line, likewise with 48 and 49.
    \_cod -----------------------------

\extravsizes{ 40:- 48:- }


    \_doc -----------------------------
    The following can be a good way to visualize baselines,
    you need the file mriezka.tex which can be found through the
    \OpTeX\ website.
    \_cod -----------------------------

% Grid

%\input mriezka
%\grid \hsize,\baselineskip(\hoffset,\dimexpr(\voffset+\topskip-\baselineskip) ,\hsize ,1.1\vsize)

\_endcode

