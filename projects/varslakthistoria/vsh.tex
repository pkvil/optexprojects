\input vsh-macros

\input vsh-titlepage

\input gostaberling

%%% Magnus släkt:

%\setbox0\hbox{abcdefghijklmnopqrstuvwxyz}
%\the\wd0

\chapS{Carl Johan Petersson}{Relation} %Carl Johan Petersson

Carl Johan föddes 29/5 1836 i Lilla Boda, Mörlunda. 
Hans far hette Peter Karlsson, född 18/7 1802 i Borg, Mörlunda. 
Denna gården har varit i släktens ägo i minst 5 generationer tillbaka till 1650-talet. Carl Johans mor hette Stina Kajsa Karlsdotter, född 6/12 1806 i Lilla Boda. Även denna gården har varit i Stina Kajsas släkts ägo sedan mitten på 1700-talet.
Peter och Stina Cajsa får fem barn: Brita Lisa 1832, Carl Johan 1836, Christina Sophia 1839, Nils Peter 1843 och Adolf Fredrik 1846.

Carl Johan bor hos sina föräldrar fram till 1859 då han tar arbete som dräng på gården Stora Boda, Mörlunda socken. Arbetar där ca 1 år innan han gifter sig med Cajsa Lisa Magnusdotter.

Peter Karlsson dör 1858 och Stina Kajsa Karlsdotter dör 1882.


\chapS{Cajsa Lisa Magnusdotter}{Magnus mormors mor} %Cajsa Lisa Magnusdotter

Cajsa Lisa föddes 15/5 1839 i Näshult, Mörlunda. Hennes föräldrar hette Magnus Jonsson, född 7/6 1792 i Näshult, Mörlunda och Anna Maria Jonsdotter, född i Torssebo, Tveta den 12/7 1798.
De får barnen: Christina Sophia 1819, Johan Peter 1822, Carl Magnus 1824, Daniel August 1827, Anna Lovisa 1830, Maria Cathrina 1833, Gustaf Fredrik Otto 1835,Cajsa Lisa 1839 och Gustava Fredrica 1842.
 
Familjens hem i Näshult har varit i hennes fars släkt sedan början på 1700-talet.
Den första personen som vi har hittat på denna gård hette Jonas Svensson. Han var Sockenman,  vilket innebar att han ägde talan vid sockenstämman. Ett hedersuppdrag.
Levde mellan 1724-1808.

1850 flyttar familjen med de fyra yngsta barnen till Hultsnäs. De äldsta barnen har flyttat hemifrån.

Cajsa Lisa bor hemma på föräldrarnas gård tills hon gifter sig.


\chapX{Carl Johan}{Petersson}{Cajsa Lisa}{Magnusdotter}{Relation}


Paret gifter sig 18600729 i Mörlunda kyrka och bosätter sig på gården Hultsnäs, Näshult i Mörlunda socken. Hennes föräldrar bor och brukar gården till 1872 då Carl Johan och Cajsa Lisa tar över ägandet.
De får 10 barn mellan 1861-1881, ett av barnen dör i ung ålder.
Deras barn fick namnen: Johan Peter 1861, Amanda Sofia 1862, Carl August 1865, Oskar 1867, Constans 1869, Mathilda 1872-1872, Olivia Gustava 1873, Nils Emil 1876 (emigrerade till Nordamerika), Axel Valfrid 1879 och Josefina Emilia 1881.


Oskar bor i föräldrarhemmet  i Hultsnäs hela barndomen och uppväxt. I 20-årsåldern arbetar han på Ryningsnäs Gods som arbetskarl, mest i jordbruket. Han var även med när sjön Ryningen skulle sänkas för att skapa mer åkermark. Det grävdes en ny kanal innan Blankan. Oskar var med i det arbetslaget som sprängde och grävde kanalen. Än idag kan ett inristat vattenmärke i Haklaberget ses, V M 1887. Vattnet fick inte gå över det märket för då svämmades åkrarna över. Om detta hände så öppnas några vattenluckor. 


När gäster var inbjudna till har- och tjäderjakt  av godsägaren var Oskar och andra arbetare involverade. Men det var framförallt andjakten i Ryningen som var mest eftertraktad av gästerna. Godsets arbetare fick ställa upp som roddare och upplockare av byten. De skulle också gående genom vassruggarna få änderna att flyga upp och allra helst då i riktning mot den plats där skyttarna stod. Under de första dagarna av den lovliga andjakten sköts inte mindre än 600 änder av ett inbjudet 3-mannalag.


Familjen bor kvar i Hultsnäs till 1890 då de flyttar till ett närliggande hus, Bygget, som troligen inte låg så långt därifrån.


Carl Johan dör 19031216. Cajsa Lisa bor därefter hos sin son Oskar i Hultsnäs och dör där 19101106.
________________

\chapS{CARL FREDRIK SKÖLD}{Relation}


Magnus Danielsson född 1791 04 22 i Mossäng, Kråkshult socken. Antagen till soldat år 1817 vid Kungliga Kalmar regemente, Förste Majorens kompani. Han får soldatnamnet nr. 58 Brandt och blir tilldelad soldattorpet Lamåsa Kåragård i Alseda. Han bor även på Språklaretorpet, Alseda.
Enligt en notering generalmönsterrullorna 1926 är Magnus Brandt 35 år gammal, har tjänstgjort i 11 år och är 5 fot 9 tum, dvs 1,72 m lång. Gift med Stina Lisa Johansdotter född 1797 och har med henne 8 barn: Johan Magnus 1817, Jonas Peter 1820, Stina Cathrina 1823, Anna Lena 1826, Carl Fredrik 1929, Johanna Maria 1832, August 1835 och Carolina 1838.


1832 flyttar familjen till Mönsterskrivare boställe, nr, 79 Hökaryd i Åseda socken.
Magnus dör 1862 och Stina Lisa 1846.


Sonen Carl Fredrik Magnusson född 1829 07 22 i Alseda. Antagen 1855 08 09 vid Kalmar Regemente, Livkompaniet, Första Korpolagskapet. Han får soldatnamnet Sköld och bor nu på nr. 3 Nyttorp, Tulunda i Mörlunda socken. Han är 6 fot 2 tum, dvs. 1,85 m lång.
1883 06 29 slutar han sin soldatbana, i generalmönsterrullorna finns en anteckning att “ han har tjent utmärkt väl “ och blir gratialist. Det betyder att han får gratifikation, en pension som utbetalas 2 gånger om året vid uppvisande om prästbetyg och tjänsteintyg. Nu kan han försörja sin familj på en annan gård.




 


________________


\chapS{HELENA CHRISTINA HENRIKSDOTTER}{Relation}%Helena Christina Henriksdotter
Helena Christina Henriksdotter. Hon är född 1829 07 24 i Tribäck, Hattemåla, torpet  Lyktan, Fagerhults socken. Hennes far heter Hindrik Jonsson, född 1803, och hennes mor heter Maria Alexandersdotter, född 18060625. Föräldrarna kommer från trakterna runt Virserum och Fagerhult där deras anfäder har bott i flera generationer.
På torpet Lyckan föds också en son Jonas Magnus 1832. En tid efter det så flyttar familjen till Liskatorp, även det i Fagerhults socken.
1834 går nästa flytt till torp Eksjögle, Backstugan i Virserum socken. Fadern är nu brukare på gården Gröndahl. Fyra söner till föds där, Gustaf Peter 1836, Carl August 1839, Johan Claes 1843 och Adrian 1847.


Helena Kristina flyttar tillbaka till Fagerhult 1844 och arbetar som piga på gården Medelgärde. Återvänder till familjen i Virserum 1846. Flyttar 1853 till Åkerö i Mörlunda socken där hon arbetar som piga. På samma gård arbetar Carl Fredrik Magnusson Sköld som dräng och de blir ett par.




%Carl Fredrik Sköld och Helena Christina Henriksdotter
\chapX{CARL FREDRIK}{SKÖLD}{HELENA CHRISTINA}{HENRIKSDOTTER}{Relation}

Carl Fredrik och Helena Christina gifter sig 7/3 1856. De bosätter sig på soldattorpet Nyttorp, Mörlunda socken och får 7 barn:


Ida Charlotta Rosetta f. 1856, emigrerar 1883 till Rockford Winnebago, Illinois
Johan August             f. 1859          “       1881 till Belvidere Boone, Illinois
Carl Oscar                  f. 1861 död 1862
Carl Magnus               f.1863,  emigrerar 1889 till Rockford Winnebago, Illinois
Oscar Algot                f. 1865          “        1890               “                           “                
Gustaf Adolf Gottfrid  f.1869           “        1890               “                           “
Maria Kristina f. 1873 är den enda av syskonen som stannar i Sverige.


1869 flyttar familjen till Orrhällan torp, Ryningsnäs i Mörlunda socken. De blir kvar där i 2 år sedan flyttar de till Lilla Bölö, Mörlunda socken.


1890 går flytten igen, denna gång till gården Hultsnäs, Mörlunda socken.


Bild


Carl Fredrik och Helena Christina brukar gården tills Carl Fredrik dör 19120809. Helena Christina bor kvar tills hon dör 19180130.




________________


%Carl August Bom
\chapS{CARL AUGUST BOM}{Relation}

Nils Persson född 1764 i Temshult, Kråksmåla. 
Han blir antagen som soldat 1788 08 18 och blir placerad på Bondeberga nr, 87 i Högsby socken. Nu får han soldatnamnet Bom.
1789-1790 tjänstgör han på Sjökompaniet, men redan 1793 slutar han sin soldatkarriär.
Han flyttar till Kloster, Kolsrum i Högsby. Under årens lopp hinner han gifta sig 3 gånger.
I husförhörslängden 1818 står han som häradstjänare, dvs. vaktmästare vid tingshuset i Staby.
En annan betydelse är fångvaktare.
Nils Persson Bom dör 1830 09 17.
Sonen Peter Nilsson Bom, född 1791 i Bondeberga, som Nils fick från det första giftet med Lena Jonsdotter, bor på gården Kloster med sin hustru Lisa Kajsa Nilsdotter och deras 11 barn.


Ett av barnen heter Gustav Persson Bom, född 1833 06 15 växer upp på gården. I mitten av 1850-talet arbetar han som dräng på en gård i Sjötorp, Fågelfors. På samma gård finns en piga som heter Christina Jonsdotter, 1833 11 05. Tycke uppstår och 1858 gifter de sig och bosätter sig på Sjötorp nr.1 som är Christinas föräldrarhem. De får 7 barn:
Mathilda Sofia 1859, Carl August 1862, Gustaf Emil 1866, Johan Rudolf 1870, Axel Reinius 1874, dödfött gossebarn 1877 och Pehr Alfrid Martin 1878.


Carl August bor hemma fram tills han gifter sig och ett par år framöver.


1881 dör Gustav Persson Bom av bröstlidande. Christina Jonsdotter dör 1934 12 10, 101 år gammal. Hon kallades Moster Bomma av hela släkten.


Bild på moster Bomma

 
%Selma Terese Åberg
\chapS{SELMA TERESE ÅBERG}{Relation}

Namnet Åberg är ett vanligt namn i Fågelfors med omnejd och de flesta av dessa har nog gemensamma förfäder. Går man på kyrkogården i Fågelfors idag så hittar man namnet Åberg på många gravstenar. Åberg-släktet i Fågelfors är en smedsläkt på 4 generationer som man kan härleda tillbaka till en Nils Larsson född i Rumskulla och som sedan fick namnet Åberg. Han blev smedmästare, i betydelse av att han var fullt utlärd smed vid Fredriksfors bruk i Döderhult. Där arbetar också hans son Lars Åberg, även han smed. Han får en son, Johan Peter Åberg, som arbetar som hjälpsmed vid Pauliström bruk, Målilla socken. När Fredriksfors järnbruk lades ner flyttades verksamheten till Fågelfors och släkten Åberg följde med dit.Johan Peter får arbete som stångjärnssmed vid Fågelfors bruk.Fjärde generationen  Frans Oskar Åberg arbetar som hammarsmed vid Fågelfors bruk.  


Frans Oskar född 1837 07 20 i Storebro, Vimmerby socken gifter sig 1861 09 07 med Johanna Gustava Jonsdotter född 1838 07 05 i Klebo, Högsby socken.
De får 11 barn däribland Selma Therese. Hon är född 1874 06 19. 
Hennes syskon: Ida Sophia 1859, Gustava Charlotta 1862, Carl Oscar 1867-1868, Carl Oscar 1869, Emilia Maria Josephina 1871-1872, Axel Reinius 1873-1873, August Bernhard 1882-1882, tvilling Alma Emilia 1882.
Två av barnen, Gustava Charlotta och Gustaf Emil, emigrerar till Nordamerika 1887.


Frans Oscar dör 1888 och Johanna Gustava dör
 1912.
Carl August Bom och Selma Åberg

I Fredrik Lobergs bok “Samhällsbyggarna” kan man läsa vad han har skrivit om Selma och Carl August Bom:
“Selma Åberg var 16 år, 18900621, när hon gifte sig med den 12 år äldre Carl August Bom. Eftersom Selma var minderårig hade familjen skrivit till Kungens befallningshavare med en önskan att hon fick gifta sig med Carl August. Kung Oscar den II:s underlydande gav klartecken med vändande post. 
Hon flyttade med sin make till Sjötorp nr.1. I det lilla torpet bodde även Carl Augusts syskon och hans mamma Kristina Jonsdotter, hon kallades för “Moster Bomma” och var änka efter Gustav Bom.
Två månader efter giftermålet föddes sonen Artur 12 augusti 1890.
18921112 flyttar den lilla familjen till Hyltan i Fågelfors.  Snart kom flera syskon Gunnar, Anna, Elin och Henry. 1901 flyttar familjen till Hemgården på Bruksgatan även det i Fågelfors.  I Juni 1903 när Henry var knappt ett år dör Carl August.Han har varit och flottat timmer på bruksdammen och trots feber och hosta fortsatte han att arbeta. Efter arbetsdagen kördes Carl August skakande och kraftlös med häst och vagn till sitt hem på Bruksgatan men han tillfrisknade aldrig. Dödsorsaken den 11 juni 1903 fastställdes till lunginflammation. Selma står nu ensam med 5 barn, endast 29 år. I början av 1890-talet hade familjen flyttat till ett hus i Fågelfors som ägdes av bruket och nu när Carl August var död meddelade brukspatron Ekströmer att de var tvungna att flytta före ett visst datum om de inte kunde köpa huset. Selma vädjar till brukspatron att han skulle ge henne en chans att hon och barnen skulle försöka tjäna ihop till de summa som krävdes för att kunna bo kvar. Han lovade om hon fick ihop pengarna så skulle hon få köpa Hemgården nr.5 på Bruksgatan 30 i Fågelfors.

Selma skrev också till sina släktingar, syskon till både henne och Carl August, som emigrerat till Amerika om hjälp. De hade inte glömt vardagsslitet i sin hembygd och med jämna mellanrum skickade de hem bidrag till familjen.
När 6 år har passerat efter Carl Augusts död hade Selma och hennes barn lyckats spara ihop 72 kronor.
 Selma kunde den 14 mars 1908 stolt stega in till Brukspatronen och meddela att hon hade pengar till att köpa loss huset om nu Brukspatron stod vid sitt ord, vilket han gjorde.
Så här såg köpekontraktet ut.


Två år senare den 27 maj 1910, anslöt sig Selmas 19-årige son Gunnar till den långa strömmen av emigranter västerut. Han färdades mot Moline i Illinois, Rock Island County. Där hade Selma syster Alma och flera arbetarfamiljer bosatt sig. Den dramatiska resan tog 3 ½ vecka, den 20 juni kunde andas ut i sin nya tillvaro i Nordamerika. Han skrev ett långt brev hem till mamma Selma och syskonen.
Brevet




Selma bor kvar i huset på Bruksgatan tills hon dör 1939.


Karl Arthur Konrad Bom föddes 1890-08-12 i Sjötorp i Fågelfors. Föräldrarna var Carl August Bom och Selma Terese Åberg.
 
Arthur och hans bror Gunnar fick redan vid 13 och 12 års ålder börja arbeta vid Fogelfors Bruk, här kan vi se utdrag ur löneboken för några månader under 1903. Vi kan se att Arthur tjänade 65 öre om dagen och Gunnar 10 öre för varje kväll efter skolan.
13-årige Artur och hans ett år yngre bror Gunnar sökte arbete på snickerifabriken. Artur tjänade 7,97 kronor varje månad, som han lämnade till sin mor. Gunnar arbetade kvällspass och tjänade 10 öre i timmen. Före och efter arbetet tog bröderna på sig de finaste tröjorna som Selma stickat.Vid fabriksporten utanför bruksentren sålde pojkarna kringlor som Selma bakat medan hon passade de yngsta barnen i hemmet. Detta gav ytterligare ett litet tillskott till hushållskassan. Ryktet om de välsmakande kringlorna spred sig snabbt i brukssamhället. Selma började också sticka vantar, sockor och tröjor på beställning. För detta tjänade hon några kronor varje månad.


Gunnar beslutade sig för att fara till släkten i Amerika och vid 19 års ålder startade hans färd. När han kom fram skrev han ett brev hem och berättade om resan och mötet med sin släkt. Originalbrevet är skrivet med en handstil som många har svårt att läsa idag, därför har vi skrivit av brevet vilket återges här.
Plats för brev.
Efter att Gunnar emigrerat låg ansvaret för familjens försörjning i ännu högre grad på storebror Arturs axlar. Redan i unga år bestämde sig Artur för att leva ett nyktert liv och vid 19 år, tillsammans med kyrkovaktmästaren Linus Johansson startade han Nationalgodtemplarordens, NTO:s Logen 1343 Liljan i Fågelfors 1907.
1912 när Artur var 22 år valdes han till NTO:s distriktssekreterare. 


1909 träffar Artur sin blivande fru Anna, de förälskar sig i varandra men giftermål får vänta några år. Båda två måste ta hand om sina föräldrar. Arturs syskon är minderåriga och Annas syskon har alla emigrerat.
1919 kan de äntligen gifta sig.


Arthurs syster Anna fick jobb som hembiträde hos Konsul Jansson i Kalmar. Hon förblev ogift och man kan anta att hon betydde mycket för honom eftersom hon fanns med i hans testamente och fick för den tiden en ganska stor summa pengar.
Plats för testamente


Arthurs andra syster Elin flyttade till Stockholm och gifte sig. De var så gott som varje sommar hemma och hälsade på släkten i Fågelfors.


Den sista i barnaskaran Henry blev Fågelfors trogen, arbetade på bruket, gifte sig och byggde hus på samma tomt som Selmas hus stod på.


Arthur och hans syskon växte upp under knappa förhållande där de lärde sig att ta ansvar.


Bilder på


Arthur




Carl August




Selma
















 
________________


Erik Magnus Nilsson


Erik Magnus föddes 25/12 1844 i ett torp som heter Lindekulla. Hans far, Nils Magnus Jonsson kom från Gryssås utanför Fogelfors och hans mor, Johanna Petersdotter från Kloster som ligger under Kolsrum Högsby. Lindekulla ligger någon kilometer väster om Massemåla på Odensvi marker. Huset är sedan länge rivet men man ser grunden och ett flertal odlingsröse i anslutning till där huset låg. Flera fruktträd växer i gläntan som öppnar sig mot skogsvägen som går 25 meter därifrån.
Bild på Lindekulla.
Hans syskon: Gustaf Fredrik 1839, Nils Peter 1841, Carl Johan 1843, Johanna Sofia 1847, Ulrika 1849, Carolina 1852 och Otto Reinhold 1858.


Erik Magnus flyttar som ung runt i trakterna av Fågelfors där han arbetar som dräng. Han kommer som 17-åring till Nötebäckshult 1861, Hässleås 1864, Gryssås 1874 och Hyltan 1876.   


Nils Magnus Jonsson dör 1883 och Johanna Petersdotter dör 1893.




________________


Gustava Charlotta Carlsdotter
Gustavas far Carl Nilsson har ättlingar som vi kan  följa tillbaka till slutet av 1400-talet, beroende på att förfäderna tillhört prästsläkter i trakterna Eksjö och Ingatorp. Den äldsta vi hittar bland Carls förfäder var Olof Månsson som var riksdagsman och krögare i Vimmerby. Han var med vid Karl den niondes kröning i Uppsala där han under ett gräl blev mördad. Hans dotter Botilda gifte sig med en kyrkoherde från Ingatorp med det fantasifullt tagna namnet Nicolaus Petri Ingatorpensis.
Några generationer senare bryts traditionen att bli präst som sin far och Knut Rosinius blir bonde i Ekeby Högsby. Denne Knut gifter sig med Maja Lisa Staf som troligen har anknytning till Staby Gästgiveri. Knut och Maja-Lisa får en sån som heter Emanuel. Emanuel gifter sig med Jungfru Ulrika Elisabeth Fock från Långhult.Detta par är Carl Nilssons farfar och farmor.
I en artikel i Vimmerby tidning kunde man läsa om Astrid Lindgren förfäder och där kan man se att även hon har riksdagsmannen Olof Månsson som förfäder.


Carl Nilsson född 1817 i Lixhult, Högsby. Gift med Johanna Gustafsdotter född 1822 i Kalvehorvan, Fliseryd.
De får barnen: Anna Christina 1843,emigrerar till Nordamerika, Johan August 1849 emigrerar till Nordamerika 1887, fd.marinsoldat med efternamnet Rosén, Per Alfred 1851, Ida Sofia 1853, Gustava Charlotta 1854, Johanna Fredrika 1858 och Karl Emil 1862 emigrerar till Nordamerika, ändrar efternamnet till Nelson.


Gustava Charlotta börjar arbeta som piga på  fyra olika gårdar i Högsby mellan 1871-1876 bl.a hos apotekare Landstedt. 1876 flyttar hon till Oskarshamn och arbetar där som piga hos handlare Axel Lundgren men efter ca ett år flyttar hon tillbaka till föräldrarna i Lixhult.


Carl Nilsson dör 1915 och Johanna dör 1900, båda två i Forsaryd, Fågelfors.




Erik Magnus Nilsson och Gustava Charlotta Carlsdotter


De gifter sig 18781109 och bosätter sig i Hyltan, Fågelfors. Erik Magnus arbetar som gårdsrättare på Fågelfors Järnbruk. 
De får barnen: Edvard Manfred Severin 1888, Erik Gunnar 1890 och Elvira Kristina 1891.
 
Erik Magnus sköter sitt arbete bra, han blir befordrad till rättare på Fogelfors Bruk och familjen flyttar 1891 till en tjänstebostad i Fåglebo, utanför Fågelfors samhälle.
Två barn till: Anna Martina Charlotta 18930627 och Carl Arvid Henry 1896.


Bild på familjen utanför huset.
Anna Martina Charlotta föddes i Foglebo Fågelfors den 27/6 1893. Föräldrarna var Erik Magnus Nilsson som var rättare på Fogelfors bruks gård i Foglebo och hennes mamma hette Gustava Charlotta Carlsdotter var född 3/11 1854 i Lixhult Nr.3.
Anna växte upp med sina fyra syskon Manfred född 1888, Gunnar född 1890, Elvira född1891 och Carl född 1896. 
Under åren 1910 till 1915 bestämmer sig alla hennes syskon för att söka lyckan i Amerika och utvandrar. Anna stannar kvar och tar hand om föräldrarna. Syskonen som utvandrade fick förmodligen hjälp från äldre släktingar som utvandrat tidigare. De finner sig tillrätta i USA och får arbete, gifter sig och får barn, så man kan tänka sig att det finns fler släktingar från den här grenen i USA än här i Sverige.
De håller kontakt med Sverige genom brev och längre fram även resor för att hälsa på sina släktingar.
Anna gifter sig 1919 med Arthur Bom.




Anna är det enda barnet som stannar i Sverige och tar hand om föräldrarna.Hennes syskon emigrerar till Nordamerika mellan 1905-1915. 


Brev från Gunnar. Info om golf, m.m
Erik f.d rättare, ny flytt.




Oskar Karlsson och Maria Kristina Sköld


Oskar och Kristina gifter sig 18970324 i Mörlunda kyrka och paret bor till en början hos Kristinas föräldrar i Hultsnäs. Deras första barn,Einar, föds där 1898 men kort därefter flyttar den lilla familjen till gården Lumsebo ca 800 meter från Hultsnäs. Andra barnet, Oskar Reinhold Emanuel föds 18990701 men hans liv blir kort. Redan dagen efter dör han av svaghet. Det finns en anteckning i kyrkoanteckningarna att “ Fadren, som sjelv verkstälde nöddopet. Barnet erhöll nöddopet som ej blef bekräftadt.”
Livet går vidare och snart kommer fler barn: Astrid 1900, Elna 1902, Agda 1904, Märta 1907, Sigurd 1910, Aina 1913.
1914 flyttar familjen tillbaka till Hultsnäs och här föds deras sista barn Erik 1918. Det var en tvillingfödsel men hans tvilling dog vid förlossningen.


Kristinas far har dött 1912 och hennes mor är undantagsänka i Hultsnäs så det blir naturligt att Oskar tar över gården i egenskap av hemmansägare.
 Familjen bor i Hultsnäs i Mörlunda socken. I familjen ingår även Oskars svärföräldrar Carl Fredrik och Lena Christina Sköld. Sigurds första barndomsminne är när hans morfar har dött 1912. Av någon anledning så dröjer det innan han körs till Mörlunda för begravning så han läggs i en kista och ställs ut i ett uthus på gården. Sigurd är då ungefär 2 ½ år.


Ett annat minne rör en gammal gumma som familjen bor granne med. Hon är fattig och bor ensam i en liten stuga som ligger ett par hundra meter från Hultsnäs. Sigurd och hans mamma gick ofta dit med mat till henne.
När Sigurds mormor dör 1918 flyttar familjen till Lilla Boda där sista barnet föds. Nu är de 8 syskon. Ytterligare två barn har dött vid födseln 1899 och 1918.
Troligen är Lilla Boda en släktgård på Oskars sida.
Runt 1930 går flytten till Hässlås, Fågelfors. Familjen har utökats med två barn, Rut som faktiskt var Oskar och Kristinas barnbarn( hennes mamma har dött och pappan kan inte ta hand om sitt lilla barn) och en fosterpojke som heter Lennart Gustafsson.


Oskar blir dement de sista åren, det är svårt att ta med honom på t.ex släktkalas så han får stanna hemma. För att han ska ha något att göra så får han en liten uppgift som de vet att han klarar av. En gång sa hans son att han kunde bära in ved i köket medans alla var borta. Det gjorde Oskar med besked, han fyllde hela köket tills det inte fick rum ett vedträ till.
Valborg har berättat om ett besök i Hässlås, lilla Ann-Mari  kryper omkring på golvet i lugn och ro. Plötsligt hör Valborg att Ann-Mari gråter och springer in i rummet där även Oskar befinner sig. Ann-Mari sitter gråtande under ett bord med alla stolar ordentligt inkörda. Oskar är väldigt nöjd, han talar stolt om att han har stängt in alla kalvar så de inte smiter sin väg.








Sigurd tar körkort, han har berättat hur det gick till. Efter några körlektioner i Oskarshamn så är det dags för uppkörning, det går bra. Teoriprovet består av två frågor, den ene var så här: Får man köra bil om man har druckit sprit? Sigurd svarar nej, det får man inte. Då säger körskoleläraren: Jo, lite får man ta. Körkortet utfärdas 21/4 1931.


Sigurd hinner med många yrken i sin karriär. Bland de första arbeten han hade var som snickare på Nya Fabriken i Fågelfors. Samtidigt hade han extrainkomst hemma på gården i Hässlås, en silverrävsfarm. Skinnen såldes och det blev lite extra pengar i mitten av 1930-talet. Sedan arbetade han som murare hos Holmström i Fågelfors. Under andra världskriget hade han kolugnar i Hässlås och sålde kol till Försvaret. Sigurd berättade hur Fogelfors bruk såg honom som konkurrent och försökte svälta ut honom genom att köpa in all ved till överpris. Men en dag kom folk från bränslekommissionen och undrade varför hans kolugn stod stilla när behovet av kol var stort, han berättade som det var, varvid kommissionen löste in brukets ved som Sigurd fick ta över. Inköparna på bruket stod där med lång näsa. Efter kriget köpte han en lastbil och körde grus, samt använde den till att hämta mjölk hos bönder och köra den till mejeri.


Sigurds syskon


Einar, 1879-1971, gifter sig med Naemi  Karlsson och får två barn, Berit och Alf. Han tar över Hässlås efter sina föräldrar. Familjen flyttar sedan till Hultsfred.


Astrid, 1900-1960, gifter sig med Gunnar Thyrén och får en son, Enar. Hon har eget företag i Fågelfors, kafé och bageri som är beläget i huset bredvid Filadelfiahuset.


Elna, 1902-1930, gifter sig med Karl Gunnar Karlsson och får en dotter, Rut. Elna dör när Rut är 1½ år, pappan kan inte ta hand om henne utan hon växer upp i Hässlås hos sina morföräldrar.


Agda, 1904-1990, gifter sig med Henry Bohm och får två barn, Gerd och Roland. Agda var en händig person, hon sydde både till familjen och andra, målade om och tapetserade när det så behövdes i hemmet. Hon var duktig vid vävstolen, kunde väva många invecklade mönster. Henry var en spjuver, han tyckte om att skoja och tävla om de mest besynnerliga saker. Han hade Fågelforsrekord i att kunna joja längst, han höll på hela natten och vann till slut. Tävlingen om vem som kunde äta mest senap vann han naturligtvis.


Märta, 1907-1978, var särlingen i familjen. Hon reser till Stockholm och söker lyckan. Hon gifter sig med Herman som senare visar sig vara oärlig. Han säljer gödsel, men det är bara vatten och för det får han sitta av sitt straff på Långholmen. Innan dess flödade pengarna, paret bjuder Märtas föräldrar till Stockholm och tar med dem till Berns Salonger. Herman dör ganska tidigt och Märta berättar att Herman går igen och visar sig som en liten figur i deras hem.Efter detta har Märta tre förhållanden. Yngve, Gustav och Lennart. När Märta dör på 1970-talet åker Sigurd och hans bror Erik till bodelningen i Stockholm. Sigurd får bl.a ärva väggklockan av guld.


Aina, 1913-1991, gifter sig med John Enocksson och får en dotter, Ann-Britt. Aina arbetar en tid hos Astrid på kafét.


Erik, 1918-1980, gifter sig med Eva Petersson och får tre barn, Monika, Inga-Lill och Carola. Samtidigt som Sigurd bygger huset Björkhaga på Parkvägen så bygger Erik sitt hus mittemot Sigurds. Kort efter att Sigurd med familj flyttar till Kvillehult så köper Erik en gård som heter Rydet och ligger på vägen till Virserum. Den äldsta dottern, Monika, vill se världen och hamnar i Argentina där hon bor med sin familj i många år. Inga-Lill gifter sig med Sixten Jacobsson och får en son, Mikael. Hon blir änka tidigt. Bor i Virserum. Carola gifter sig med Jan Strömberg och får tre barn. Efter Eriks död tar Carola med familj över Rydet och flyttar dit.


Fosterbror Lennart Gustafsson gifter sig med Berit och får tre barn. Familjen bor i Fågelfors och räknas som släkt hela tiden.


Arvegods från Sigurd och Valborg: väggklockan kommer från Sigurds föräldrahem. Vid bodelningen efter hans mor,1959, så var det ingen som ville ha den gamla klockan. Efter en stund säger någon av syskonen ” Ge den till pojken ” och så blev det, Magnus fick den. Örnen högst upp på klockan saknades. Många år senare så snidade Lindström i Fågelfors en ny. Magnus fick igång klockan och den gick så fint tills en natt då vi vaknade av att den slog och slog. När vi hade räknat till över 60 slag så fick det vara nog och Magnus stannade den. Det blå skåpet stod en gång i tiden i Hultsnäs ( Sigurds barndomshem ) innan det följde med till Hässlås.
Guldklockan kommer från Sigurds syster Märta. Skrivbordet som Magnus har nu var Sigurd och Valborgs skrivbord i Kvillehult. Fia-spelet köpte Valborg på en marknad i Virserum, på spelet har Pälle Näver skrivit en dikt, ”Gick här och frös, då kom en tös, solig och grann, och kylan försvann”.
Den blå mjölkpallen och mjölkkannan har varit Valborgs farmor. Den bruna kaffekvarnen är troligen hennes också samt den lilla kaffekokaren av koppar. I Kvillehult i hallen uppe står en grön sammetssoffa som stod i Fågelfors. Selma Åberg köpte den i Vetlanda. Den gröna kaffekvarnen kommer från Sigurds föräldrar.
Tavlan med fåglarna
Magnus fick ett fickur av sin pappa när han fyllde 25 år. När Peter fyllde 25 år fick han den av Magnus.


Ett barnbarn till Kristina, Gerd Sinnerström, har delat med sig om sina minnen om sin mormor Kristina. Både händelser som hon har fått sig berättat eller själv har upplevt.
“ När Kristina i unga år var på väg till skolan en dag så skadade hon sitt knä så illa, oklart hur,  att hennes ben efter det blev krokigt och hon behövde gå med hjälp av en krycka. På äldre dagar använde hon kryckan till att lägga den böjda delen om två av sina barnbarns barn hals och hala in dem till sig. Magnus var en av dem, hans kusin Ulf den andre. Båda två var lite rädda för henne och tyckte inte om att bli inhalad på det viset.


Kristina var en stabil, rättvis kvinna som visste vad hon ville och var inte rädd av sig. När Oskar och Kristina köpte gården Hässleås utanför Fågelfors 1923, så kördes flyttlasset med häst och vagn från Hultsnäs. Det var Kristina som höll i tömmarna,men hon hade inte varit på väg mer än en kort sträcka förrän olyckan var framme. Vagnen välter och rullar ner för en liten slänt och hela lasset åker ut i naturen. Ingen blev dock skadad och när häst, vagn och bohaget var uppe på skogsvägen så fortsatte hon resan till nya huset.


Gerd och hennes bror Roland hälsade ofta på hos sin mormor och morfar i Hässleås. Oskar börjar tycka det är tungt att sköta gården, så äldste sonen Einar tar över gården 1937. Hans syskon får 1000 kr vardera och han blir ensam ägare. Einar gifter sig med Naima  och får två barn i samma ålder som Gerd och hennes bror. De leker ute på gårdsplan och när de blir hungriga springer de in till Kristina, som bjuder dem på hennes goda bröd, som hon har bakat i den stora bakugnen i köket. De får stora brödskivor med hemgjort smör på. Gerd minns ännu hur gott det var tillsammans med mjölk.


Åren gick och Oskars hälsa är inte den bästa, men Kristina som fortfarande tycker om att träffa folk får hjälp av dottern Astrid och hennes man Gunnar att t.ex åka och handla i. Hon får också följa med dem på Filadelfias olika tältmöten vilket hon uppskattar, hon är medlem i den kyrkan.


Oskar drabbas av demens och Kristina orkar till slut  inte sköta honom själv. Dottern Astrid tar emot dem i sitt hem och det är en stor hjälp för dem båda. 
1949 dör Oskar. Astrids café och hem ligger mittemot skolan som Gerd går i. Hon tittar ut genom fönstret  och får se 








________________






19230417 går flytten igen, Oskar har köpt Hässelås, en gård som ligger ca en halvmil utanför Fågelfors.


Bild


Oskar och Sigurds möten med gårdens tomtar.
Djur på gården lite senare i tiden. Sigurds arbetsliv fram till han gifter sig. Syskonens liv.
________________






________________








Arthur och Anna Bom


Första åren bor familjen på övervåningen i ett gult hus bredvid mejeriet i Fågelfors.
Född 18/12 1921 i Fågelfors.Föräldrarna hette Arthur och Anna Bom. Hon har 2 yngre syskon. 


Valborg går 6 år i skolan. 1937 har hon gått en 6 veckors kurs i Kalmar Läns Södra landstings skolkök. I kursen ingår enklare matlagning, bakning, kortfattad födoämneslära samt ordning och renlighet.Från sommaren 1939 arbetar hon som köksbiträde i 2 månader och som barnhusa i 1 år och 6 månader hos familjen Ekströmer på deras herrgård i Fågelfors. Tiden på herrgården är Valborg väldigt stolt över hela sitt liv. Det var nämligen bara flickor som kom från ”fina” familjer som fick arbete där. 1941 tjänstgör hon på barnkolonien Walldahemmet i Halland mellan 15 juni och 15 augusti. 15 juni 1942 och 15 januari 1943 arbetar hon som hembiträde hos Selma Henriksson i Djursholm. 1946 går hon en kurs i vävning i regi av Södra Kalmar Läns Hembygdsförening.








Barndomsminne: Elin och Valborg sover i en utdragssoffa. En kväll blir de törstiga och stiger upp för att dricka vatten i köket. Direkt efter att de är ur bädden så trillar väggklockan ner och slår upp ett stort märke i soffan och hamnar precis där flickorna har legat. Där var nog en änglavakt inblandad.
Kökssoffan står nu i Kvillehult och klockan hänger på väggen här i köket på Åsvägen. Klockan har en egen historia, släkt från Amerika har haft den med sig vid ett Sverigebesök. Det fattas en målad bild i ”glasfönstret”. Den fick några släktingar som minne av dem som hade skickat klockan en gång i tiden.


  
Elin,1926, gifter sig med Ruben Franzén 1961 i prästgården i Högsby.  Elin bär en vit, lång klänning, brudkrona och har röda rosor i sin brudbukett. Bröllopsmiddag intas på Hotell Morén. De träffades redan 1945, men av olika anledningar dröjde bröllopet. Elin utbildar sig till hemvårdarinna. Innan giftermålet bor och arbetar  hon i Mönsterås och Kalmar. Efter vigseln köper paret ett hus i Nybro. Elins sista arbete blir på Madesjö skola. Ruben är ingenjör och vägmästare.


Mauritz, 1930, gifter sig med Waldy Johansson i Kalmar slottskyrka. Mottagning efteråt i deras bostad på Storgatan i Högsby. De får två barn, Anders och Birgitta.
Ett starkt minne från den 28 januari 1935: Mauritz sitter på sin mammas arm och de står och tittar på den stora branden vid Fogelfors Bruk. Han minns att hans pappa var med och försökte släcka elden och hur han var klädd i en stor tjock päls som av kyla och vattensläckning står för sig själv när han tar av sig den.


1936 flyttar Arthur och Anna med barnen till den  nybyggda skogvaktarbostaden på Klevstigen.
________________








Sigurd och Valborg Kvillegård




Sigurd och Valborg träffades på ett lite märkligt sätt genom olika personer i bägges närhet.
Valborgs farmor, Selma Åberg Bom bor på Bruksgatan i Fågelfors. När Arthurs bror Henry har gift sig med sin Agda, så får de bygga sitt hus på samma tomt.
Nu kommer det märkliga. Henry är Valborgs farbror och Agda är Sigurds syster. I och med detta så umgås bägge familjerna flitigt men det dröjer flera år innan Sigurd och Valborg upptäcker varandra.
Dvs. Sigurd gör nog det med en gång men Valborg är förlovad med en annan. Fästmannen flyttar till Stockholm och det är väl meningen att hon ska komma efter men under tiden träffar han en annan flicka och förlovningen bryts.
Tiden går och en dag får Valborg ett brev från Sigurd. Han skriver om sina känslor för henne och frågar om det finns någon chans att det ska bli de två. Han vill dessutom att hon ska bränna brevet när hon har läst det. Det gör hon ju naturligtvis inte utan hon gömmer det i en kokbok. Brevet hittar vi av en slump 50 år senare när deras hem säljs och alla saker gås igenom och delas upp.
De gifter sig 24 maj 1947 i Fågelfors prästgård. Mottagning hos hennes föräldrar i det röda huset. Hela släkten är bjudna. Nils Forsblom som är gift med Valborgs faster Elin har tagit med sig fyrverkerier till bröllopet. När han kastar upp en pjäs i luften så fastnar den i en tall som börjar brinna, men allt slutar bra, alla hjälps åt att släcka.
Det är fortfarande svårt att få köpa allt som behövs, så släkten bidrar med kuponger till bl.a mat. Valborgs faster Anna, som bor i Kalmar, kommer med en stor krokan som är 1 meter hög. Den översta delen, själva kronan på verket, sparar Valborg som ett minne. Sigurds systrar Astrid och Aina har bakat smörbakelser och kringlor i sitt bageri.
Sigurd har byggt ett hus på Parkvägen i Fågelfors som får heta Björkhaga och där flyttar brudparet in och trivs riktigt bra. Snett mittemot har Sigurds bror Erik byggt ett  hus till sin familj.  .
Men Sigurd kan inte släppa sin dröm om en egen gård och efter ett par år kommer tillfället som gör att han kan förverkliga drömmen. Valborg är mer tveksam, gården med tillhörande skog är nedgången och det krävs en stor renovering för att kunna bo där. Sigurd vill så gärna ha gården och ber henne att ge det en chans, han lovar henne att om hon efter några år inte vill bo kvar så ska de flytta till ett bättre hus. När åren har gått kan Valborg inte tänka sig att bo på något annat ställe, så de stannar kvar.


Sigurd och Valborg köpte Kvillehult 1951 av Wilhelm och Gerda Jonsson för 32 000 kr. I köpet ingick hus, ladugård, skog, åkrar och en fotogenlampa som numera hänger över köksbordet här på Åsvägen. Det krävdes många och omfattande renoveringar innan familjen kunde flytta in. Det inköptes traktor av märket Ferguson, kor, grisar och höns.
På 1950-talet var Blankan en levande och folkrik by med många  barnfamiljer. Det fanns en affär som ägdes av Evert och Nanna Gahne, snickeriverkstad, kvarn, kraftstation och en smedja.
________________


Lisbeths släkt:
________________


Johan Peter Johansson


Anfäder till Johan Peter var alla mjölnare och kvarnägare till Landsbro kvarn i Nottebäck socken, Kronobergs län. Den förste var hans farfar Johan Fredrik Svensson, född 1804 04 04 i Strömmahult, Näsby socken. 1824 kommer han från Bäckseda till Landsbro kvarn och såg där han får arbete som mjölnare. Johan Fredrik gifter sig 1824 06 25 med kvarnägarens dotter Sara Christina Johansdotter Nordqvist. De får en dotter 1825 01 02 men de hann inte vara gifta så länge, bara ca 18 månader. Johan Fredrik dör i fläckfeber ( tyfus ) 1826 02 10. Sara Christina är då gravid med parets son som föds 1826 07 07 och han får namnet Johan Fredrik Johansson efter sin far.
I bouppteckningen efter fadern Johan Fredrik Svensson visar Landsbro kvarns inventariesumma att den är värd 823 Riksdaler 23 Shilling. Skulden är 47 Riksdaler 25 Shilling. Kvarnen med 2ne par stenar samt 8 Kappeland skattelagd jord är värt 500 Riksdaler. Även en fjärdepart på samma ställe belägna grovbladig såg, värd 50 Riksdaler.


Sara Christina gifter om sig med Peter Pehrsson som också är mjölnare. Johan Fredrik och styvfadern äger längre fram i tiden kvarnen och sågen tillsammans.


1850 gifter sig Johan Fredrik med Marta Regina Johansdotter från Sjöarp, Stenberga.
Marta Reginas far hette Johannes Wahlström. Han är rusthållare på Boda Djupsgård i Stenberga. Hennes farfar hette Petter Wahlström, han var kyrkovärd, nämndeman och häradsdomare i Stenberga. Tillhörde Bondeståndet. Farfars far Pehr Wahlström var ryttare vid Jansbotorp i Stenberga.


Johan Fredrik och Marta Regina bosätter sig på Landsbro kvarn och såg. De får 8 barn mellan 1852-1863, äldste sonen Johan Peter född 1852 02 15. 


Johan Fredrik dör 1897 12 31 i Landsbro och Marta Regina dör 1911 09 17 i Virserum där hon bor hos ett av sina barn.








________________


Kristina Sofia Johansdotter
Kristina hade flera anfäder som var soldater. Hennes farfars far Peter Sten, född 1789, var husar i Bringebecks Husartorp nr. 73 av Smålands regemente, Rosenlund, Kronobergs län.
Farmors far Magni Midas, född 1788, var soldat nr. 32 i Lindås rote 1814-1840. Förtjänst utmärkt vid fälttåg 1808-1809 ( Finska kriget ) och 1813-1814 ( Allierad Nordarmé i Tyskland, mot Napoleon ). 
Morfars far Sven Stolt, född 1777, soldat  för Kalmar regemente, Södra Vedbo kompani, rote nr.71 Ekekull, torpet Gölåsen. Antagen den 4 dec. 1801,noterad vara Småläning, 23 år och ogift. 5 fot 71/2 tum, dvs 167 cm lång. I rullorna noteras att Sven är kommenderad till Pommern 1806. Från 1817 tjänstgör han vid Göta Kanal som frivillig under vissa perioder. Han dör  under arbetet där den 18 maj 1828 och är troligen begravd på annan plats än hemsocknen.
Mormors far Jonas Hake, född 1779, antagen vid Kalmar regemente, Östra Härads kompani, rote nr.119 Trälarp den 4 april 1801. Kommenderad i fält 1805, Pommerska kriget. Han dör i det kriget 1805.


Kristina Sofia född 1864 09 23 i Mosjödal, Skruv Södregård, Nottebäck socken i Kronobergs län. Hennes far hette Jonas Peter Petersson, född 1840 08 29 på gården Rosenholm under Klackhult i Hornaryds socken, Kronobergs län och hennes mor Christina Lovisa Magnidotter, född 1836 12 04 i Backstugan Skurubo Södregård i Nye socken, Kronobergs län.
1886 flyttar familjen till Lunnagård, torpet Sjödala och där utökas familjen med barnen Matilda Charlotta 1870, Karl Peter 1874 och Augusta Emilia 1877.


Kristina Sofia flyttar 1884 10 27 till Björnhult, Södra Solberga, där hon arbetar som piga. Därifrån till Landsbro Kvarn och Såg, för nytt arbete som piga hos sin blivande make.


Hennes föräldrar och syskon flyttar till Kulla, Södra Solberga 1888-04-25. 
Fadern dör 1917 03 18 och modern 1925 05 15.
 










Johan Peter Johansson och Kristina Sofia Johansdotter


Johan Peter var gift två gånger, första giftermålet 1875 04 18 med Eva Sofia Nilsdotter född 1846 11 15 i Södra Solberga. De får 4 barn: Emma Kristina Seraphina 1876, Johan August 1878, mjölnare,  Karl Stefanus 1881.Det finns en anteckning i husförhörslängden om honom:” Fick nöddop av sin farmor Marta Regina Johansdotter i Landsbro, varefter barnet följande dag dog.” Det fjärde barnet Ester Maria (Maja) f. 1883. Endast 22 år gammal och nyligen utexaminerad från småskoleseminariet i Växjö tog hon sig an lärartjänsten i Påvelsmåla. Ensam ledde hon undervisningen i skolans alla klasser, ett fyrtiotal barn i olika åldrar. Skolan låg mitt ute i skogen, utan elektriskt ljus eller telefon och ett par kilometer till nästa granne. Där bodde hon ensam i skolans lilla kammare. Ester Maria utförde hela sin lärargärning inom Algutsboda skoldistrikt fram till sin pension vid 60 års ålder 1943. En av hennes elever var författaren Vilhelm Moberg och det var av henne han fick sänkt betyg i uppförande och ordning. Men han har också sagt att det var hon som fick honom att hitta böckernas värld.


Eva Sofia Nilsdotter insjuknar i lunginflammation och dör 1885.


Kristina Sofia Johansdotter börjar arbeta som piga hos Johan Peter 1887 11 01 och redan 1887 12 27 gifter de sig. De får 5 gemensamma barn:
Knut Gunnar f. 1889, hemmansägare i Sillre i Borgsjö socken. Ingrid Elisabet Kristina f. 1891, hon gifter sig med Emil Elvin som äger en pälsvaruaffär i Vetlanda. De säljer pälsar, hattar och mössor. Efter sin mans död 1927 så förestår hon affären själv. Carl f.1895, skräddarmästare och egenföretagare.
Carl Jonathan Johansson, född 17 januari 1895 i Stocksberg, flyttar till Bäck, Korsberga. Familjen består av föräldrarna Johan Peter och Marta Regina Johansson och fem äldre syskon. Carl blir faderlös vid 7-års-åldern. Pengar var det ont om och att köpa nya kläder till Carl hände sällan, han fick ärva det mesta av sina syskon , t.o.m flickkängor med klack av sin storasyster Ingrid.När han är 12 år arbetar han vid den kvarn-och sågverksrörelse vilken tidigare har ägts av hans far. Två år senare kom han i skräddarlära i sin hemsocken. Enligt sonen Carl-Eric lärde sig Carl att sy hos skräddare Ekholm som troligtvis var alkoholist, för när han inte kunde få tag på sprit så åt han skokräm, eftersom en av ingredienserna var just sprit. 1915 genomgick Carl svenska tillskärar-och yrkeskolan och får sedan en anställning i Axel Karlssons skrädderiaffär i Åseda.
 Erik f. 1898, mjölnare, chaufför och konstnär. Märta f. 1902, bor hemma hos modern. Hon träffar en man som hon förlovar sig med när hon blir gravid, men förlovningen bryts av Märta och hon stannar hos sin mor och föder en dotter 1924 som får namnet Maj-Lis. När Kristina Sofia dör 1930 står Märta ensam med sin dotter. Dessvärre är Märta inte psykiskt stabil och kan inte ta hand om vare sig själv eller barnet. 1931 finns det ingen annan lösning utan Maj-Lis flyttar till sin moster Maja, (Ester Maria) i Algutsboda. Hennes uppväxt blir mycket bra, äntligen får hon en kärleksfull familj som bryr sig om henne på alla sätt.
Märta flyttas till ålderdomshemmet i Korsberga där hon lever hela sitt liv.
Maj-Lis har ingen kontakt med sin mor, men hon åker till hennes begravning 1975. Det finns nog en liten önskan hos Maj-Lis att Märta hade undrat hur dotterns liv hade varit, så hon frågar prästen om modern någon gång hade frågat efter henne. Inte en enda gång, svarar prästen.


Johan Peter och Kristina Sofias äktenskap tar ett hastigt slut då Johan Peter dör 19011216.
Hon står ensam med 4 barn och även tre av sin mans barn i hans första gifte, men de flyttar hemifrån mellan 1903-1904.  Dessutom är Kristina Sofia gravid. De bor i en liten stuga och ingen info finns hur de klarar sig utan mannens inkomst. 


 


________________


Isak Pettersson


Född 18620314 på gården Hornaryds Lund
Föräldrarna hette Peter Nilsson och Maria Lisa Johannesdotter. När Isak är en dag gammal flyttar familjen till Galtabäck Fösingsgård, Nottebäck socken, Kronobergs län. Isak har två syskon: Sara Lisa f. 1859 och Mathilda Lovisa f. 1865.




Isak bor hela sin uppväxt på gården. När han gifter sig med Stina Maria bosätter de sig där också.


________________


Stina Maria Johansdotter


Född 18580717 på gården Spångaberg i Dädesjö socken, Kronobergs län. Föräldrarna hette Johan Nilsson och Helena Nilsdotter. Stina Maria har tre syskon: Carl Magnus f. 1845, (emigrerar till Nordamerika 1866), Peter f. 1848 och Johanna f. 1852. Deras mor, Helena, dör 18590313 och några år senare gifter Johan om sig och en flicka, Helena Louisa föds 1864.


Stina Maria flyttar 18850826 till Galtabäck Fösingsgård, Nottebäck socken,Kronobergs län. 
Hon har gift sig med sonen på gården, Isak Petersson, som har tagit över gården från sina föräldrar. 










Isak Pettersson och Stina Maria Johansdotter


De gifter sig 18851031 och familjen blir snart större när barnen föds: Johan 1886, Elna 1889, Gustaf 1890, Helga 1892, Linnéa 1894, Albert och Axel 1897 och Ottilia 1899.


Livet är inte alltid så lätt, sonen Axel dör bara ett år gammal 1898, samma år insjuknar de andra barnen i difteri. Isak tillkallar prästen som ordnar med medicin från doktorn. Av någon anledning så räcker inte medicinen till alla barnen utan Helga dör endast ett år gammal. Stina gick ofta till lilla Helgas grav ända tills en dag hon tyckte sig höra en röst som sa att hon inte skulle komm mer utan lämna henne ifred. Isaks dotter Elna har berättat detta för sin systerdotter Gullan Stéen och gett henne en knapp som ska ha suttit i Helgas kofta.


Isak är inte den lättaste man att leva med, han är inte så snäll mot Stina, behandlar henne mer som en piga än hustru. Han bryr sig inte om att sköta gården utan samlar hellre sina grannar i gården för att umgås och dricka alkohol. Gården börjar snart kallas för Gubbagården dvs. där gubbarna träffas och festar. 


1909 är gården körd i botten och familjen tvingas flytta därifrån. De hamnar i Klavreström stationssamhälle där Isak i församlingsboken står som slaktare till yrket.


Carls blivande fru, Linnèa är född 1894 i Gubbagården,i byn Galtabäck i Nottebäck socken. Föräldrarna heter Isak Pettersson och Stina Maria Johansson, de har sex barn.Gubbagården är Isaks föräldrahem som han har fått överta. Isak är tyvärr inte så mån om gården och dess skötsel , till slut måste familjen gå ifrån den. Namnet Gubbagården kommer från talesättet att det var den gården som gubbarna samlades i. Carl-Erics minne av sin morfar är att han mest låg och sov på soffan. Hela familjen flyttar sedermera till Stenhuset i Klavreström. Linnéa kommer som ung flicka till Åseda 1914.Hon har fått anställning som kammarjungfru hos kyrkoherde Bergdahl. En dotter,Gullan Stéen, berättar följande historia: Linnéa är i Folkets Park och står vid dansbanan, lite längre bort står tre flickor och pratar. Hon hör att de pratar om henne, en av flickorna, Astrid Spjut,säger: Det är prästpigan. Linnéa tycker att det låter nedlåtande,men hon får en liten hämnd. En tid senare är hon i hattaffären, hon ser en hatt som hon genast tycker om men får veta av expediten att Astrid Spjut har tänkt köpa den. Hatten kostar 6 kronor (mycket pengar på den tiden) men Linnéa tar sig råd med tanken att ”Astrid Spjut ska inte få den”.




 Linnéa flyttar till Växjö 1914 och arbetar som piga i en familj, åter till Klavreström 1916 och till Åseda 1917 där hon är anställd som piga i prästgården.
Äldste brodern Johan arbetade som verkstadsbokhållare vid Klavreströms Bruk. Han spelade flöjt i en orkester, han var intelligent och duktig i matematik. Han hjälpte Linnéas blivande man Carl med hans bokföring. 


Nästa bror, Gustaf arbetar som charkuterist. Albert är metodistpastor i Kristinehamn, Arboga och Malmö, han har t.o.m ett eget radioprogram där han håller sina predikningar. Han är gift med Ebba och har två barn, Gertrud och Sven. Det finns en historia om Sven och Carl-Eric, pojkarna hade lagt ut kräftburar och när det var dags att vittja dem, så hade de glömt att ta med något att förvara kräftorna i. De fick göra det bästa av situationen, så efterhand som de plockar upp kräftorna ur burarna så la de dem i sina kepsar. Det måste ha känts väldigt märkligt att ha levande kräftor krälande i håret  under kepsarna. Lillasyster Ottilia, ( som enligt Carl-Eric var en mycket högdragen person) gifter sig med en äldre man och bosätter sig i Värmland. Innan hon gifte sig arbetade hon på Stockholm slott.


Deras syster Elnas liv börjar bra, hon träffar Sigurd Pettersson som arbetar som konduktör. De gifter sig 1913 och bor i Växjö. Sigurd är en trevlig och gladlynt man utåt sett och det verkar vara ett lyckligt äktenskap för alla. Men en dag 1932 ska paret gå på promenad när Sigurd plötsligt säger att han har glömt något hemma och går tillbaka till hemmet. Elna väntar och väntar men han kommer inte. Hon går hem igen och när hon kommer innanför dörren så får hon se att han har hängt sig i trappan. Elna och hennes närmaste har ingen aning om varför han tog detta beslut.
Eftersom de har levt på Sigurds inkomst så blev livet efter hans död hårt för Elna i dubbel bemärkelse. Hennes man är borta och hon har inga pengar till sitt uppehälle, men hon finner på råd. Hon tar emot inackorderingar, stryker tvätt åt andra mot betalning och undervisar  som hushållslärare i sitt hem. Elna gifter aldrig om sig.


Isak och Stina bor kvar i Klavreström tills de dör, Isak 1942 och Stina 1944.








________________


Johan Peter Cederblad


Cederblad är ett soldatnamn och den förste i släkten är Johan Samuelsson född 1805 12 14 i Lillökna,Ökna.  Hans föräldrar hette Samuel Samuelsson, född 1775 06 01 i Karlstorp. Han var hälftenbrukare och kyrkoväktare och Lisa Johansdotter, född 1778 02 19 i Ökna. De får 5 barn, däribland Johan.1818 dör Lisa och Samuel gifter ganska snart om sig med Maja Stina Persdotter. I detta andra giftermål föds ytterligare 6 barn.


1825 07 06 blir Johan antagen som husar vid Smålands Husarregemente, Vetlanda Skvadron. Hans soldatnamn blir Cederblad. 1835 bor han på Emhults Skattegård Soldat nr. 7. Transporteras 1842 till Sandbro soldattorp, Repperda i Alseda. Soldat nr. 29 
Han är 5 fot 10 tum vilket omräknat blir 1,75 m lång.
Husar är en lätt kavallerist, beväpnad i första hand med sabel, dessutom pistol och karbin ( ett kortare luntlåsgevär eller flintlåsgevär ), ibland med lätt lans.


Johan gifter sig med Anna Svensdotter och de får 4 barn tillsammans. Johan Peter, Adolph, Maria Lovisa och Anna Mathilda. 1846 dör Anna och Johan gifter om sig med Annas tio år yngre syster Lisa Svensdotter. Med Lisa får Johan 8 barn.
Johan Cederblad dör 1866 09 11, Sandbro, Alseda.


Johan Peter föddes 1838 01 19 på husartorpet Sandbro, tillhörigt Repperda Sunegård, Alseda socken.
Efter att ha läst för prästen började han som lärling hos en skräddare i Skirö. Någon skräddare blev han dock aldrig, men sydde till sina söner senare i livet.






________________


Johanna Christina Lång
Carl Johan Pettersson  Glyck född 1789 03 02 i Övlandehult Backegård, Ökna.
Han blir antagen 1810 12 17 till Smålands lätta dragoner, kompani Vetlanda Skvadron, rote: Ämmaryd Björsagård, Alseda. Dragon nr. 40.
1833 blir han antagen till Smålands Husarregemente, Vetlanda Skvadron,rote: Ämmaryd Björsagård, Alseda. Han var 5 fot 7 tum ( 1,68m ).
Dragon, soldat som förflyttade sig med häst men som stred till fots. Beväpnad med musköter eller karbin samt blanka vapen.
Död 18690303 i Ämmaryd Björsagård, Alseda.




Lång är ett soldatnamn och den förste i släkten är Carl Johan Carlsson född 1818 05 05 i Ämmaryd, Alseda. Han blir antagen 1841 10 14 som grenadjär vid Smålands Grenadjärbataljon, Östra Härads Kompani, Skirö socken, Rote Snuggarp med efternamnet Lång. Grenadjär är ursprungligen (1700-talet), en soldat som vid strid hade till uppgift att kasta handgranater (franska grenad). En grenadjär skulle vara minst tre alnar lång (1,80 m ) och av ståtligt utseende. Senare används titeln om en infanterist vid ett elitförband.
Han var 6 fot lång vilket omräknat blir 1,80 m.
Dör 18770606 i Sköndal, Skönberga, Alseda.








Johanna Christina född på ett soldattorp tillhörigt gården Snuggarp (nuv. Skönberga) i Skirö socken 1843 07 01. Dotter till Carl Johan Lång och Anna Johansdotter född 1817 11 02. 




Johan Peter Cederblad och Johanna Christina Lång


Johan Peter och Johanna Christina gifter sig 1865 03 04 och de bosätter sig på Sjöarps Skattegård i Skirö socken, Jönköpings län. Deras första barn föds, Thilda Christina 1865 09 23 men redan samma år 6/11 flyttar familjen till Kleva Hytta, Alseda socken. Johan Peter har fått arbete på bruket Kleva Nickelverk, som drevs av Lessebro Bruk, med Bergsrådet Johan Lorentz Aschan som ägare. 




Deras barn:


Matilda Kristina Cederblad, f. 1865-1939.. Flyttar till Vidbo, Stockholm 1881. Gift med August Leonard Ludvig Lundberg. Han äger ett plåtslageri. Bor i Upplands Väsby med man och fosterdotter Ingbritt Maria Eleonora Cederblad, vars biologiska mor var Matildas syster Hilda.


Maria Lovisa, f. 1867-1914, flyttar till St. Jakobs församling i Stockholm 1885. Arbetar som tjänarinna på Elfsjö gård, Stockholm.


Anna, f. 1869-1938, arbetade på Upplanda Herrgård i Vetlanda hos familjen Adlercreutz, hon hade anställning där när den stora skandalen i familjen hände. Dottern i familjen var gift med löjtnant Sixten Sparre och de hade två barn, men han förälskade sig i  lindanserskan Elvira Madigan och rymde med henne till Danmark. När deras pengar tog slut, så såg de ingen annan utgång än döden. Sparre sköt dem båda med sin revolver. Denna händelse inträffar i slutet av 1880-talet så det blev en väldig uppståndelse. Dramat har även blivit film. Skrivbordet och det fina blå fatet som jag har nu  kommer från Annas ägodelar. Klockan som ligger i sin originalask kommer troligtvis från Anna också. Det lilla skrivbordet tillhörde Anna,gick i arv till Viva och senare till Lisbeth.




Karolina Josefina, f. 1871-1965, flyttar till Stockholm och får arbete hos Grosshandlare Brisman och hans familj. Senare bor hon hos en dotter i samma familj.
Efter Karolinas död, så åker Vivas kusiner till Stockholm i en hyrd lastbil och hämtar hennes ägodelar. Senare hålls en auktion i Vetlanda.


Carl-Oskar, f. 1874-1956, maskinist. Gift med Anna Lovisa Carlsdotter, 7 barn.


Johan Henrik, f. 1876-1901.


Johanna (Hanna) Augusta, f. 1878. Gift med Carl Fredrik Köhler, 5 barn. De tre första dör unga i Sverige. Familjen emigrerar till Amerika.


Gustav Adolf, f. 1881-1963, maskinist. Gift med Sigrid Virena Emilia Reik, 6 barn.


Claes Ferdinand, f. 1883-1884.


Hilda Alma Elisabet, f. 1886-1977. Ogift, 2 barn. Ett av barnen blir fosterdotter hos Hildas syster Matilda Kristina. Hilda flyttar runt ofta, först till Eksjö Stadsförsamling 1906, sen till Eksjö Ränneslätt 1907. Nästa anhalt är Barnarps prästgård 1908,  22/10 1909 går färden till Säby, kvarteret Berget 102. Där blir hon gravid, åker hem till Ädelfors och föder en son, Carl Ragnar 1910, åker tillbaka till Säby med pojken. 12/5 1911 återfinns Hilda med son i en lista med rubriken
”Personer utan fast bostad”. Hon  flyttar till Hammarby 5/3 1913 och sonen kommer efter 16/1 1914. Ännu en gång blir hon gravid  och får en dotter, Ingbritt Maria Cecilia 10/2 1915. Hilda är inte gift, men en man som heter Karl August Lundgren ”låter anteckna sig som barnets fader”. Han är rättare i Björkboda.
Hilda arbetar som tvätterska i ett tvätteri som hon har ihop med syster Matilda. Detta har de i sitt hem och tvättar i bäcken som rinner förbi i närheten av huset. De hämtar tvätt hos ”finare” familjer och lämnar tillbaka den tvättad, struken och manglad.


Lisbeths morfar, Anders Fritiof Cederblad, född 21/10 1888 i Kibbe, nuvarande Ädelfors. Gick i Kibbe skola och konfirmerades i Alseda kyrka den 24/4 1903. Gifte sig 23 maj 1913 med smeddottern Hilma Teresia Caesar från Bärbäcksbron i Alseda.
Gjorde värnplikten vid Kalmar Regemente på Hultsfreds slätt, 30/4 – 27/12 =240 dagar.


Två eller tre somrar reste Johan Peter till Stockholm och arbetade som byggnadsarbetare.
Det var kanske bättre betalt än arbetet vid bruket. Han fick då gå, eller kunde kanske få åka med någon forbonde som körde till Oskarshamn för att hämta varor. Därifrån åkte han med båt till Stockholm. Någon järnväg fanns ju inte på nära håll. Han hade också varit rallare vid Södra Stambanan.


Han fick 1896 Patriotiska Sällskapets medalj för långvarig trogen tjänst, i samband med att Lessebo Bruk AB lade ner sin verksamhet i Kibbe, och sålde sina gruvor och egendomar i Östra Härad, till en tysk konsul Bieber, som tidigare under tre år arrenderat och drivit Guldgruvan. Han bildade nu Ädelfors Guldverksaktiebolag för fortsatt drift av Ädelfors Guldgruva.


Johan Peter fick ett årligt bidrag från ``Stenkvarnefonden'', en fond för gamla bruksarbetare, förmodligen startad av Lessebo Bruk AB, men förvaltades av Bergmästareämbetet, dit ansökan fick insändas.




Johanna Christina var känd för att vara något av "klok gumma", där både skrock och egenhändigt tillverkad örtmedicin ingick.


Hilma Cederblad, gift med sonen Anders, har berättat att när hon beklagade sig för att den några månader gamle sonen inte ville sova på nätterna utan skrek och var besvärlig, så sa Johanna Christina att det kan du väl få något för. Så kom hon med en liten påse med okänt innehåll, som skulle hängas i ett band om halsen på pojken. Det skulle absolut hjälpa. När Hilma tvekade att använda "medicinen'' så blev svaret : “ Är du så dum så du inte tror på det, så må du ha att [besväret]."


Blev mycket omtalad i hela bygden då hon lyckades bota en av stationsinspektor Axel Kjellins döttrar som drabbats av "skerva", engelska sjukan, med sin dekokt. De hade tidigare sökt läkarhjälp utan resultat.


Johanna Christina dör 19191121 och Johan Peter 19260520.




________________


Johan August Caesar
Soldat Zachris Månsson Ekholm levde mellan 1750-1810. Bodde med fru och barn på soldattorpet nr. 69 Bistockskulla, Ekholma i  Åseda socken. Ett av deras barn heter Magnus Zachrisson, han föddes 1786 och ca 1807 blir antagen till soldat vid Kalmar Regemente, tredje majorens kompani Aspeland, Järeda. Nu får han soldatnamnet Caesar.
Han var 6 fot 3 tum, dvs 1,88 m lång.
Några år senare bodde han med fru och 8 barn på soldattorpet nr. 80 Tynshult, Virserum.
Anteckning i husförhörsboken: “ Död vid arbete vid Göta kanal, kommenderad att arbeta vid kanalen men dog dagen efter sin ankomst 7 maj 1830. Begravd på militäriskt vis. “
Dödsorsaken var lungsot, dvs. tuberkulos.


En av hans söner hette Carl Johan Caesar, levde mellan 1812-1882. Han arbetade som pappersmakargesäll vid Qvills pappersbruk i Ökna. Gift med Anna Maria Magnidotter som levde mellan 1816-1883.
De fick barnen: Gustav Adolf 1841, Gustaf Martin 1844, tvillingarna Per Alfred och Johan August 1846, Carl Peter Alfrid 1849, Frans Algot 1852 och Gustaf Adolph 1855.
Fyra av barnen dör i tidig ålder och två emigrerar till Nordamerika.










Johan August född 1846 03 31, Qills Pappersbruk Ökna. Hans föräldrar och syskon flyttar till Källedal,Trollhälla i Alseda socken 1847 och efter några år där till Sandslätt på Trollhällas ägor i Alseda.




1866 04 01 åker Johan August till Stockholm och arbetar där till november. Återvänder hem, men redan 7 november flyttar han till Lindshammar, Nottebäcks socken och arbetar som dräng hos en man som är anställd som smed på järnbruket. 18681207 flyttar han hem till Sandslätt igen.










Johanna Lovisa Johansdotter
Född 13/11 1845, i Godanstorp, Alseda socken.
Gården har varit i släktens ägo sedan 1806 när Johannas mormors far kom hit.
Hennes föräldrar heter Johan Henrik Svensson och Anna Lisa Johansdotter, hon växer upp med sex syskon. Två systrar och en bror emigrerar till Nordamerika. Äldste brodern tar sedermera över gården från fadern. Gården är i släktens ägor till 1925.
Johanna bor hemma tills 1870 då hon tar tjänst som piga på gården Apelhestra Wästergård, arbetar där ett år. 


Bild på gården














Johan August Caesar och Johanna Lovisa Johansdotter
Johan August gifter sig 1/12 1871 med Johanna Lovisa Johansdotter , från Godanstorp Alseda. 


1876 köpte han stugan på Broatorpet (Bärbäcksbron) som låg på Trollhälla gårds marker. Han hade tidigare fått löfte att bygga en stuga i närheten, men då torparen avflyttat och den stugan blev ledig, avstod han från att bygga. Priset för Bärbäcksbron blev 150 kr. Trollhälla gård ägdes av Alseda socken och där fanns även socknens fattiggård. Troligen byggde han i samband med detta smedjan nere vid brofästet till bron över Emån.


Köpekontrakt


Förutom det vanliga “bondsmidet” med reparationer och skoning av hästar och dyligt, tillverkade han diverse verktyg som potatishackor, lövhackar och “riskar” som han sedan sålde på torget i Vetlanda. Åkte då med någon bonde som hade egna ärenden till torget. Hans dotter Hilma har berättat att när han kom hem från torgresan så tömde han ut smålandspungen på spiselhällen, och räknade noggrant ihop dagskassan. Det var väl inte alltid så lätt att få slantarna att räcka till och han beklagade sig ibland. ``En är så gällskyldig så en vet inte ut vart in en ska vända sig.'' Gällskyldig betyder skuldsatt.


Deras barn:


Karl Johan Levi Caesar född 16/5 1872 emigrerar till Amerika.
Frida Lovisa Josefina Caesar född 15/5 1875 emigrerar till Amerika.
Gustav Alfred Irenius Caesar född 15/12 1877 emigrerar till Amerika.
Helga Emilia Caesar född 17/11 1880. Gift med snickaremästare Sven Lindberg i Slättåkra. Han tillverkade hyvelbänkar och slöjdbänkar.
Hjalmar Fredrik Caesar född 31/10 1883. Smed. Gift med Gulli Maria Teresia Franzén.
Emma Vicktoria Caesar född 16/1 1887. Gift med August Elof Karlsson 1916, innan dess har hon arbetat som tjänarinna  på Vagnhester Kåragård.
Hilma Teresia Caesar född 27/7 1890. Gift med Anders Cederblad 1913.
Hilma Caesar född den 27/7 1890 i Bärbäcksbron, Alseda. Familjen består av hennes föräldrar, Johan August Caesar och Johanna Lovisa Johansdotter, 7 barn varav Hilma är yngst.
En syster och två bröder emigrerar till Nordamerika.




Gick i Alseda kyrkskola, och konfirmerades i Alseda kyrka.
Därefter blev det att ta anställning som piga på olika gårdar.
Hon arbetade på Vallsjö gård, Sävsjö, Nyaby Berg, Slättåkra, hos Carl Zakrisson Repperda, och hos Emil Gustavsson Repperda Sunegård.




Johan August dör 1903 1018 av vattusot, dvs. ödem i kroppen.


När Johan August dör 1903 övertas smedjan av sonen Hjalmar Fredrik. Han gifter sig med Gulli Maria Teresia Franzén 1917 bygger om och till stugan i Bärbäcksbron. Johanna får dock bo kvar där några år. Men lönsamheten i smedjan blir allt sämre, och han måste söka arbete på annat håll. Stugan i Bärbäcksbron säljs 1923 och Johanna flyttar till dottern Hilma Cederblad i Ädelfors. Efter en tid där flyttar hon till dottern Emma Karlsson i Slättåkra, där hon bor några år innan hon får sluta sina dagar  23/6 1930. Hon är då åderförkalkad och har hjärtfel.




(Bouppteckning)
-----------------------------------------------------------
År 1904 den 16 Januari instälde sig undertecknade i Bärbäcksbron under Trollhälla i Alsheda Socken, att förrätta bouppteckning efter aflidne Smeden Johan August Caesar, som afled den 18 Okt 1903 och efterlämnade Enkan Johanna Lovisa Johansdotter samt 6 med henne ägande barn. Sonan Gustaf Alfrid Irenius, sonen Hjamlar Fredrik myndiga och närvarande, dottren Frida Lovisa Josefina, som vistas i Norra America, dottren Helga Emelia myndig, dottren Emma Victoria född den 16 Januari 1887 och dottren Hilma Theresia född den 27 Juli 1890.


Som ingen godvilligt åtog sig godman- och förmyndareskapet, för ofvannämnda dottren Frida Lovisa Josefina och omyndiga döttrarna Emma Victoria och Hilma Theresia får vi vördsamt anhålla att Häradsrätten täcktes såvidt möjligt tillsätta någon lämplig sådan. Enkan uppgavf boet, sådant det befanns, då mannen afled, samt uppvisade alla boets handlingar, då mannen afled, samt uppvisade alla boets handlingar och skrifter, hvarefter uppteckning företogs i följande ordning:




	Kron
	Öre
	Kron
	Öre
	Koppar
	

	

	

	

	2 st. krus 1kr 2st. kaffekokare 1,25 skopa 25 öre
	2
	50
	2
	50
	Möbler
	

	

	

	

	1 st. Chiffonier 5 kr. 2 st. dragkistor 4.50 
	9
	50
	

	

	2 st. sängar 2 kr. 2 st. soffor 2 kr. 3 st. slagbord  1 kr.
	5
	-
	

	

	2 st rundbord 3 kr. 1 st. skänk 2 kr. 1 st. byrå 50 öre.
	5
	50
	

	

	1 st. väggur 5 kr. 2 st. speglar 2 kr. 4 st. stolar 4 kr.
	11
	-
	

	

	4 st. (gamla) stolar 50 öre. 4 st. väggtaflor 2 kr. 
	2
	50
	36
	00
	Säng- och Linnekläder
	

	

	

	

	3 st. madrasser 2 kr. 4 st. dynor 2 kr. 
	4
	-
	4
	-
	(Transport Kronor)
	

	

	42
	50
	2 st. täcke 2 kr. 1 st. filt 5 kr. 3 st. lakan 2 kr. 
	9
	00
	

	

	2 st. dukar 3 kr. 2 st. servietter 1 kr.
	4
	00
	

	

	2 st. handdukar 50 öre. 3 par gardiner 1,50 kr.
	2
	00
	15
	00
	Gångkläder
	

	

	

	

	1 st. Öfverrock (gammal) 1 kr. 1 svart kostym
	6
	00
	

	

	1 rock, byxor 4 kr. 1 st. undertröja mössa 2 kr.
	6
	00
	

	

	1 par stöflar 1 par skor 3 kr. diverse gångkläder 30
	33
	00
	45
	00
	Glas och porslin
	

	

	

	

	2 st. karaffer 75 öre 6 st. glas 50 öre halvt dus kaffekoppar
	2
	25
	

	

	1 st. kaffekanna 50 öre 1 sockerskål gräddkopp 25 öre
	

	75
	3
	00
	Diverse saker
	

	

	

	

	2 st. lampor 1,50 1 st. gryta 3 st. klampar 2 pannor
	2
	00
	

	

	1 st. brännare 2 mortlar 50 öre 1/2 knifvar och gafflar
	1
	50
	

	

	1/2 dus matskedar 1/2 kaffeskedar
	3
	00
	

	

	1 sats 1 bricka 25 öre 2 par kardor 25 öre
	

	50
	

	

	2 st. bleckkärl 1 kr. mattor 1 kr. 1 eldtång
	2
	10
	

	

	3 st. bistånd 30 kr. 8 st. bikupor 6 kr. 
	36
	00
	

	

	1 st. koffert 4.50 1 st. saltkar 1 kr. 1 kimma 25 öre
	5
	75
	

	

	diverse blusar 1 kr. halsdon 2 kr. 3 skjortor 3 kr. 
	6
	00
	

	

	rakdon 1 kr. böcker 2 kr. 
	3
	00
	

	

	1 stugbyggnad 50 kr. 1 smedja 10 kr.
	60
	00
	

	

	smedverktyg 30 kr. diverse småsaker 2 kr.
	32
	00
	151
	85
	Summa Inventarier Krona
	

	

	257
	85
	

Under edlig förpligtelse att icke något är med vett eller vilja dold i eller utelämnat, utan riktigt uppgifvet underskrift
Bärbäcksbron som ofvan
Johanna Johansdotter
Rederligen upptecknadt och värderadt intyga
C.J. Hagberg                O.E. Rudvall
Bouppteckningsman
Närvarande vid förestående förrättning
Gustaf Caesar                Hjalmar Caesar
-----------------------------------------------------------
________________


________________












Carl Johansson och Linnéa Isaksson


Carl och Linnéa träffas troligen i prästgården. Linnéa arbetar ju där och Carl har flyttat till Åseda 16/09 1917 och hyr ett rum hos kyrkoherden. Tycke uppstår och de gifter sig 12/12 1917. Deras första hem ligger på Karlavägen, vid bokhandeln bakom missionshuset.
1923 åker han till Stockholm till J-E Kollbergs tillskärarskola. Hans titel blir skräddarmästare.Samma år startar han sin egen skrädderirörelse i Åseda. 1942 köper han en fastighet på Kexholmen, Åseda där han också har en kappaffär. Carl är en av initiativtagarna till Åseda Hantverks-och industriförening, tillhör styrelsen. Han är också med i Kronobergs län och Östra Smålands skräddar mästareförening.Ett annat uppdrag är ledamot i kyrkofullmäktige. Det sista uppdraget blir ordförandeskapet i Åseda Pensionärsförening, han avgår där 1970.


Den blå urnan som stod på vitrinskåpet i stora rummet hos Carl-Eric och Viva kommer från Åseda. Den har en liten historia: När Carl hade skrädderi-och kappaffär så kom det in en dam från Stockholm och beställde två kappor, som skulle skickas till henne. Kapporna syddes och levererades till damen. Nu visar det sig att hon inte kunde betala dem med pengar utan hon skickar den blå urnan som betalning, hon ägde en antikvitetsaffär. Vi har i alla år trott att den skulle vara värdefull, men nu har vi värderat den och den är bara värd ca 400 kr. Det var ingen bra affär för farfar.


 De får fyra barn, Maj-Britt 1918, Ann-Margret 1920, Carl-Eric 1921 och Gullan 1926. Efter några år flyttar familjen till det som kallas Blombergahuset, se bild.
Född 19 mars 1921 i Åseda. Föräldrarna hette Carl och Linnéa Johansson.  


Carl-Eric hyser en dröm att bli postiljon (brevbärare) men efter konfirmationen börjar han arbeta på ett plåtslageri i Åseda. Den anställningen blir inte så lång, han har dessvärre väldigt svårt för höjder och att lägga plåttak är inte att tänka på. Första arbetet är på ett kyrktak, så den karriären blev kort. Dessutom trycker  Carl på att han ska lära sig till skräddare och ta över rörelsen så småningom. Carl är oerhört duktig i sin yrkesroll, allt ska vara perfekt innan han godkänner något. C-E har berättat att om det så var bara ett stygn som inte var som det skulle i Carls ögon så tog han tag i sömmen och rev isär den.
Carl-Erics syskon:


Maj-Britt, 1918, gift med Torsten Johansson. En son, Roland.


Ann-Margret, 1920, gift med Folke Johansson. En son, Lennart.


Gullan, 1926, gift med Lars Stéen. Tre döttrar, Margareta och tvillingarna Kristina och Katarina.
1942 köper de hus i Kexholm, där Carls skrädderi och kappaffär finns på nedervåningen.


Ann-Margrets son, Lennart, har berättat några minnen om Linnéa för mig. Hennes far var väldigt glad i sprit och gick gärna i god för hans vänner när de behövde pengar. Till slut gick det så illa att familjen blev tvungna att gå ifrån gården och flytta till Klavreström. Han var nog inte den bästa maken och pappa.
Med detta som bakgrund så fanns det ingen sprit i Carl och Linnéas hem, med ett undantag. I en kakelugn i ett sovrum som inte användes hade Linnéa en flaska konjak och när hon kände att en förkylning på gång eller liknande så hämtades flaskan och hon hällde upp konjak i en sked och tog den som medicin. Det var mycket viktigt att inte använda ett glas, för då var det inte medicin, utan för nöjes skull. Det var Lennarts pappa Folke som fick hämta ut konjaken, varken Carl eller Linnéa tyckte att de kunde göra det själva.
Lennart har ett litet bord med en utdragslåda i sin ägo, det var Linnéas pappas matbord och i lådan hade han sina bestick. Vid detta bord åt han sina måltider ensam, resten av famljen fick stå och äta.








Anders Cederblad och Hilma Caesar


Under tiden i Repperda blev hon medlem i Logen Ädelfors Framtid av NOV, som fanns i Ädelfors i början av 1900-talet. Medlem var också Anders Cederblad från Ädelfors. Kanske var det på logemötena som de lärde känna varandra.
De gifter sig i maj 1913 och den första åren får de bo hos Anders föräldrar som har en lägenhet i ett flerfamiljshus i Ädelfors. Huset kallas Smedstugan som ligger en bit förbi Stenhuset in på vägen till Ökna. Det var arbetarbostäder för personer som arbetade vid Kleva Nickelverk.
I september 1913 föds en son, Georg, 1916 ytterligare en son, Henry. En dotter föds 1918, Berta Viola. Hon får inte leva mer sex månader, spanska sjukan tar hennes liv. 
Georg berättade för mig att han mindes sin lillasyster, han var ju 5 år när hon föddes.
Det var en tung sorg, för efter Bertas död talades det aldrig om henne. Hilma nämnde henne aldrig mer och då gjorde ingen annan det heller.


Arbetade som smältare vid masugnen i Ädelfors 1914 – 1918 där han förlorade sitt ena öga vid  olycksfall i arbetet. Han fick ett stänk från smältan i ögat och synen gick ej att rädda. Arbetsförmågan ansågs nedsatt till 25\% och han tillerkändes en livränta på 300 kr/år som utbetalades med 75 kr i kvartalet. Prästbevis att han fortfarande fanns måste skickas in 2 ggr/år.
1919 upphörde verksamheten vid masugnen i Ädelfors.
Gården Germunderyds Ebbagård som varit i Brukets ägo såldes till AB Kalmar Kol och Trävaruaffär, som genast satte igång med avverkning på den skogrika gården. Då blev det skogs-och sågverksarbete och även kolning förekom. Anders fick arbete där och senare blev han kantare vid sågen i Ädelfors.
1932 lade Kalmar Kol och Trävaruaffär ned verksamheten och sålde Germunderyds Ebbagård till Domänverket.
En bit in på 1930-talet blev det dåliga tider, med stor arbetslöshet och därmed svårt att få ett arbete. Anders arbetar med stenröjning på åkrar i Möcklarp, byggnadssnickare, fick arbete som kantare på olika sågverk bl.a vid Carl Blomstrands såg i Holsbybrunn ett par vintrar.
Under krigsåren 1940 – 1945 högg han meterved på Domänverkets gård i Germunderyd. Sommaren 1945 började han arbeta på Bröderna Granlunds Båtbyggeri i Ädelfors där han stannade till semestern 1958, samma år som han fyllde 70 år.
Georg har intervjuat Anders om hans yrkesliv, det finns sparat på en cd som jag har. I bakgrunden( om man lyssnar noga ) kan man höra mormor Hilmas röst ibland och ljudet från en väggklocka.


1920 flyttar hela familjen till Brostugan, även kallat “huset vid åbron”. Det finns fortfarande kvar, ett rött tvåvåningshus med vita knutar som ligger vid bron innan man kommer till Stenhuset. 
Anders pappa ( hans mamma dog 1919 ) Johan Peter Cederblad och nu även Hilmas mamma, Johanna Lovisa Johansdotter, bor här också. 
9 november samma år föds en dotter, Viva och livet blir nog lättare för både Anders och Hilma.
Viva har berättat att hon ofta gick till sin mormor till hennes rum, Hilma hittade henne alltid där glatt mumsande på något gott.
Hilma var som vanligt var på den tiden hemmafru, skötte hem, barn och de gamla föräldrarna. Skaffade sig symaskin, och var duktig att sy både till sig själv och barnen. Lånade ibland vävstol och vävde trasmattor till hemmet.


Under några år i början av äktenskapet hade man också hushållsgris som slaktades strax före jul. Att ta hand som slakt hade hon ju lärt sig på gårdarna hon varit på. Så det blev blodpalt och pölsa och flera sorters korv, pressylta och rullsylta och julskinka lagades till. Fläsket saltades ned i trätunnor för kommande behov.


1927 tar Anders ett inteckningslån på 2000kr och tillsammans med sin syster Anna Gustava som går in med 2000 kr, med löfte att få bo på övervåningen om ett rum och kök så länge hon lever. De köper ett gammalt timmerhus av Emil Edborg i Repperda, som får utgöra stommen i det nya huset. Det var dock längre än det nya skulle bli, så därför tog man bort en bit på mitten och fogade ihop de båda halvorna. Husdelarna transporterades med häst och vagn till Ädelfors.




Pengarna räcker tyvärr inte till att köpa tomten där huset ska stå, utan de hyr den av Ädelfors bruk. Anders tog ledigt ett par sommarmånader och med hjälp av två snickare så byggdes det nya huset som får namnet Cederslund. Den 22/12 1930 lyckas han dock med att köpa tomten, lagfart blir beviljad 29/10 1931. Anders renoverar huset, men det förblir med dagens mått mätt rätt så omodernt. Enbart kallt vatten indraget, kakelugnar och vedspis som värmekälla och till matlagning. Utedass och och inget badrum. Kylskåp och frys fanns inte heller utan de flesta kylvaror förvarades i en matkällare. Det som behövdes dagligen fick plats i ett skafferi i köket. Så här ser det ut ända in på 1970-talet.


Skolhuset ligger bredvid Cederslund så Viva behöver inte gå långt för att komma dit. Hon går i skolan i 6 år och efter det en skolkökskurs.
Bild och hennes berättelse.


På 1930-talet, om man ville åka på semester och det inte fanns så mycket pengar, då åkte man på cykelsemester. Så gjorde även Viva och hennes vänner. Bl.a cyklade de till Tranås en sommar. Hennes bröder Georg och Henry, samt en kamrat, cyklade till Gripsholms slott och fortsatte till Trollhättan.Väl där skickade de ett vykort hem och skrev att de skulle ta en ”avstickare” till Göteborg innan de kom hem igen. De cyklade också till Kalmar, vidare till Karlskrona för att slutligen hamna i Ystad som var målet med resan. De skulle titta på en utställning där.
På sina cyklar hade de all utrustning som de behövde såsom tält, mat, fotogenkök, kläder m.m


1936 hände något som alla i Ädelfors talade om länge. Viva hade lämnat in en tipskupong och vann högsta vinsten. 15 040 kr och 67 öre. Det var en hel förmögenhet på den tiden. Hon hjälpte sina föräldrar att betala av lån på huset och bekostade sin egen utbildning till damfrisörska. Denna utbildning fanns i Vetlanda och troligtvis bodde hon där i veckorna. Hon kallades därefter för Lyckaflickan i Ädelfors.
Bild på tipskupong


Viva flyttar till Åseda 23 november 1939, hon har nu fått sitt första arbete på Fru Johanssons damfrisering. Hennes bostad är en liten lägenhet ovanför salongen. Hennes lön är 30 kr/mån.
Ibland var det nog inte så roligt att ta hand om kunder. Det var sällsynt med damer som hade tvättat håret innan de kom för att bli fina. Det var inte ovanligt att hon fick börja med att avlusa och tvätta håret innan hon kunde påbörja klippning och permanentning.


Bild på huset.
 Fram tills nu har Viva inte haft tillgång till sitt bankkonto, hon är ju inte myndig. När hon behöver pengar måste hon ta kontakt med sin pappa Anders som i sin tur får ta bussen in till Vetlanda och ta ut pengar på banken, som Viva sedan får från honom.
1940 blir hon erbjuden att köpa damfriseringen och nu skickas det många brev mellan Åseda och föräldrarna i Ädelfors hur hon ska göra. Det diskuteras fram och tillbaka, Viva får nog både goda råd och förmaningar att tänka igenom allting noga, innan hon tar ett beslut.
I oktober samma år köper hon salongen av Fru Johansson för 4000 kronor och anställer en flicka som hjälpreda.


Några händelser från brevväxlingen mellan Hilma och Viva:
1940: Missionsauktion i Repperda ( Missionshuset är det gula huset uppe på höjden till höger om vägen in till Ädelfors). Hilma ropar in två hemostar, 6 och 7 kr kilot och 1 kilo kaffe för 5 kr och 50 öre. 
Nu är det tilldelta kuponger för varje familj för nästan för nästan allt man behöver köpa.


1941: Januari: vattenledningen i Cederslund har frusit, minus 33 grader och inte mycket snö.
          Februari: Hilma och Viva byter matvaror med varandra t.ex Viva skickar socker till sin 
          mamma och får ägg istället.
          Vädret har blivit mildare och nu är det mycket snö.
          Mars: Georg och Henry har fått uppgiften att snickra ett bakbord till Viva. Enligt Hilma så 
          var de lite slöa med att få det färdigt, så Hilma fick prata med dem “på skarpen”.
          April: 18.e, Hilma är mycket när hon har pratat med Viva i telefon. Tydligen har Vivas 
          anställda hjälpreda sagt upp sig utan förvarning. Några dagar senare har en ny flicka  
          Några dagar senare har en ny flicka anställts och Hilma är glad igen. Hon tar tåget till 
          Åseda och stannar över natten.
          Maj:










Anders dör 28 juli 1970 på Eksjö Lasarett. Georg och Henry åker till Eksjö och är med honom till slutet.




Hilma råkade våren 1971 ut för en olyckshändelse (lårbensbrott). På morgonen denna dag körde brevbäraren sin vanliga runda tidigt på morgonen. Hilma hade fått post denna dag så han skulle bara lägga den i brevlådan, men tänker i samma stund att “ det var länge sedan jag träffade Hilma, jag går in med posten så kan jag prata med henne en stund och höra hur hon har det”. När han kommer in i köket så ligger Hilma på golvet och har brutit lårbenet. Troligtvis går han in till Georg som bor tvärs över vägen och ambulans tillkallas. Ingen vet varför brevbäraren just denna dag bestämmer sig för att gå in till Hilma, för det tillhör inte vanligheten att göra det, men vilken otrolig tur att han gjorde det. Efter lasarettsvistelsen i Eksjö, flyttas hon till äldreboendet Österliden i Holsbybrunn. På egen begäran fick hon sedan stanna kvar där till sommaren 1981, då hon insjuknade och fick komma till Vetlanda Sjukstuga, där hon avled den 12/10 1981.








Hilmas syskon:


Karl Johan Levi Caesar född 16/5 1872 emigrerar till Amerika.
Frida Lovisa Josefina Caesar född 15/5 1875 emigrerar till Amerika.
Gustav Alfred Irenius Caesar född 15/12 1877 emigrerar till Amerika.
Helga Emilia Caesar född 17/11 1880. Gift med snickaremästare Sven Lindberg i Slättåkra. Han tillverkade hyvelbänkar och slöjdbänkar.
Hjalmar Fredrik Caesar född 31/10 1883. Smed. Gift med Gulli Maria Teresia Franzén.
Emma Vicktoria Caesar född 16/1 1887. Gift med August Elof Karlsson 1916, innan dess har hon arbetat som tjänarinna  på Vagnhester Kåragård.


________________


Carl-Eric Johansson och Viva Gunhild Maria Cederblad




C-E och Viva träffas så småningom och blir ett par. Detta är under pågående 2:a världskriget så C-E blir inkallad som beredskapsman och blir stationerad i Skåne. Paret förlovar sig och innan bröllopet blir det björkdrag i Ädelfors. Detta är en väldigt gammal tradition som går ut på att brudparets vänner på lysningsdagen kommer dragandes på en stor björk och överlämnar den. Av björken ska sedan brudgummen tillverka någon möbel till det framtida egna hemmet. Det blev Vivas bror Georg som svarvade ljusstakar och nötknäckare av björken.
23 oktober 1943 vigdes C-E och Viva i prästgården i Åseda av prosten d:r Isak Krook. Brudpar och släkt åkte sedan till Ädelfors för bröllopsmiddag hos Vivas föräldrar Anders och Hilma Cederblad. Faster Gullan minns att Vivas bror Henry spelade dragspel. Vid hemkomsten till Vivas lägenhet så blir brudparet och den närmsta släkten bjudna på kaffe av Vivas granne, Gerda Olsson.


Efter vigseln måste Viva sälja sin salong och börja sitt nya liv som gift. Det var ovanligt att kvinnan fortsatte att arbeta när hon hade gift sig.


Vissa omständigheter gör att deras första tid som gifta blir omvälvande för båda två. Viva är gravid i femte månaden och måste nu sälja sin damfrisering och flytta från sin lägenhet. Carl och Linnéa ger dem husrum i sitt hus, ett rum och kök i en tillbyggnad mot trädgården.
 Det måste ha varit en svår lärotid, tills C-E själv blev skräddare. Han fortsatte att arbeta för sin far ända till i början av 1960-talet då han fick ny anställning på Garpens konfektionsfabrik. På sin fritid ägnade sig C-E åt fotboll, ishockey, bandy och tennis. På den här tiden (1930,-40-50 och -60 talen) fanns ingen särskild anställd som klippte gräset på fotbollsplanen eller på vintern spolade is i ishockeyrinken m.m. Utan det var de aktiva inom vardera sport som fick hjälpa till med det. Det var även problem med utrustningen när pengarna inte räckte till nyköp. C-E hade t.ex inte råd att köpa sig ishockeybenskydd så han sydde ett par istället.


Jennys intervju av C-E om hans fotbollsliv.
 C-E får inte tillräckligt med lön av sin far för att kunna försörja sin familj utan måste låna av Vivas sparade pengar. De lägger upp en avbetalningsplan som C-E lyckas hålla. I mars 1944 föds Bengt och det dröjer ända till maj 1952 innan nästa barn, Lars, utökar familjen. Någon gång mellan 1952 och 1956 flyttar de till en alldeles ny hyreslägenhet på Södra Esplanaden i Åseda. I december 1956 föds en dotter Lisbeth.
Slutet av 1964 har C-E fått ett arbete som förman på Bestons konfektionsfabrik i Berga och flytten går denna gång till Högsby.


Arvegods från Carl-Eric och Viva: guldringen med en liten vit sten fick mamma av pappa en gång för länge sedan. Hon hade den alltid mellan sin förlovnings- och vigselring. Glasskålen på fot fick Anders och Hilma i lysningspresent 1913. Taklampan som jag har är köpt i Åseda.
Stekgrytan med mammas namn och ett årtal på locket är en lysningspresent från min farmors syskon. Den är tillverkad i Klavreströms Bruk.
Halsbandssmycket i form av en gulddroppe som Jenny har är ursprungligen min farmors ringar.




Vivas syskon:


Georg Cederblad, född 24/9 1913 i Smedstugan, Ädelfors. Gift med Ingrid. Två barn, Åke och Kristina.
Georg första arbeten får han i skogen och på ett sågverk i Ädelfors. 1935-1948 arbetar han i Källströms möbelfabrik och sedan hos Bröderna Granlund. 1952 till sin pensionering i Hermanssons möbelfabrik i Fluguby.
Hans fru, Ingrid, arbetar som sömmerska i en syateljé som ägs av Sjöberg.
Georg byggde hus mittemot Cederslund och kallade det för Lillevång, (nästan alla hus i Ädelfors har ett namn).


Henry Cederblad, född 29/6 1916 i Smedstugan, Ädelfors. Gift med Inga.
Han arbetar liksom som sin bror i Hermanssons möbelfabrik i Fluguby.
Hans fru, Inga, övertog tjänsten som stationsföreståndare 1934 på Tällängs station efter sin far, Oskar Tapper. Han hade dock ansvaret för stationen tills Inga blev myndig. 1937 får Oskar nytt arbete på Ädelfors station där han arbetar till 1943. Inga övertar hans arbete och blir kvar till 1961. Poststationen dras in då persontrafiken läggs ner på sträckan Vetlanda-Gårdveda. Inga förflyttas till Alseda station och arbetar där som platsvakt fram till 1966 då också den poststationen dras in. Hon får anställning på posten i Vetlanda där hon är kvar till sin pension.

\bye