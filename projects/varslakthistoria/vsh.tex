\input vsh-macros

{\footline{}%
\input vsh-titlepage
}

\person{Johan Peter Cederblad}

%\rightleaf\ 
Cederblad är ett soldatnamn och den förste i släkten är Johan Samuelsson född 14 december 1805 i Lillökna, Ökna. Hans föräldrar hette Samuel Samuelsson, född 1 juni 1775 i Karlstorp. Han var hälftenbrukare och kyrkoväktare och Lisa Johansdotter, född 19 februari 1778 i Ökna. De får fem barn, däribland Johan.

1818 dör Lisa och Samuel gifter ganska snart om sig med Maja Stina Persdotter. I detta andra giftermål föds ytterligare sex barn.

6 juli 1825 blir Johan antagen som husar vid Smålands Husarregemente, Vetlanda Skvadron. Hans soldatnamn blir Cederblad. 1835 bor han på Emhults Skattegård Soldat nr. 7. Transporteras 1842 till Sandbro soldattorp, Repperda i Alseda. Soldat nr. 29 
Han är fem fot tio tum vilket omräknat blir 1,75m lång.
Husar är en lätt kavallerist, beväpnad i första hand med sabel, dessutom pistol och karbin (ett kortare luntlåsgevär eller flintlåsgevär), ibland med lätt lans.

Johan gifter sig med Anna Svensdotter och de får fyra barn tillsammans. Johan Peter, Adolph, Maria Lovisa och Anna Mathilda. 1846 dör Anna och Johan gifter om sig med Annas tio år yngre syster Lisa Svensdotter. Med Lisa får Johan åtta barn.
Johan Cederblad dör 11 september 1866 i Sandbro, Alseda.

Johan Peter föddes 19 januari 1838 på husartorpet Sandbro, tillhörigt Repperda Sunegård, Alseda socken.
Efter att ha läst för prästen började han som lärling hos en skräddare i Skirö. Någon skräddare blev han dock aldrig, men sydde till sina söner senare i livet. Bl.a sydde han sin son Anders “läsekostym” dvs hans konfirmationskostym.


\person{Johanna Christina Lång}

Carl Johan Pettersson Glyck föddes 2 mars 1789 i Övlandehult Backegård, Ökna.
Han blir antagen 17 december 1810 till Smålands lätta dragoner, kompani Vetlanda Skvadron, rote: Ämmaryd Björsagård, Alseda. Dragon nr. 40.
1833 blir han antagen till Smålands Husarregemente, Vetlanda Skvadron, rote: Ämmaryd Björsagård, Alseda. Han var 5 fot 7 tum (1,68m).
Dragon, soldat som förflyttade sig med häst men som stred till fots. Beväpnad med musköter eller karbin samt blanka vapen.
Död 3 mars 1869 i Ämmaryd Björsagård, Alseda.

Lång är ett soldatnamn och den förste i släkten är Carl Johan Carlsson född 5 maj 1818 i Ämmaryd, Alseda. Han blir antagen 14 oktober 1841 som grenadjär vid Smålands Grenadjärbataljon, Östra Härads Kompani, Skirö socken, Rote Snuggarp med efternamnet Lång. Grenadjär är ursprungligen (1700-talet), en soldat som vid strid hade till uppgift att kasta handgranater {\setff{+case}\currvar(}franska {\it grenad}\/{\setff{+case}\currvar)}. En grenadjär skulle vara minst tre alnar lång (1,80m) och av ståtligt utseende. Senare används titeln om en infanterist vid ett elitförband.
Han var 6 fot lång vilket omräknat blir 1,80m.
Dör 6 juni 1877 i Sköndal, Skönberga, Alseda.

Johanna Christina född på ett soldattorp tillhörigt gården Snuggarp (nuvarande Skönberga) i Skirö socken 1 juli 1843. Dotter till Carl Johan Lång och Anna Johansdotter född 2 november 1817. 


\couple{Johan Peter Cederblad}{Johanna Christina Lång}

\noindent\titlcap Johan Peter och Johanna Christina gifter sig 4 mars 1865 och de bosätter sig på Sjöarps Skattegård i Skirö socken, Jönköpings län. Deras första barn, Thilda Christina, föds 23 september 1865, men redan samma år den 6 november flyttar familjen till Kleva Hytta, Alseda socken. Johan Peter har fått arbete på bruket Kleva Nickelverk, som drevs av Lessebro Bruk, med Bergsrådet Johan Lorentz Aschan som ägare. 

Deras barn:

Matilda Kristina Cederblad, f. 1865-1939. Flyttar till Vidbo, Stockholm 1881. Gift med August Leonard Ludvig Lundberg. Han äger ett plåtslageri. Bor i Upplands Väsby med man och fosterdotter Ingbritt Maria Eleonora Cederblad, vars biologiska mor var Matildas syster Hilda.

Maria Lovisa, f. 1867-1914, flyttar till St. Jakobs församling i Stockholm 1885. Arbetar som tjänarinna på Elfsjö gård, Stockholm.

Anna, f. 1869-1938, arbetade på Upplanda Herrgård i Vetlanda hos familjen Adlercreutz, hon hade anställning där när den stora skandalen i familjen hände. Dottern i familjen var gift med löjtnant Sixten Sparre och de hade två barn, men han förälskade sig i  lindanserskan Elvira Madigan och rymde med henne till Danmark. När deras pengar tog slut, så såg de ingen annan utgång än döden. Sparre sköt dem båda med sin revolver. Denna händelse inträffar i slutet av 1880-talet så det blev en väldig uppståndelse. Dramat har även blivit film. Skrivbordet och det fina blå fatet som jag har nu  kommer från Annas ägodelar. Klockan som ligger i sin originalask kommer troligtvis från Anna också. Det lilla skrivbordet tillhörde Anna,gick i arv till Viva och senare till Lisbeth.


Karolina Josefina, f. 1871-1965, flyttar till Stockholm och får arbete hos Grosshandlare Brisman och hans familj. Senare bor hon hos en dotter i samma familj. 1935 får hon Kungliga Patriotiska medaljen för “lång och trogen tjänst”. Det är familjen som hon har tjänat som trotjänarinna i som har nominerat henne till medaljen. 
Efter Karolinas död, så åker Vivas kusiner till Stockholm i en hyrd lastbil och hämtar hennes ägodelar. Senare hålls en auktion i Vetlanda.

Carl-Oskar, f. 1874-1956, maskinist. Gift med Anna Lovisa Carlsdotter, 7 barn.

Johan Henrik, f. 1876-1901.

Johanna (Hanna) Augusta, f. 1878. Gift med Carl Fredrik Köhler, 5 barn. De tre första dör unga i Sverige. Familjen emigrerar till Amerika.

Gustav Adolf, f. 1881-1963, maskinist. Gift med Sigrid Virena Emilia Reik, 6 barn.

Claes Ferdinand, f. 1883-1884.

Hilda Alma Elisabet, f. 1886-1977. Ogift, 2 barn. Ett av barnen blir fosterdotter hos Hildas syster Matilda Kristina. Hilda flyttar runt ofta, först till Eksjö Stadsförsamling 1906, sen till Eksjö Ränneslätt 1907. Nästa anhalt är Barnarps prästgård 1908,  22/10 1909 går färden till Säby, kvarteret Berget 102. Där blir hon gravid, åker hem till Ädelfors och föder en son, Carl Ragnar 1910, åker tillbaka till Säby med pojken. 12/5 1911 återfinns Hilda med son i en lista med rubriken
”Personer utan fast bostad”. Hon  flyttar till Hammarby 5/3 1913 och sonen kommer efter 16/1 1914. Ännu en gång blir hon gravid  och får en dotter, Ingbritt Maria Cecilia 10/2 1915. Hilda är inte gift, men en man som heter Karl August Lundgren ”låter anteckna sig som barnets fader”. Han är rättare i Björkboda.
Hilda arbetar som tvätterska i ett tvätteri som hon har ihop med syster Matilda. Detta har de i sitt hem och tvättar i bäcken som rinner förbi i närheten av huset. De hämtar tvätt hos ”finare” familjer och lämnar tillbaka den tvättad, struken och manglad.

Lisbeths morfar, Anders Fritiof Cederblad, född 21/10 1888 i Kibbe, nuvarande Ädelfors. Gick i Kibbe skola och konfirmerades i Alseda kyrka den 24/4 1903. Gifte sig 23 maj 1913 med smeddottern Hilma Teresia Caesar från Bärbäcksbron i Alseda.
Gjorde värnplikten vid Kalmar Regemente på Hultsfreds slätt, 30/4 – 27/12 =240 dagar.

Två eller tre somrar reste Johan Peter till Stockholm och arbetade som byggnadsarbetare.
Det var kanske bättre betalt än arbetet vid bruket. Han fick då gå, eller kunde kanske få åka med någon forbonde som körde till Oskarshamn för att hämta varor. Därifrån åkte han med båt till Stockholm. Någon järnväg fanns ju inte på nära håll. Han hade också varit rallare vid Södra Stambanan.

Han fick 1896 Patriotiska Sällskapets medalj för långvarig trogen tjänst, i samband med att Lessebo Bruk AB lade ner sin verksamhet i Kibbe, och sålde sina gruvor och egendomar i Östra Härad, till en tysk konsul Bieber, som tidigare under tre år arrenderat och drivit Guldgruvan. Han bildade nu Ädelfors Guldverksaktiebolag för fortsatt drift av Ädelfors Guldgruva.

Johan Peter fick ett årligt bidrag från ``Stenkvarnefonden'', en fond för gamla bruksarbetare, förmodligen startad av Lessebo Bruk AB, men förvaltades av Bergmästareämbetet, dit ansökan fick insändas.


Johanna Christina var känd för att vara något av "klok gumma", där både skrock och egenhändigt tillverkad örtmedicin ingick.

Hilma Cederblad, gift med sonen Anders, har berättat att när hon beklagade sig för att den några månader gamle sonen inte ville sova på nätterna utan skrek och var besvärlig, så sa Johanna Christina att det kan du väl få något för. Så kom hon med en liten påse med okänt innehåll, som skulle hängas i ett band om halsen på pojken. Det skulle absolut hjälpa. När Hilma tvekade att använda "medicinen'' så blev svaret : “ Är du så dum så du inte tror på det, så må du ha att [besväret]."

Blev mycket omtalad i hela bygden då hon lyckades bota en av stationsinspektor Axel Kjellins döttrar som drabbats av "skerva", engelska sjukan, med sin dekokt. De hade tidigare sökt läkarhjälp utan resultat.

Johanna Christina dör 19191121 och Johan Peter 19260520.



\bye
