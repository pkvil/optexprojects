% gostaberling.tex

\noindent Äntligen stod prästen i predikstolen. Församlingens huvuden lyftes. Så, där var han ändå! Det skulle inte bli mässfall denna söndagen såsom den förra och många söndagar förut.

Prästen var ung, hög, smärt och strålande vacker. Om man hade välvt en hjälm över hans huvud och hängt svärd och brynja på honom, skulle man ha kunnat hugga honom i marmor och uppkalla bilden efter den skönaste av atenare.

Prästen hade en skalds djupa ögon och en fältherres fasta, runda haka, allt hos honom var skönt, fint, uttrycksfullt, genomglödgat av snille och andligt liv.

Folket i kyrkan kände sig underligt kuvat vid att se honom sådan. Det var mera vant vid att han kom raglande ut från krogen i sällskap med glada kamrater, sådana som Beerencreutz, översten med de tjocka, vita mustascherna, och den starke kapten Kristian Bergh.

Han hade supit så förfärligt, att han inte på flera veckor hade kunnat sköta sin tjänst, och församlingen hade måst klaga på honom, först hos hans prost och sedan hos biskop och domkapitel. Nu var biskopen kommen till socknen för att hålla räfst och visitation. Han satt i koret med guldkorset på bröstet, skolpräster från Karlstad och präster från grannförsamlingarna sutto runtomkring honom.

Det var inte tvivel om att prästens uppförande hade gått över det tillåtnas gräns. Då, på adertonhundratjugotalet, var man överseende i fråga om att dricka, men denna mannen hade försummat sitt ämbete för dryckenskapens skull, och nu skulle han förlora det.

Han stod på predikstolen och väntade, medan sista versen av predikstolspsalmen sjöngs.

Det kom en visshet över honom, medan han stod där, att han hade idel fiender i kyrkan, fiender i alla bänkar. Bland herrskaperna på läktaren, bland bönderna nere i kyrkan, bland nattvardsbarnen i koret hade han fiender, idel fiender. Det var en fiende, som trampade orgeln, en fiende, som spelade den. I kyrkovärdarnas bänk hade han fiender. Alla hatade honom, alltifrån de små barnen, som hade burits in i kyrkan, ända till kyrkvaktaren, en stel och styv soldat, som hade varit med vid Leipzig.
Prästen skulle ha velat störta ned på sina knän och bedja dem om förbarmande.

Men ögonblicket därpå kom en dov vrede över honom. Han mindes nog hurudan han hade varit, då han för ett år sedan besteg denna predikstol för första gången. Han var en tadelfri man den gången, och nu stod han där och såg ned på mannen med guldkorset om halsen, som var ditkommen för att döma honom.

Medan han läste upp inledningen, sköljde blodvåg efter blodvåg upp i hans ansikte: det var vreden.

Det var nog sant, att han hade supit, men vem hade rätt att anklaga honom därför? Hade någon sett prästgården, där han skulle leva? Granskogen stod mörk och dyster tätt inpå fönstren. Fukten dröp ned genom de svarta taken, utför de mögliga väggarna. Behövdes det inte brännvin för att kunna hålla modet uppe, då regnet eller yrsnön jagade in genom bräckta rutor, då den vanskötta jorden inte ville ge bröd nog för att hålla hungern fjärran?

Han tänkte, att han hade varit just en sådan präst, som de förtjänade. De söpo ju allesammans. Varför skulle han ensam lägga band på sig? Mannen, som hade begravt sin hustru, söp sig full på gravölet, fadern, som hade döpt sitt barn, höll supgille efteråt. Kyrkfolket drack på hemvägen från kyrkan, så att de flesta voro fulla vid hemkomsten. Det var gott nog åt dem att ha en försupen präst.

Det var på ämbetsresor, då han i sin tunna kappa hade åkt milslångt över frusna sjöar, där alla kalla vindar hade stämt möte, det var, då han hade kastats omkring på dessa samma sjöar i båt under storm och ösregn, det var, då han under yrväder hade måst stiga ur släden och bana hästen väg genom drivor höga som hus, eller då han hade vadat fram genom skogsträsken, det var då, som han hade lärt sig att älska brännvinet.

Årets dagar hade släpat sig fram i tung dysterhet. Bonde och herreman hade gått med alla tankar bundna vid jordens stoft, men om kvällen hade andarna kastat sina bojor, befriade av brännvinet. Ingivelsen kom, hjärtat värmdes, livet blev strålande, sången klingade, rosor doftade. Gästgivargårdens krogrum hade då för honom blivit en sydländsk blomstergård: druvor och oliver hängde ned över hans huvud, marmorstoder blänkte i mörkt lövverk, vise och diktare vandrade under palmer och plataner. 