% Kristina Sofia Johansdotter
% Lisbets farfars mor

\person{Kristina Sofia Johansdotter}{Lisbeths farfars mor}
\label[KSJ]\wlabel{}

Kristina hade flera anfäder som var soldater. Hennes farfars far Peter Sten, född 1789, var husar i Bringebecks Husartorp nr 73 av Smålands regemente, Rosenlund, Kronobergs län.
Farmors far Magni Midas, född 1788, var soldat nr 32 i Lindås rote 1814-1840. Förtjänst utmärkt vid fälttåg 1808-1809 (Finska kriget) och 1813-1814 (Allierad Nordarmé i Tyskland, mot Napoleon). 
Morfars far Sven Stolt, född 1777, soldat  för Kalmar regemente, Södra Vedbo kompani, rote nr 71 Ekekull, torpet Gölåsen. Antagen den 4 december 1801, noterad vara Småläning, 23 år och ogift. 5 fot 7 1/2 tum, dvs 167 cm lång. I rullorna noteras att Sven är kommenderad till Pommern 1806. Från 1817 tjänstgör han vid Göta Kanal som frivillig under vissa perioder. Han dör under arbetet där den 18 maj 1828 och är troligen begravd på annan plats än hemsocknen.
Mormors far Jonas Hake, född 1779, antagen vid Kalmar regemente, Östra Härads kompani, rote nr 119 Trälarp den 4 april 1801. Kommenderad i fält 1805, Pommerska kriget. Han dör i det kriget 1805.

Kristina Sofia född 23 september 1864 i Mosjödal, Skruv Södregård, Nottebäck socken i Kronobergs län. Hennes far hette Jonas Peter Petersson, född 29 augusti 1840 på gården Rosenholm under Klackhult i Hornaryds socken, Kronobergs län och hennes mor Christina Lovisa Magnidotter, född 4 december 1836 i Backstugan Skurubo Södregård i Nye socken, Kronobergs län.
1886 flyttar familjen till Lunnagård, torpet Sjödala och där utökas familjen med barnen Matilda Charlotta 1870, Karl Peter 1874 och Augusta Emilia 1877.

Kristina Sofia flyttar 27 oktober 1884 till Björnhult, Södra Solberga, där hon arbetar som piga. Därifrån till Landsbro Kvarn och Såg, för nytt arbete som piga hos sin blivande make.

Hennes föräldrar och syskon flyttar till Kulla, Södra Solberga den 25 april 1888. 
Fadern dör 18 mars 1917 och modern 15 maj 1925.
