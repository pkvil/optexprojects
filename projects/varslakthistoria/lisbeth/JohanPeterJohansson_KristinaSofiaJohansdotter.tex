% Johan Peter Johansson och Kristina Sofia Johansdotter
%

\couple{Johan Peter Johansson}{Kristina Sofia Johansdotter}
\label[JPJ_KSJ]\wlabel{}

Johan Peter var gift två gånger, första giftermålet 18 april 1875 med Eva Sofia Nilsdotter född 15 november 1846 i Södra Solberga. De får fyra barn: Emma Kristina Seraphina 1876, Johan August 1878, mjölnare, Karl Stefanus 1881. Det finns en anteckning i husförhörslängden om honom: ”Fick nöddop av sin farmor Marta Regina Johansdotter i Landsbro, varefter barnet följande dag dog.” Det fjärde barnet Ester Maria (Maja) född 1883. Endast 22 år gammal och nyligen utexaminerad från småskoleseminariet i Växjö tog hon sig an lärartjänsten i Påvelsmåla. Ensam ledde hon undervisningen i skolans alla klasser, ett fyrtiotal barn i olika åldrar. Skolan låg mitt ute i skogen, utan elektriskt ljus eller telefon och ett par kilometer till nästa granne. Där bodde hon ensam i skolans lilla kammare. Ester Maria utförde hela sin lärargärning inom Algutsboda skoldistrikt fram till sin pension vid 60 års ålder 1943. En av hennes elever var författaren Vilhelm Moberg och det var av henne han fick sänkt betyg i uppförande och ordning. Men han har också sagt att det var hon som fick honom att hitta böckernas värld.

Eva Sofia Nilsdotter insjuknar i lunginflammation och dör 1885.

Kristina Sofia Johansdotter börjar arbeta som piga hos Johan Peter 1 november 1887 och redan 27 december 1887 gifter de sig. De får fem gemensamma barn:
Knut Gunnar född 1889, hemmansägare i Sillre i Borgsjö socken. Ingrid Elisabet Kristina född 1891, hon gifter sig med Emil Elvin som äger en pälsvaruaffär i Vetlanda. De säljer pälsar, hattar och mössor. Efter sin mans död 1927 så förestår hon affären själv. Carl född 1895, skräddarmästare och egenföretagare.

Carl Jonathan Johansson, född 17 januari 1895 i Stocksberg, flyttar till Bäck, Korsberga. Familjen består av föräldrarna Johan Peter och Marta Regina Johansson och fem äldre syskon. Carl blir faderlös vid sjuårsåldern. Pengar var det ont om och att köpa nya kläder till Carl hände sällan, han fick ärva det mesta av sina syskon , t.o.m flickkängor med klack av sin storasyster Ingrid. När han är tolv år arbetar han vid den kvarn- och sågverksrörelse vilken tidigare har ägts av hans far. Två år senare kom han i skräddarlära i sin hemsocken. Enligt sonen Carl-Eric lärde sig Carl att sy hos skräddare Ekholm som troligtvis var alkoholist, för när han inte kunde få tag på sprit så åt han skokräm, eftersom en av ingredienserna var just sprit. 1915 genomgick Carl svenska tillskärar- och yrkeskolan och får sedan en anställning i Axel Karlssons skrädderiaffär i Åseda.

Erik född 1898, mjölnare, chaufför och konstnär. Märta född 1902, bor hemma hos modern. Hon träffar en man som hon förlovar sig med när hon blir gravid, men förlovningen bryts av Märta och hon stannar hos sin mor och föder en dotter 1924 som får namnet Maj-Lis. När Kristina Sofia dör 1930 står Märta ensam med sin dotter. Dessvärre är Märta inte psykiskt stabil och kan inte ta hand om vare sig själv eller barnet. 1931 finns det ingen annan lösning utan Maj-Lis flyttar till sin moster Maja, (Ester Maria) i Algutsboda. Hennes uppväxt blir mycket bra, äntligen får hon en kärleksfull familj som bryr sig om henne på alla sätt.

Märta flyttas till ålderdomshemmet i Korsberga där hon lever hela sitt liv.
Maj-Lis har ingen kontakt med sin mor, men hon åker till hennes begravning 1975. Det finns nog en liten önskan hos Maj-Lis att Märta hade undrat hur dotterns liv hade varit, så hon frågar prästen om modern någon gång hade frågat efter henne. Inte en enda gång, svarar prästen.

Johan Peter och Kristina Sofias äktenskap tar ett hastigt slut då Johan Peter dör 16 december 1901.
Hon står ensam med fyra barn och även tre av sin mans barn i hans första gifte, men de flyttar hemifrån mellan 1903-1904.  Dessutom är Kristina Sofia gravid. De bor i en liten stuga och ingen info finns hur de klarar sig utan mannens inkomst.

