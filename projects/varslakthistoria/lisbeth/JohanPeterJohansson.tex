% Johan Peter Johansson
% Lisbeths farfars far

\person{Johan Peter Johansson}{Lisbeths farfars far}
\label[JPJ]\wlabel{}

Anfäder till Johan Peter var alla mjölnare och kvarnägare till Landsbro kvarn i Nottebäck socken, Kronobergs län. Den förste var hans farfar Johan Fredrik Svensson, född 4 april 1804 i Strömmahult, Näsby socken. 1824 kommer han från Bäckseda till Landsbro kvarn och såg där han får arbete som mjölnare. Johan Fredrik gifter sig 25 juni 1824 med kvarnägarens dotter Sara Christina Johansdotter Nordqvist. De får en dotter 2 januari 1825 men de hann inte vara gifta så länge, bara ca 18 månader. Johan Fredrik dör i fläckfeber (tyfus) den 10 februari 1826. Sara Christina är då gravid med parets son som föds 7 juli 1826 och han får namnet Johan Fredrik Johansson efter sin far.
I bouppteckningen efter fadern Johan Fredrik Svensson visar Landsbro kvarns inventariesumma att den är värd 823 Riksdaler 23 Shilling. Skulden är 47 Riksdaler 25 Shilling. Kvarnen med tvenne par stenar samt 8 Kappeland skattelagd jord är värt 500 Riksdaler. Även en fjärdepart på samma ställe belägna grovbladig såg, värd 50 Riksdaler.

Sara Christina gifter om sig med Peter Pehrsson som också är mjölnare. Johan Fredrik och styvfadern äger längre fram i tiden kvarnen och sågen tillsammans.

1850 gifter sig Johan Fredrik med Marta Regina Johansdotter från Sjöarp, Stenberga.
Marta Reginas far hette Johannes Wahlström. Han är rusthållare på Boda Djupsgård i Stenberga. Hennes farfar hette Petter Wahlström, han var kyrkovärd, nämndeman och häradsdomare i Stenberga. Tillhörde Bondeståndet. Farfars far Pehr Wahlström var ryttare vid Jansbotorp i Stenberga.

Johan Fredrik och Marta Regina bosätter sig på Landsbro kvarn och såg. De får åtta barn mellan 1852-1863, äldste sonen Johan Peter född 15 februari 1852. 

Johan Fredrik dör 31 december 1897 i Landsbro och Marta Regina dör 17 september 1911 i Virserum där hon bor hos ett av sina barn.

