% Johan Peter Cederblad
% Lisbeths morfars far

\person{JohanPeterCederblad}{Lisbeths morfars far}
\label[JPC]\wlabel{}

Cederblad är ett soldatnamn och den förste i släkten är Johan Samuelsson född den 14 december 1805 i Lillökna, Ökna.  Hans föräldrar hette Samuel Samuelsson, född 1 juni 1775 i Karlstorp. Han var hälftenbrukare och kyrkoväktare och Lisa Johansdotter, född 19 februari 1778 i Ökna. De får 5 barn, däribland Johan. 1818 dör Lisa och Samuel gifter ganska snart om sig med Maja Stina Persdotter. I detta andra giftermål föds ytterligare sex barn.

Den 6 juni 1825 blir Johan antagen som husar vid Smålands Husarregemente, Vetlanda Skvadron. Hans soldatnamn blir Cederblad. 1835 bor han på Emhults Skattegård Soldat nr 7. Transporteras 1842 till Sandbro soldattorp, Repperda i Alseda. Soldat nr 29 
Han är 5 fot 10 tum vilket omräknat blir 1,75 m lång.
Husar är en lätt kavallerist, beväpnad i första hand med sabel, dessutom pistol och karbin (ett kortare luntlåsgevär eller flintlåsgevär), ibland med lätt lans.

Johan gifter sig med Anna Svensdotter och de får fyra barn tillsammans. Johan Peter, Adolph, Maria Lovisa och Anna Mathilda. 1846 dör Anna och Johan gifter om sig med Annas tio år yngre syster Lisa Svensdotter. Med Lisa får Johan åtta barn.
Johan Cederblad dör 1866 09 11, Sandbro, Alseda.

Johan Peter föddes 19 januari 1838 på husartorpet Sandbro, tillhörigt Repperda Sunegård, Alseda socken.
Efter att ha läst för prästen började han som lärling hos en skräddare i Skirö. Någon skräddare blev han dock aldrig, men sydde till sina söner senare i livet. Bl.a sydde han sin son Anders “läsekostym” dvs. hans konfirmationskostym.

