% Carl Johansson och Linnéa Isaksson
%

\couple{Carl Johansson}{Linnéa Isaksson}
\label[CJ_LI]\wlabel{}

Carl och Linnéa träffas troligen i prästgården. Linnéa arbetar ju där och Carl har flyttat till Åseda 16 september 1917 och hyr ett rum hos kyrkoherden. Tycke uppstår och de gifter sig 12 december 1917. Deras första hem ligger på Karlavägen, vid bokhandeln bakom missionshuset.
1923 åker han till Stockholm till J-E Kollbergs tillskärarskola. Hans titel blir skräddarmästare. Samma år startar han sin egen skrädderirörelse i Åseda. 1942 köper han en fastighet på Kexholmen, Åseda där han också har en kappaffär. Carl är en av initiativtagarna till Åseda Hantverks-och industriförening, tillhör styrelsen. Han är också med i Kronobergs län och Östra Smålands skräddar mästareförening. Ett annat uppdrag är ledamot i kyrkofullmäktige. Det sista uppdraget blir ordförandeskapet i Åseda Pensionärsförening, han avgår där 1970.

Den blå urnan som stod på vitrinskåpet i stora rummet hos Carl-Eric och Viva kommer från Åseda. Den har en liten historia: När Carl hade skrädderi-och kappaffär så kom det in en dam från Stockholm och beställde två kappor, som skulle skickas till henne. Kapporna syddes och levererades till damen. Nu visar det sig att hon inte kunde betala dem med pengar utan hon skickar den blå urnan som betalning, hon ägde en antikvitetsaffär. Vi har i alla år trott att den skulle vara värdefull, men nu har vi värderat den och den är bara värd ca 400 kr. Det var ingen bra affär för farfar.

De får fyra barn, Maj-Britt 1918, Ann-Margret 1920, Carl-Eric 1921 och Gullan 1926. Efter några år flyttar familjen till det som kallas Blombergahuset, se bild.
Född 19 mars 1921 i Åseda.  

Carl-Eric hyser en dröm att bli postiljon (brevbärare) men efter konfirmationen börjar han arbeta på ett plåtslageri i Åseda. Den anställningen blir inte så lång, han har dessvärre väldigt svårt för höjder och att lägga plåttak är inte att tänka på. Första arbetet är på ett kyrktak, så den karriären blev kort. Dessutom trycker Carl på att han ska lära sig till skräddare och ta över rörelsen så småningom. Carl är oerhört duktig i sin yrkesroll, allt ska vara perfekt innan han godkänner något. C-E har berättat att om det så var bara ett stygn som inte var som det skulle i Carls ögon så tog han tag i sömmen och rev isär den.
 På sin fritid ägnade sig C-E åt fotboll, ishockey, bandy och tennis. På den här tiden (1930,-40-50 och -60 talen) fanns ingen särskild anställd som klippte gräset på fotbollsplanen eller på vintern spolade is i ishockeyrinken m.m. Utan det var de aktiva inom vardera sport som fick hjälpa till med det. Det var även problem med utrustningen när pengarna inte räckte till nyköp. C-E hade t.ex inte råd att köpa sig ishockeybenskydd så han sydde ett par istället.

Carl-Erics syskon:

Maj-Britt, 1918, gift med Torsten Johansson. En son, Roland.

Ann-Margret, 1920, gift med Folke Johansson. En son, Lennart.

Gullan, 1926, gift med Lars Stéen. Tre döttrar, Margareta och tvillingarna Kristina och Katarina.

Ann-Margrets son, Lennart, har berättat några minnen om Linnéa för mig. Hennes far var väldigt glad i sprit och gick gärna i god för hans vänner när de behövde pengar. Till slut gick det så illa att familjen blev tvungna att gå ifrån gården och flytta till Klavreström. Han var nog inte den bästa maken och pappa.
Med detta som bakgrund så fanns det ingen sprit i Carl och Linnéas hem, med ett undantag. I en kakelugn i ett sovrum som inte användes hade Linnéa en flaska konjak och när hon kände att en förkylning på gång eller liknande så hämtades flaskan och hon hällde upp konjak i en sked och tog den som medicin. Det var mycket viktigt att inte använda ett glas, för då var det inte medicin, utan för nöjes skull. Det var Lennarts pappa Folke som fick hämta ut konjaken, varken Carl eller Linnéa tyckte att de kunde göra det själva.
Lennart har ett litet bord med en utdragslåda i sin ägo, det var Linnéas pappas matbord och i lådan hade han sina bestick. Vid detta bord åt han sina måltider ensam, resten av famljen fick stå och äta.

