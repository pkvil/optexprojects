% Anders Cederblad och Hilma Caesar
%

\couple{Anders Cederblad}{Hilma Caesar}
\label[AC_HC]\wlabel{}

Under tiden i Repperda blev Hilma medlem i Logen Ädelfors Framtid av NOV, som fanns i Ädelfors i början av 1900-talet. Medlem var också Anders Cederblad från Ädelfors. Kanske var det på logemötena som de lärde känna varandra.
De gifter sig i maj 1913 och den första åren får de bo hos Anders föräldrar som har en lägenhet i ett flerfamiljshus i Ädelfors. Huset kallas Smedstugan som ligger en bit förbi Stenhuset in på vägen till Ökna. Det var arbetarbostäder för personer som arbetade vid Kleva Nickelverk.
I september 1913 föds en son, Georg, 1916 ytterligare en son, Henry. En dotter föds 1918, Berta Viola. Hon får inte leva mer sex månader, spanska sjukan tar hennes liv. 
Georg berättade för mig att han mindes sin lillasyster, han var ju fem år när hon föddes.
Det var en tung sorg, för efter Bertas död talades det aldrig om henne. Hilma nämnde henne aldrig mer och då gjorde ingen annan det heller.

Arbetade som smältare vid masugnen i Ädelfors 1914 – 1918 där han förlorade sitt ena öga vid  olycksfall i arbetet. Han fick ett stänk från smältan i ögat och synen gick ej att rädda. Arbetsförmågan ansågs nedsatt till 25\% och han tillerkändes en livränta på 300 kr/år som utbetalades med 75 kr i kvartalet. Prästbevis att han fortfarande fanns måste skickas in 2 ggr/år.
1919 upphörde verksamheten vid masugnen i Ädelfors.
Gården Germunderyds Ebbagård som varit i Brukets ägo såldes till AB Kalmar Kol och Trävaruaffär, som genast satte igång med avverkning på den skogrika gården. Då blev det skogs-och sågverksarbete och även kolning förekom. Anders fick arbete där och senare blev han kantare vid sågen i Ädelfors.
1932 lade Kalmar Kol och Trävaruaffär ned verksamheten och sålde Germunderyds Ebbagård till Domänverket.
En bit in på 1930-talet blev det dåliga tider, med stor arbetslöshet och därmed svårt att få ett arbete. Anders arbetar med stenröjning på åkrar i Möcklarp, byggnadssnickare, fick arbete som kantare på olika sågverk bl.a vid Carl Blomstrands såg i Holsbybrunn ett par vintrar.
Under krigsåren 1940 – 1945 högg han meterved på Domänverkets gård i Germunderyd. Sommaren 1945 började han arbeta på Bröderna Granlunds Båtbyggeri i Ädelfors där han stannade till semestern 1958, samma år som han fyllde 70 år.
Georg har intervjuat Anders om hans yrkesliv, det finns sparat på en cd som jag har. I bakgrunden (om man lyssnar noga) kan man höra mormor Hilmas röst ibland och ljudet från en väggklocka.

1920 flyttar hela familjen till Brostugan, även kallat “huset vid åbron”. Det finns fortfarande kvar, ett rött tvåvåningshus med vita knutar som ligger vid bron innan man kommer till Stenhuset. 
Anders pappa (hans mamma dog 1919) Johan Peter Cederblad och nu även Hilmas mamma, Johanna Lovisa Johansdotter, bor här också. 
9 november samma år föds en dotter, Viva och livet blir nog lättare för både Anders och Hilma.
Viva har berättat att hon ofta gick till sin mormor till hennes rum, Hilma hittade henne alltid där glatt mumsande på något gott.
Hilma var som vanligt var på den tiden hemmafru, skötte hem, barn och de gamla föräldrarna. Skaffade sig symaskin, och var duktig att sy både till sig själv och barnen. Lånade ibland vävstol och vävde trasmattor till hemmet.

Under några år i början av äktenskapet hade man också hushållsgris som slaktades strax före jul. Att ta hand som slakt hade hon ju lärt sig på gårdarna hon varit på. Så det blev blodpalt och pölsa och flera sorters korv, pressylta och rullsylta och julskinka lagades till. Fläsket saltades ned i trätunnor för kommande behov.

1927 tar Anders ett inteckningslån på 2000 kr och tillsammans med sin syster Anna Gustava som går in med 2000 kr, med löfte att få bo på övervåningen om ett rum och kök så länge hon lever. De köper ett gammalt timmerhus av Emil Edborg i Repperda, som får utgöra stommen i det nya huset. Det var dock längre än det nya skulle bli, så därför tog man bort en bit på mitten och fogade ihop de båda halvorna. Husdelarna transporterades med häst och vagn till Ädelfors.

Pengarna räcker tyvärr inte till att köpa tomten där huset ska stå, utan de hyr den av Ädelfors bruk. Anders tog ledigt ett par sommarmånader och med hjälp av två snickare så byggdes det nya huset som får namnet Cederslund. Den 22 december 1930 lyckas han dock med att köpa tomten, lagfart blir beviljad 29 oktober 1931. Anders renoverar huset, men det förblir med dagens mått mätt rätt så omodernt. Enbart kallt vatten indraget, kakelugnar och vedspis som värmekälla och till matlagning. Utedass och och inget badrum. Kylskåp och frys fanns inte heller utan de flesta kylvaror förvarades i en matkällare. Det som behövdes dagligen fick plats i ett skafferi i köket. Så här ser det ut ända in på 1970-talet.

Skolhuset ligger bredvid Cederslund så Viva behöver inte gå långt för att komma dit. Hon går i skolan i sex år och efter det en skolkökskurs.

[Bild och hennes berättelse.]

På 1930-talet, om man ville åka på semester och det inte fanns så mycket pengar, då åkte man på cykelsemester. Så gjorde även Viva och hennes vänner. Bl.a cyklade de till Tranås en sommar. Hennes bröder Georg och Henry, samt en kamrat, cyklade till Gripsholms slott och fortsatte till Trollhättan. Väl där skickade de ett vykort hem och skrev att de skulle ta en ”avstickare” till Göteborg innan de kom hem igen. De cyklade också till Kalmar, vidare till Karlskrona för att slutligen hamna i Ystad som var målet med resan. De skulle titta på en utställning där.
På sina cyklar hade de all utrustning som de behövde såsom tält, mat, fotogenkök, kläder m.m

1936 hände något som alla i Ädelfors talade om länge. Viva hade lämnat in en tipskupong och vann högsta vinsten. 15 040 kr och 67 öre. Det var en hel förmögenhet på den tiden. Hon hjälpte sina föräldrar att betala av lån på huset och bekostade sin egen utbildning till damfrisörska. Denna utbildning fanns i Vetlanda och troligtvis bodde hon där i veckorna. Hon kallades därefter för Lyckaflickan i Ädelfors.

[Bild på tipskupong]

Viva flyttar till Åseda 23 november 1939, hon har nu fått sitt första arbete på Fru Johanssons damfrisering. Hennes bostad är en liten lägenhet ovanför salongen. Hennes lön är 30 kr/mån.
Ibland var det nog inte så roligt att ta hand om kunder. Det var sällsynt med damer som hade tvättat håret innan de kom för att bli fina. Det var inte ovanligt att hon fick börja med att avlusa och tvätta håret innan hon kunde påbörja klippning och permanentning.

[Bild på huset.]

Fram tills nu har Viva inte haft tillgång till sitt bankkonto, hon är ju inte myndig. När hon behöver pengar måste hon ta kontakt med sin pappa Anders som i sin tur får ta bussen in till Vetlanda och ta ut pengar på banken, som Viva sedan får från honom.
1940 blir hon erbjuden att köpa damfriseringen och nu skickas det många brev mellan Åseda och föräldrarna i Ädelfors hur hon ska göra. Det diskuteras fram och tillbaka, Viva får nog både goda råd och förmaningar att tänka igenom allting noga, innan hon tar ett beslut.
I oktober samma år köper hon salongen av Fru Johansson för 4000 kronor och anställer en flicka som hjälpreda. Hennes nettoinkomst är 70-80 kr/månaden.

Några händelser från brevväxlingen mellan Hilma och Viva:
1940: Missionsauktion i Repperda ( Missionshuset är det gula huset uppe på höjden till höger om vägen in till Ädelfors). Hilma ropar in två hemostar, 6 och 7 kr kilot och 1 kilo kaffe för 5 kr och 50 öre. 
Nu är det tilldelta kuponger för varje familj för nästan allt man behöver köpa.

1941: Januari: vattenledningen i Cederslund har frusit, minus 33 grader och inte mycket snö.
          Februari: Hilma och Viva byter matvaror med varandra t.ex Viva skickar socker till sin 
          mamma och får ägg istället.
          Vädret har blivit mildare och nu är det mycket snö.
          Mars: Georg och Henry har fått uppgiften att snickra ett bakbord till Viva. Enligt Hilma så 
          var de lite slöa med att få det färdigt, så Hilma fick prata med dem “på skarpen”.
          April: Hilma skriver “potatis är snart det enda man får köpa utan rabattkupong”.
          Vivas anställda flicka säger upp sig utan förvarning, mormor är mycket upprörd. Ett par
          Dagar senare så har allt ordnat sig med en ny flicka, Hilma är glad igen. Hon tar tåget till 
          Åseda, stannar över natten.
          Maj: Hilma går Riksmarschen, en tävling inom motionsidrotten på 1940-talet. Sveriges
          Kommuner tävlade mot varandra, damerna gick 10 km och herrarna 15 km.
          Hilmas tid blev 1 timme och 28 minuer.
          Juni: Det var många kalas och kafferep i Ädelfors och alla damer hade finklänning och  
          den bästa hatten. Hilma behöver en ny så hon åker in till Vetlanda och köper en fin hatt 
          även att hon tycker den är väldigt dyr, hela 16 kronor!
          September: Hilma skickar lingon, morötter och äpplen till Viva. I slutet av månaden skriver
           hon att det rinner vatten i nya kanalen, så nästa gång Viva kommer hem kan hon paddla
          kanot.
          December: Viva undrar hur mycket pengar hon har på banken.Hon får veta att hon har 
          5000 kronor.
          1942: Januari. -35 grader, allt i skafferiet har frusit. Inte ens Georg vågar sig ut, inget
          arbete på fabriken för köldens skull.
          Februari: 39-40 minusgrader.
          Maj: Vinter igen, snö på marken.
          Augusti: Hilma cyklar till Åseda och äter middag hos Viva.



Tyvärr finns inte Vivas brev till Hilma kvar. De skrev till varandra ett par gånger i veckan. Det hade ju varit roligt att få hennes berättelser också. Efter 1942 finns inga brev kvar. Kanske ringde de till varandra, det fanns ju en telefonstation i Ädelfors och Viva hade egen telefon då. 1966 får Anders och Hilma egen telefon.

Anders var den snällaste och finaste morfar som jag och mina bröder kunde ha. Han skrev också till Viva med den allra finaste handstilen. 
    
Anders dör 28 juli 1970 på Eksjö Lasarett. Georg och Henry åker till Eksjö och är med honom till slutet.

Hilma råkade våren 1971 ut för en olyckshändelse (lårbensbrott). På morgonen denna dag körde brevbäraren sin vanliga runda tidigt på morgonen. Hilma hade fått post denna dag så han skulle bara lägga den i brevlådan, men tänker i samma stund att “ det var länge sedan jag träffade Hilma, jag går in med posten så kan jag prata med henne en stund och höra hur hon har det”. När han kommer in i köket så ligger Hilma på golvet och har brutit lårbenet. Troligtvis går han in till Georg som bor tvärs över vägen och ambulans tillkallas. Ingen vet varför brevbäraren just denna dag bestämmer sig för att gå in till Hilma, för det tillhör inte vanligheten att göra det, men vilken otrolig tur att han gjorde det. Efter lasarettsvistelsen i Eksjö, flyttas hon till äldreboendet Österliden i Holsbybrunn. På egen begäran fick hon sedan stanna kvar där till sommaren 1981, då hon insjuknade och fick komma till Vetlanda Sjukstuga, där hon avled den 12 oktober 1981.

Hilmas syskon:

Karl Johan Levi Caesar född 16/5 1872 emigrerar till Amerika.
Frida Lovisa Josefina Caesar född 15/5 1875 emigrerar till Amerika.
Gustav Alfred Irenius Caesar född 15/12 1877 emigrerar till Amerika.
Helga Emilia Caesar född 17/11 1880. Gift med snickaremästare Sven Lindberg i Slättåkra. Han tillverkade hyvelbänkar och slöjdbänkar.
Hjalmar Fredrik Caesar född 31/10 1883. Smed. Gift med Gulli Maria Teresia Franzén.
Emma Vicktoria Caesar född 16/1 1887. Gift med August Elof Karlsson 1916, innan dess har hon arbetat som tjänarinna  på Vagnhester Kåragård.

