% Johan Peter Cederblad och Johanna Christina Lång
%

\couple{Johan Peter Cederblad}{Johanna Christina Lång}
\label[JPC_JCL]\wlabel{}

Johan Peter och Johanna Christina gifter sig 4 mars 1865 och de bosätter sig på Sjöarps Skattegård i Skirö socken, Jönköpings län. Deras första barn föds, Thilda Christina 23 september 1865 men redan samma år 6 november flyttar familjen till Kleva Hytta, Alseda socken. Johan Peter har fått arbete på bruket Kleva Nickelverk, som drevs av Lessebro Bruk, med Bergsrådet Johan Lorentz Aschan som ägare. 

Deras barn:

Matilda Kristina Cederblad, född 1865-1939. Flyttar till Vidbo, Stockholm 1881. Gift med August Leonard Ludvig Lundberg. Han äger ett plåtslageri. Bor i Upplands Väsby med man och fosterdotter Ingbritt Maria Eleonora Cederblad, vars biologiska mor var Matildas syster Hilda.

Maria Lovisa, född 1867-1914, flyttar till St. Jakobs församling i Stockholm 1885. Arbetar som tjänarinna på Elfsjö gård, Stockholm.

Anna, född 1869-1938, arbetade på Upplanda Herrgård i Vetlanda hos familjen Adlercreutz, hon hade anställning där när den stora skandalen i familjen hände. Dottern i familjen var gift med löjtnant Sixten Sparre och de hade två barn, men han förälskade sig i lindanserskan Elvira Madigan och rymde med henne till Danmark. När deras pengar tog slut, så såg de ingen annan utgång än döden. Sparre sköt dem båda med sin revolver. Denna händelse inträffar i slutet av 1880-talet så det blev en väldig uppståndelse. Dramat har även blivit film. Skrivbordet och det fina blå fatet som jag har nu  kommer från Annas ägodelar. Klockan som ligger i sin originalask kommer troligtvis från Anna också. Det lilla skrivbordet tillhörde Anna, gick i arv till Viva och senare till Lisbeth.

Karolina Josefina, född 1871-1965, flyttar till Stockholm och får arbete hos Grosshandlare Brisman och hans familj. Senare bor hon hos en dotter i samma familj. 1935 får hon Kungliga Patriotiska medaljen för “lång och trogen tjänst.” Det är familjen som hon har tjänat som trotjänarinna i som har nominerat henne till medaljen. 
Efter Karolinas död, så åker Vivas kusiner till Stockholm i en hyrd lastbil och hämtar hennes ägodelar. Senare hålls en auktion i Vetlanda.

Carl-Oskar, född 1874-1956, maskinist. Gift med Anna Lovisa Carlsdotter, sju barn.

Johan Henrik, född 1876-1901.

Johanna (Hanna) Augusta, född 1878. Gift med Carl Fredrik Köhler, fem barn. De tre första dör unga i Sverige. Familjen emigrerar till Amerika.

Gustav Adolf, född 1881-1963, maskinist. Gift med Sigrid Virena Emilia Reik, sex barn.

Claes Ferdinand, född 1883-1884.

Hilda Alma Elisabet, född 1886-1977. Ogift, två barn. Ett av barnen blir fosterdotter hos Hildas syster Matilda Kristina. Hilda flyttar runt ofta, först till Eksjö Stadsförsamling 1906, sen till Eksjö Ränneslätt 1907. Nästa anhalt är Barnarps prästgård 1908, 22 oktober 1909 går färden till Säby, kvarteret Berget 102. Där blir hon gravid, åker hem till Ädelfors och föder en son, Carl Ragnar 1910, åker tillbaka till Säby med pojken. 12 maj 1911 återfinns Hilda med son i en lista med rubriken
”Personer utan fast bostad.” Hon flyttar till Hammarby 5 mars 1913 och sonen kommer efter 16 januari 1914. Ännu en gång blir hon gravid och får en dotter, Ingbritt Maria Cecilia 10 februari 1915. Hilda är inte gift, men en man som heter Karl August Lundgren ”låter anteckna sig som barnets fader”. Han är rättare i Björkboda.
Hilda arbetar som tvätterska i ett tvätteri som hon har ihop med syster Matilda. Detta har de i sitt hem och tvättar i bäcken som rinner förbi i närheten av huset. De hämtar tvätt hos ”finare” familjer och lämnar tillbaka den tvättad, struken och manglad.

Lisbeths morfar, Anders Fritiof Cederblad, född 21 oktober 1888 i Kibbe, nuvarande Ädelfors. Gick i Kibbe skola och konfirmerades i Alseda kyrka den 24 april 1903. Gifte sig 23 maj 1913 med smeddottern Hilma Teresia Caesar från Bärbäcksbron i Alseda.
Gjorde värnplikten vid Kalmar Regemente på Hultsfreds slätt, 30 april – 27 december = 240 dagar.

Två eller tre somrar reste Johan Peter till Stockholm och arbetade som byggnadsarbetare.
Det var kanske bättre betalt än arbetet vid bruket. Han fick då gå, eller kunde kanske få åka med någon forbonde som körde till Oskarshamn för att hämta varor. Därifrån åkte han med båt till Stockholm. Någon järnväg fanns ju inte på nära håll. Han hade också varit rallare vid Södra Stambanan.

Han fick 1896 Patriotiska Sällskapets medalj för långvarig trogen tjänst, i samband med att Lessebo Bruk AB lade ner sin verksamhet i Kibbe, och sålde sina gruvor och egendomar i Östra Härad, till en tysk konsul Bieber, som tidigare under tre år arrenderat och drivit Guldgruvan. Han bildade nu Ädelfors Guldverksaktiebolag för fortsatt drift av Ädelfors Guldgruva.

Johan Peter fick ett årligt bidrag från "Stenkvarnefonden", en fond för gamla bruksarbetare, förmodligen startad av Lessebo Bruk AB, men förvaltades av Bergmästareämbetet, dit ansökan fick insändas.

Johanna Christina var känd för att vara något av "klok gumma", där både skrock och egenhändigt tillverkad örtmedicin ingick.

Hilma Cederblad, gift med sonen Anders, har berättat att när hon beklagade sig för att den några månader gamle sonen inte ville sova på nätterna utan skrek och var besvärlig, så sa Johanna Christina att det kan du väl få något för. Så kom hon med en liten påse med okänt innehåll, som skulle hängas i ett band om halsen på pojken. Det skulle absolut hjälpa. När Hilma tvekade att använda "medicinen'' så blev svaret : “Är du så dum så du inte tror på det, så må du ha att [besväret]."

Blev mycket omtalad i hela bygden då hon lyckades bota en av stationsinspektor Axel Kjellins döttrar som drabbats av "skerva", engelska sjukan, med sin dekokt. De hade tidigare sökt läkarhjälp utan resultat.

Johanna Christina dör 21 november 1919 och Johan Peter 20 maj 1926.

