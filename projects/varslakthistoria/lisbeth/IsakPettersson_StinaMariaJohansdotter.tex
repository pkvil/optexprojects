% Isak Pettersson och Stina Maria Johansdotter
%

\couple{Isak Pettersson}{Stina Maria Johansdotter}
\label[IP_SMJ]\wlabel{}

De gifter sig 31 oktober 1885 och familjen blir snart större när barnen föds: Johan 1886, Elna 1889, Gustaf 1890, Helga 1892, Linnéa 1894, Albert och Axel 1897 och Ottilia 1899.

Livet är inte alltid så lätt, sonen Axel dör bara ett år gammal 1898, samma år insjuknar de andra barnen i difteri. Isak tillkallar prästen som ordnar med medicin från doktorn. Av någon anledning så räcker inte medicinen till alla barnen utan Helga dör endast ett år gammal. Stina gick ofta till lilla Helgas grav ända tills en dag hon tyckte sig höra en röst som sa att hon inte skulle komma mer utan lämna henne ifred. Isaks dotter Elna har berättat detta för sin systerdotter Gullan Stéen och gett henne en knapp som ska ha suttit i Helgas kofta.
Ann-Margrets son, Lennart, har berättat några minnen om Linnéa för mig. Hennes far var väldigt glad i sprit och gick gärna i god för hans vänner när de behövde pengar. Till slut gick det så illa att famljen blev tvungna att gå ifrån gården och flytta till Klavreström. Han var nog inte den bästa maken och pappa. För Linnéas mamma var det värst, gården var ju hennes barndomshem.
Med detta som bakgrund så fanns det ingen sprit i Carl och Linnéas hem, med ett undantag. I en kakelugn i ett sovrum som inte användes hade Linnéa en flaska konjak och när hon kände att en förkylning på gång eller liknande så hämtades flaskan och hon hällde upp konjak i en sked och tog den som medicin. Det var mycket viktigt att inte använda ett glas, för då var det inte medicin utan för nöjes skull. Det var Lennarts pappa Folke som fick hämta ut konjaken, varken Carl eller Linnéa tyckte att de kunde göra det själva.
Lennart har ett litet bord med en utdragslåda i sin ägo, det var Linnéas pappas matbord och i lådan hade han sina bestick. Vid detta bord åt han sina måltider ensam, resten av famljen fick stå och äta.

Isak är inte den lättaste man att leva med, han är inte så snäll mot Stina, behandlar henne mer som en piga än hustru. Han bryr sig inte om att sköta gården utan samlar hellre sina grannar i gården för att umgås och dricka alkohol. Gården börjar snart kallas för Gubbagården dvs. där gubbarna träffas och festar. 

1909 är gården körd i botten och familjen tvingas flytta därifrån. De hamnar i Klavreström stationssamhälle där Isak i församlingsboken står som slaktare till yrket.

Carls blivande fru, Linnéa är född 1894 i Gubbagården, i byn Galtabäck i Nottebäck socken. Föräldrarna heter Isak Pettersson och Stina Maria Johansson, de har sex barn. Gubbagården är Isaks föräldrahem som han har fått överta. Isak är tyvärr inte så mån om gården och dess skötsel, till slut måste familjen gå ifrån den. Namnet Gubbagården kommer från talesättet att det var den gården som gubbarna samlades i. Carl-Erics minne av sin morfar är att han mest låg och sov på soffan. Hela familjen flyttar sedermera till Stenhuset i Klavreström. Linnéa kommer som ung flicka till Åseda 1914.Hon har fått anställning som kammarjungfru hos kyrkoherde Bergdahl. En dotter, Gullan Stéen, berättar följande historia: Linnéa är i Folkets Park och står vid dansbanan, lite längre bort står tre flickor och pratar. Hon hör att de pratar om henne, en av flickorna, Astrid Spjut, säger: Det är prästpigan. Linnéa tycker att det låter nedlåtande, men hon får en liten hämnd. En tid senare är hon i hattaffären, hon ser en hatt som hon genast tycker om men får veta av expediten att Astrid Spjut har tänkt köpa den. Hatten kostar 6 kronor (mycket pengar på den tiden) men Linnéa tar sig råd med tanken att ”Astrid Spjut ska inte få den”.

Linnéa flyttar till Växjö 1914 och arbetar som piga i en familj, åter till Klavreström 1916 och till Åseda 1917 där hon är anställd som piga i prästgården.
Äldste brodern Johan arbetade som verkstadsbokhållare vid Klavreströms Bruk. Han spelade flöjt i en orkester, han var intelligent och duktig i matematik. Han hjälpte Linnéas blivande man Carl med hans bokföring. 

Nästa bror, Gustaf arbetar som charkuterist. Albert är metodistpastor i Kristinehamn, Arboga och Malmö, han har t.o.m ett eget radioprogram där han håller sina predikningar. Han är gift med Ebba och har två barn, Gertrud och Sven. Det finns en historia om Sven och Carl-Eric, pojkarna hade lagt ut kräftburar och när det var dags att vittja dem, så hade de glömt att ta med något att förvara kräftorna i. De fick göra det bästa av situationen, så efterhand som de plockar upp kräftorna ur burarna så la de dem i sina kepsar. Det måste ha känts väldigt märkligt att ha levande kräftor krälande i håret  under kepsarna. Lillasyster Ottilia, (som enligt Carl-Eric var en mycket högdragen person) gifter sig med en äldre man och bosätter sig i Värmland. Innan hon gifte sig arbetade hon på Stockholm slott.

Deras syster Elnas liv börjar bra, hon träffar Sigurd Pettersson som arbetar som konduktör. De gifter sig 1913 och bor i Växjö. Sigurd är en trevlig och gladlynt man utåt sett och det verkar vara ett lyckligt äktenskap för alla. Men en dag 1932 ska paret gå på promenad när Sigurd plötsligt säger att han har glömt något hemma och går tillbaka till hemmet. Elna väntar och väntar men han kommer inte. Hon går hem igen och när hon kommer innanför dörren så får hon se att han har hängt sig i trappan. Elna och hennes närmaste har ingen aning om varför han tog detta beslut.
Eftersom de har levt på Sigurds inkomst så blev livet efter hans död hårt för Elna i dubbel bemärkelse. Hennes man är borta och hon har inga pengar till sitt uppehälle, men hon finner på råd. Hon tar emot inackorderingar, stryker tvätt åt andra mot betalning och undervisar som hushållslärare i sitt hem. Elna gifter aldrig om sig.

Isak och Stina bor kvar i Klavreström tills de dör, Isak 1942 och Stina 1944.

