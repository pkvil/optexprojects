% Johan August Caesar
% Lisbeths mormors far

\person{Johan August Caesar}{Lisbeths mormors far}
\label[JAC]\wlabel{}

Soldat Zachris Månsson Ekholm levde mellan 1750-1810. Bodde med fru och barn på soldattorpet nr 69 Bistockskulla, Ekholma i Åseda socken. Ett av deras barn heter Magnus Zachrisson, han föddes 1786 och ca 1807 blir antagen till soldat vid Kalmar Regemente, tredje majorens kompani Aspeland, Järeda. Nu får han soldatnamnet Caesar.
Han var 6 fot 3 tum, dvs 1,88 m lång.
Några år senare bodde han med fru och åtta barn på soldattorpet nr 80 Tynshult, Virserum.
Anteckning i husförhörsboken: “Död vid arbete vid Göta kanal, kommenderad att arbeta vid kanalen men dog dagen efter sin ankomst 7 maj 1830. Begravd på militäriskt vis.“
Dödsorsaken var lungsot, dvs. tuberkulos.

En av hans söner hette Carl Johan Caesar, levde mellan 1812-1882. Han arbetade som pappersmakargesäll vid Qvills pappersbruk i Ökna. Gift med Anna Maria Magnidotter som levde mellan 1816-1883.
De fick barnen: Gustav Adolf 1841, Gustaf Martin 1844, tvillingarna Per Alfred och Johan August 1846, Carl Peter Alfrid 1849, Frans Algot 1852 och Gustaf Adolph 1855.
Fyra av barnen dör i tidig ålder och två emigrerar till Nordamerika.

Johan August född 31 mars 1846, Qills Pappersbruk Ökna. Hans föräldrar och syskon flyttar till Källedal,Trollhälla i Alseda socken 1847 och efter några år där till Sandslätt på Trollhällas ägor i Alseda.

1 april 1866 åker Johan August till Stockholm och arbetar där till november. Återvänder hem, men redan 7 november flyttar han till Lindshammar, Nottebäcks socken och arbetar som dräng hos en man som är anställd som smed på järnbruket. Den 7 december 1868 flyttar han hem till Sandslätt igen.

