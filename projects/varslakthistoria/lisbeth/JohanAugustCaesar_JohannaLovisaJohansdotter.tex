% Johan August Caesar och Johanna Lovisa Johansdotter

\couple{Johan August Caesar}{Johanna Lovisa Johansdotter}
\label[JAC_JLJ]\wlabel{}

Johan August gifter sig 1 december 1871 med Johanna Lovisa Johansdotter, från Godanstorp Alseda. 

1876 köpte han stugan på Broatorpet (Bärbäcksbron) som låg på Trollhälla gårds marker. Han hade tidigare fått löfte att bygga en stuga i närheten, men då torparen avflyttat och den stugan blev ledig, avstod han från att bygga. Priset för Bärbäcksbron blev 150 kr. Trollhälla gård ägdes av Alseda socken och där fanns även socknens fattiggård. Troligen byggde han i samband med detta smedjan nere vid brofästet till bron över Emån.

Köpekontrakt

Förutom det vanliga “bondsmidet” med reparationer och skoning av hästar och dyligt, tillverkade han diverse verktyg som potatishackor, lövhackar och “riskar” som han sedan sålde på torget i Vetlanda. Åkte då med någon bonde som hade egna ärenden till torget. Hans dotter Hilma har berättat att när han kom hem från torgresan så tömde han ut smålandspungen på spiselhällen, och räknade noggrant ihop dagskassan. Det var väl inte alltid så lätt att få slantarna att räcka till och han beklagade sig ibland. “En är så gällskyldig så en vet inte ut vart in en ska vända sig.” Gällskyldig betyder skuldsatt.

Deras barn:

Karl Johan Levi Caesar född 16/5 1872 emigrerar till Amerika.
Frida Lovisa Josefina Caesar född 15/5 1875 emigrerar till Amerika.
Gustav Alfred Irenius Caesar född 15/12 1877 emigrerar till Amerika.
Helga Emilia Caesar född 17/11 1880. Gift med snickaremästare Sven Lindberg i Slättåkra. Han tillverkade hyvelbänkar och slöjdbänkar.
Hjalmar Fredrik Caesar född 31/10 1883. Smed. Gift med Gulli Maria Teresia Franzén.
Emma Vicktoria Caesar född 16/1 1887. Gift med August Elof Karlsson 1916, innan dess har hon arbetat som tjänarinna  på Vagnhester Kåragård.
Hilma Teresia Caesar född den 27/7 1890 i Bärbäcksbron, Alseda.

Hilma gick i Alseda kyrkskola, och konfirmerades i Alseda kyrka.
Därefter blev det att ta anställning som piga på olika gårdar.
Hon arbetade på Vallsjö gård, Sävsjö, Nyaby Berg, Slättåkra, hos Carl Zakrisson Repperda, och hos Emil Gustavsson Repperda Sunegård.

Johan August dör 18 oktober 1903 av vattusot, dvs. ödem i kroppen.

När Johan August dör 1903 övertas smedjan av sonen Hjalmar Fredrik. Han gifter sig med Gulli Maria Teresia Franzén 1917 bygger om och till stugan i Bärbäcksbron. Johanna får dock bo kvar där några år. Men lönsamheten i smedjan blir allt sämre, och han måste söka arbete på annat håll. Stugan i Bärbäcksbron säljs 1923 och Johanna flyttar till dottern Hilma Cederblad i Ädelfors. Efter en tid där flyttar hon till dottern Emma Karlsson i Slättåkra, där hon bor några år innan hon får sluta sina dagar 23 juni 1930. Hon är då åderförkalkad och har hjärtfel.

\section{Bouppteckning}

\startblockquote
År 1904 den 16 Januari instälde sig undertecknade i Bärbäcksbron under Trollhälla i Alsheda Socken, att förrätta bouppteckning efter aflidne Smeden Johan August Caesar, som afled den 18 Okt 1903 och efterlämnade Enkan Johanna Lovisa Johansdotter samt 6 med henne ägande barn. Sonan Gustaf Alfrid Irenius, sonen Hjamlar Fredrik myndiga och närvarande, dottren Frida Lovisa Josefina, som vistas i Norra America, dottren Helga Emelia myndig, dottren Emma Victoria född den 16 Januari 1887 och dottren Hilma Theresia född den 27 Juli 1890.

Som ingen godvilligt åtog sig godman- och förmyndareskapet, för ofvannämnda dottren Frida Lovisa Josefina och omyndiga döttrarna Emma Victoria och Hilma Theresia får vi vördsamt anhålla att Häradsrätten täcktes såvidt möjligt tillsätta någon lämplig sådan. Enkan uppgavf boet, sådant det befanns, då mannen afled, samt uppvisade alla boets handlingar, då mannen afled, samt uppvisade alla boets handlingar och skrifter, hvarefter uppteckning företogs i följande ordning:

[Tabell]

Under edlig förpligtelse att icke något är med vett eller vilja dold i eller utelämnat, utan riktigt uppgifvet underskrift
Bärbäcksbron som ofvan
Johanna Johansdotter
Rederligen upptecknadt och värderadt intyga
C.J. Hagberg        O.E. Rudvall
Bouppteckningsman
Närvarande vid förestående förrättning
Gustaf Caesar        Hjalmar Caesar

\stopblockquote

