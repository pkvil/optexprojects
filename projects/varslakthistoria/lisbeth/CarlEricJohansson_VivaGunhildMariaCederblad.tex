% Carl-Eric Johansson och Viva Gunhild Maria Cederblad
%

\couple{Carl-Eric Johansson}{Viva Gunhild Maria Cederblad}
\label[CEJ_VGMC]\wlabel{}

C-E och Viva träffas så småningom och blir ett par. Detta är under pågående 2:a världskriget så C-E blir inkallad som beredskapsman och blir stationerad i Skåne. Paret förlovar sig och innan bröllopet blir det björkdrag i Ädelfors. Detta är en väldigt gammal tradition som går ut på att brudparets vänner på lysningsdagen kommer dragandes på en stor björk och överlämnar den. Av björken ska sedan brudgummen tillverka någon möbel till det framtida egna hemmet. Det blev Vivas bror Georg som svarvade ljusstakar och nötknäckare av björken.
23 oktober 1943 vigdes C-E och Viva i prästgården i Åseda av prosten dr Isak Krook. Brudpar och släkt åkte sedan till Ädelfors för bröllopsmiddag hos Vivas föräldrar Anders och Hilma Cederblad. Faster Gullan minns att Vivas bror Henry spelade dragspel. Vid hemkomsten till Vivas lägenhet så blir brudparet och den närmsta släkten bjudna på kaffe av Vivas granne, Gerda Olsson.

Efter vigseln måste Viva sälja sin salong och börja sitt nya liv som gift. Det var ovanligt att kvinnan fortsatte att arbeta när hon hade gift sig.

Vissa omständigheter gör att deras första tid som gifta blir omvälvande för båda två. Viva är gravid i femte månaden och måste nu sälja sin damfrisering och flytta från sin lägenhet. Carl och Linnéa ger dem husrum i sitt hus, ett rum och kök i en tillbyggnad mot trädgården.
Det måste ha varit en svår lärotid, tills C-E själv blev skräddare. Han fortsatte att arbeta för sin far ända till i början av 1960-talet då han fick ny anställning på Garpens konfektionsfabrik. På sin fritid ägnade sig C-E åt fotboll, ishockey, bandy och tennis. På den här tiden (1930,-40-50 och -60 talen) fanns ingen särskild anställd som klippte gräset på fotbollsplanen eller på vintern spolade is i ishockeyrinken m.m. Utan det var de aktiva inom vardera sport som fick hjälpa till med det. Det var även problem med utrustningen när pengarna inte räckte till nyköp. C-E hade t.ex. inte råd att köpa sig ishockeybenskydd så han sydde ett par istället.

[Jennys intervju av C-E om hans fotbollsliv.]

\startblockquote
Min morfar Carl-Eric Johansson spelade mycket fotboll som liten men han började inte att spela på allvar förrän han fyllde 18 år och fick vara med i Åseda IFs juniorlag 1938. Han fortsatte sedan att spela i B-laget där det gick mycket bra vilket t.ex. en match mot Fagerhult visade. Åseda gjorde mål på mål och till slut tog Fagerhultspelarna bollen och lämnade planen. Men spelet återupptogs sedan domaren vädjat och hotat om serieavstängning. Matchen slutade med resultaten 15-1. 1941 spelade han sin första match i A-laget som låg i division 1 i sydöstra gruppen i Smålandsserien. Morfars position i laget var vänsterinner (yttermittfältare). Han var lagets speluppläggare och duktig på att nicka.

Morfar tycker att spelet i matcherna var mycket bättre förr. Spelet var inte lika snabbt som idag utan byggde mer på passnings- och växelspel. Spelet var uppbyggt på yttrarna medan det idag mest bygger på kedjespelarna. Det var inget hårt spel, utan ett spel utan hårda närkamper oc hglidtacklingar. Morfar spelade i 26 år utan att få en enda allvarlig skada.

Det fanns inga röda och gula kort, men domaren kunde ge varningar. Han kallade då till sig spelaren och skrev upp varningen. Deras utrustning på matcherna var ungefär samma som idag. Det enda som skiljer är fotbollskorna. Skorna var gjorda av läder och de gick upp över anklarna därför var de rätt så tunga. Dubbarna var också gjorda av läder vilket gjorde att dessa slets väldigt snabbt eftersom man spelade mycket på grus.

Morfars roligaste minne är när Åseda vann Smålandsserien div. 1. De vann då över Blomstermåla på hemmaplan med 3-2. Både Åseda och Älmhult hade då vunnit varsin division i Smålandsserien. Åseda den sydöstra och Älmhult den sydvästra. Båda lagen skulle nu kvala mor varandra till Sydöstran. Sedan de båda lagen vunnit var sin match skulle en avgörande spelas på en neutral plan, Värendsvallen i Växjö. I första halvlek spelade Åseda totalt ut Älmhult och ledde med hela 4-0 i halvtid. I andra halvlek var planen regnvåt och hal och Älmhult började att reducera och vid full tid var ställningen 5-5. Första förlängningen på 30 minuter blev utan mål men i andra förlängningen efter 2 1/2 timmes speltid lyckades Älmhult göra segermålet 6-5.

Något år tidigare hade det ryska fotbollslaget Dynamo Moskva vunnit stort över det svenska topplagen. Då hade man talat om dynamopiller, ett mystiskt preparat som kunde höja prestationsförmågan. Efter den här matchen gick det då rykten om att en läkare hade givit Älmhultspelarna så kallade dynamopiller i halvtidspausen. Men de dynamopiller som senare dök upp på marknaden var bara vanligt druvsocker. Mer troliga saker till att Älmhult vann var att de ändrade om sin laguppställning i den första halvleken och att Åsedas duktiga centerhalv blev skadad.

Ett annat roligt minne är att morfar sate 36 straffar i följd. Han fintade bort målvakterna likadant varje gång. Han lade upp bollen på straffpunkten, ställde sig några meter snett åt vänster bakom bollen så att han själv, bollen, och högra målstolpen var i linje. Sedan tog hansats och precis innan han slog till bollen fintade han genom att titta i vänstra hörnet och markera med kroppen som om han skulle slå den i vänstra hörnet. Målvakten slängde sig då åt vänster och han behövde bara rulla in den i högra hörnet.

En gång innan en match sa målvakten till morfar att om det blev en straff så skulle han ta bollen. Morfar sa då att om det blev straff så skulle han skjuta den i högra hörnet. Och så blev det en straff i första halvlek. Morfar lade upp bollen, fintade och sköt den i högra hörnet och målvakten slängde sig åt vänster. Sedan blev det en straff till i andra halvlek. Morfar sa återigen att han skulle skjuta bollen i högra hörnet och målvakten slängde sig i det vänstra hörnet. Målvakten blev då så arg att han började jaga morfar runt planen och kickade till honom därbak.

En anna rolig match var när Åseda vann mot Torsås med 18-3. I halvlek stod det 9-0 och morfar tippade mest på skoj att det skulle bli 18-3 och han fick rätt. I den matchen gjorde Åsedas center 10 mål. Efter den matchen blev Åseda nerbjudna att spela mot Kalmar FF på Fredrikskans. Åseda gjorde en mycket bra match men tyvärr blev resultatet 4-3 till Kalmar.

Morfar slutade spela fotboll när han var 42 år och hade då hunnit med att spela 565 A-lagsmatcher. Att morfar höll på med fotboll väldigt läge visare en B-lagsmatch mot Alstermo. Då spelade morfar tillsammans med två andra spelare som också varit med ett tag. De tre tillsammans utgjorde lagets innertrio och tillsammans var de 134 år.

[Bild: Laget. Bildtext: Åsedas lag som 1948 vann Smålandsserien division 1 med min morfar Carl-Eric Johansson längst till höger.]

[Bild: Unga fotbollspelare. Bildtext: "'Lyckan är att vara ett lag. Egentligen spelar det ingen roll vem som gjorde målet, bara det blev gjort. Lyckan är att känna att man gjorde sitt bästa inte för sig själv utan för laget.' (ur boken Fotbollens kval och lycka av Tore Nilsson.)"
\stopblockquote

 C-E får inte tillräckligt med lön av sin far för att kunna försörja sin familj utan måste låna av Vivas sparade pengar. De lägger upp en avbetalningsplan som C-E lyckas hålla. I mars 1944 föds Bengt och det dröjer ända till maj 1952 innan nästa barn, Lars, utökar familjen. Någon gång mellan 1952 och 1956 flyttar de till en alldeles ny hyreslägenhet på Södra Esplanaden i Åseda. I december 1956 föds en dotter Lisbeth.
Slutet av 1964 har C-E fått ett arbete som förman på Bestons konfektionsfabrik i Berga och mellan jul och nyår flyttar familjen till en lägenhet på 2 rum och kök. Det sista som lastas på flyttbilen är den klädda julgranen, Viva vill absolut ha med den. 
Hyreshuset på Ringvägen är alldeles nybyggt men den stora nyheten är att vi har en egen telefon. I Åseda hade vi inte det utan när vi behövde ringa, så fick vi gå till telefonstationen och betala en avgift för att använda deras telefon. Det var olika avgifter om man ringde riks- eller lokalsamtal.

Arvegods från Carl-Eric och Viva: guldringen med en liten vit sten fick mamma av pappa en gång för länge sedan. Hon hade den alltid mellan sin förlovnings- och vigselring. Glasskålen på fot fick Anders och Hilma i lysningspresent 1913. Taklampan som jag har är köpt i Åseda.
Stekgrytan med mammas namn och ett årtal på locket är en lysningspresent från min farmors syskon. Den är tillverkad i Klavreströms Bruk.
Halsbandssmycket i form av en gulddroppe som Jenny har är ursprungligen min farmors ringar.


Vivas syskon:

Georg Cederblad, född 24/9 1913 i Smedstugan, Ädelfors. Gift med Ingrid. Två barn, Åke och Kristina.
Georg första arbeten får han i skogen och på ett sågverk i Ädelfors. 1935-1948 arbetar han i Källströms möbelfabrik och sedan hos Bröderna Granlund. 1952 till sin pensionering i Hermanssons möbelfabrik i Fluguby.
Hans fru, Ingrid, arbetar som sömmerska i en syateljé som ägs av Sjöberg.
Georg byggde hus mittemot Cederslund och kallade det för Lillevång, (nästan alla hus i Ädelfors har ett namn).

Henry Cederblad, född 29 juni 1916 i Smedstugan, Ädelfors. Gift med Inga.
Han arbetar liksom som sin bror i Hermanssons möbelfabrik i Fluguby.
Hans fru, Inga, övertog tjänsten som stationsföreståndare 1934 på Tällängs station efter sin far, Oskar Tapper. Han hade dock ansvaret för stationen tills Inga blev myndig. 1937 får Oskar nytt arbete på Ädelfors station där han arbetar till 1943. Inga övertar hans arbete och blir kvar till 1961. Poststationen dras in då persontrafiken läggs ner på sträckan Vetlanda-Gårdveda. Inga förflyttas till Alseda station och arbetar där som platsvakt fram till 1966 då också den poststationen dras in. Hon får anställning på posten i Vetlanda där hon är kvar till sin pension.

