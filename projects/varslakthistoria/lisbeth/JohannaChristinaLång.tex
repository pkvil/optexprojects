% Johanna Christina Lång
% Lisbeths morfars mor

\person{Johanna Christina Lång}{Lisbeths morfars mor}
\label[JCL]\wlabel{}

Carl Johan Pettersson Glyck född 2 mars 1789 i Övlandehult Backegård, Ökna.
Han blir antagen 17 december 1810 till Smålands lätta dragoner, kompani Vetlanda Skvadron, rote: Ämmaryd Björsagård, Alseda. Dragon nr 40.
1833 blir han antagen till Smålands Husarregemente, Vetlanda Skvadron, rote: Ämmaryd Björsagård, Alseda. Han var 5 fot 7 tum (1,68 m).
Dragon, soldat som förflyttade sig med häst men som stred till fots. Beväpnad med musköter eller karbin samt blanka vapen.
Död 3 mars 1869 i Ämmaryd Björsagård, Alseda.

Lång är ett soldatnamn och den förste i släkten är Carl Johan Carlsson född 5 maj 1818 i Ämmaryd, Alseda. Han blir antagen 14 oktober 1841 som grenadjär vid Smålands Grenadjärbataljon, Östra Härads Kompani, Skirö socken, Rote Snuggarp med efternamnet Lång. Grenadjär är ursprungligen (1700-talet), en soldat som vid strid hade till uppgift att kasta handgranater (franska grenad). En grenadjär skulle vara minst tre alnar lång (1,80 m) och av ståtligt utseende. Senare används titeln om en infanterist vid ett elitförband.
Han var 6 fot lång vilket omräknat blir 1,80 m.
Dör 6 juni 1877 i Sköndal, Skönberga, Alseda.

Johanna Christina född på ett soldattorp tillhörigt gården Snuggarp (nuvarande Skönberga) i Skirö socken 1 juli 1843. Dotter till Carl Johan Lång och Anna Johansdotter född 2 november 1817.
