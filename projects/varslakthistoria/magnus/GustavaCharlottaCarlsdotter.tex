% Gustava Charlotta Carlsdotter
% Magnus ???

\person{Gustava Charlotta Carlsdotter}{Magnus mormors mor}
\label[GCC]\wlabel{}

Gustavas far Carl Nilsson har ättlingar som vi kan följa tillbaka till slutet av 1400-talet, beroende på att förfäderna tillhört prästsläkter i trakterna Eksjö och Ingatorp. Den äldsta vi hittar bland Carls förfäder var Olof Månsson som var riksdagsman och krögare i Vimmerby. Han var med vid Karl den niondes kröning i Uppsala där han under ett gräl blev mördad. Hans dotter Botilda gifte sig med en kyrkoherde från Ingatorp med det fantasifullt tagna namnet Nicolaus Petri Ingatorpensis.

Några generationer senare bryts traditionen att bli präst som sin far och Knut Rosinius blir bonde i Ekeby Högsby. Denne Knut gifter sig med Maja Lisa Staf som troligen har anknytning till Staby Gästgiveri. Knut och Maja-Lisa får en sån som heter Emanuel. Emanuel gifter sig med Jungfru Ulrika Elisabeth Fock från Långhult. Detta par är Carl Nilssons farfar och farmor.
I en artikel i Vimmerby tidning kunde man läsa om Astrid Lindgren förfäder och där kan man se att även hon har riksdagsmannen Olof Månsson som förfäder.

Carl Nilsson född 1817 i Lixhult, Högsby. Gift med Johanna Gustafsdotter född 1822 i Kalvehorvan, Fliseryd.
De får barnen: Anna Christina 1843,emigrerar till Nordamerika, Johan August 1849 emigrerar till Nordamerika 1887, fd.marinsoldat med efternamnet Rosén, Per Alfred 1851, Ida Sofia 1853, Gustava Charlotta 1854, Johanna Fredrika 1858 och Karl Emil 1862 emigrerar till Nordamerika, ändrar efternamnet till Nelson.

Gustava Charlotta börjar arbeta som piga på fyra olika gårdar i Högsby mellan 1871-1876 bl.a hos apotekare Landstedt. 1876 flyttar hon till Oskarshamn och arbetar där som piga hos handlare Axel Lundgren men efter ca ett år flyttar hon tillbaka till föräldrarna i Lixhult.

Carl Nilsson dör 1915 och Johanna dör 1900, båda två i Forsaryd, Fågelfors.

