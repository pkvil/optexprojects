% Oskar Karlsson och Maria Kristina Sköld
%

\couple{Oskar Karlsson}{Maria Kristina Sköld}
\label[OK_MKS]\wlabel{}

Oskar och Kristina gifter sig 24 mars 1897 i Mörlunda kyrka och paret bor till en början hos Kristinas föräldrar i Hultsnäs. Deras första barn, Einar, föds där 1898 men kort därefter flyttar den lilla familjen till gården Lumsebo ca 800 meter från Hultsnäs. Andra barnet, Oskar Reinhold Emanuel föds 1 juli 1899 men hans liv blir kort. Redan dagen efter dör han av svaghet. Det finns en anteckning i kyrkoanteckningarna att “Fadren, som sjelv verkstälde nöddopet. Barnet erhöll nöddopet som ej blef bekräftadt.”
Livet går vidare och snart kommer fler barn: Astrid 1900, Elna 1902, Agda 1904, Märta 1907, Sigurd 1910, Aina 1913.
1914 flyttar familjen tillbaka till Hultsnäs och här föds deras sista barn Erik 1918. Det var en tvillingfödsel men hans tvilling dog vid förlossningen.

Kristinas far har dött 1912 och hennes mor är undantagsänka i Hultsnäs så det blir naturligt att Oskar tar över gården i egenskap av hemmansägare.
Familjen bor i Hultsnäs i Mörlunda socken. I familjen ingår även Oskars svärföräldrar Carl Fredrik och Lena Christina Sköld. Sigurds första barndomsminne är när hans morfar har dött 1912. Av någon anledning så dröjer det innan han körs till Mörlunda för begravning så han läggs i en kista och ställs ut i ett uthus på gården. Sigurd är då ungefär 2 ½ år.

Ett annat minne rör en gammal gumma som familjen bor granne med. Hon är fattig och bor ensam i en liten stuga som ligger ett par hundra meter från Hultsnäs. Sigurd och hans mamma gick ofta dit med mat till henne.
När Sigurds mormor dör 1918 flyttar familjen till Lilla Boda där sista barnet föds. Nu är de 8 syskon. Ytterligare två barn har dött vid födseln 1899 och 1918.
Troligen är Lilla Boda en släktgård på Oskars sida.
Den 17 april 1923 går flytten till Hässlås,[bild finns på huset] Fågelfors. Familjen har utökats med två barn, Rut som faktiskt var Oskar och Kristinas barnbarn (hennes mamma har dött och pappan kan inte ta hand om sitt lilla barn) och en fosterpojke som heter Lennart Gustafsson.

Oskar blir dement de sista åren, det är svårt att ta med honom på t.ex släktkalas så han får stanna hemma. För att han ska ha något att göra så får han en liten uppgift som de vet att han klarar av. En gång sa hans son att han kunde bära in ved i köket medans alla var borta. Det gjorde Oskar med besked, han fyllde hela köket tills det inte fick rum ett vedträ till.
Valborg har berättat om ett besök i Hässlås, lilla Ann-Mari kryper omkring på golvet i lugn och ro. Plötsligt hör Valborg att Ann-Mari gråter och springer in i rummet där även Oskar befinner sig. Ann-Mari sitter gråtande under ett bord med alla stolar ordentligt inkörda. Oskar är väldigt nöjd, han talar stolt om att han har stängt in alla kalvar så de inte smiter sin väg.




Sigurd tar körkort, han har berättat hur det gick till. Efter några körlektioner i Oskarshamn så är det dags för uppkörning, det går bra. Teoriprovet består av två frågor, den ene var så här: Får man köra bil om man har druckit sprit? Sigurd svarar nej, det får man inte. Då säger körskoleläraren: Jo, lite får man ta. Körkortet utfärdas 21 april 1931.

Sigurd hinner med många yrken i sin karriär. Bland de första arbeten han hade var som snickare på Nya Fabriken i Fågelfors. Samtidigt hade han extrainkomst hemma på gården i Hässlås, en silverrävsfarm. Skinnen såldes och det blev lite extra pengar i mitten av 1930-talet. Sedan arbetade han som murare hos Holmström i Fågelfors. Under andra världskriget hade han kolugnar i Hässlås och sålde kol till Försvaret. Sigurd berättade hur Fogelfors bruk såg honom som konkurrent och försökte svälta ut honom genom att köpa in all ved till överpris. Men en dag kom folk från bränslekommissionen och undrade varför hans kolugn stod stilla när behovet av kol var stort, han berättade som det var, varvid kommissionen löste in brukets ved som Sigurd fick ta över. Inköparna på bruket stod där med lång näsa. Efter kriget köpte han en lastbil och körde grus, samt använde den till att hämta mjölk hos bönder och köra den till mejeri.

Sigurds syskon

Einar, 1879-1971, gifter sig med Naemi  Karlsson och får två barn, Berit och Alf. Han tar över Hässlås efter sina föräldrar. Familjen flyttar sedan till Hultsfred.

Astrid, 1900-1960, gifter sig med Gunnar Thyrén och får en son, Enar. Hon har eget företag i Fågelfors, kafé och bageri som är beläget i huset bredvid Filadelfiahuset.

Elna, 1902-1930, gifter sig med Karl Gunnar Karlsson och får en dotter, Rut. Elna dör när Rut är 1½ år, pappan kan inte ta hand om henne utan hon växer upp i Hässlås hos sina morföräldrar.

Agda, 1904-1990, gifter sig med Henry Bohm och får två barn, Gerd och Roland. Agda var en händig person, hon sydde både till familjen och andra, målade om och tapetserade när det så behövdes i hemmet. Hon var duktig vid vävstolen, kunde väva många invecklade mönster. Henry var en spjuver, han tyckte om att skoja och tävla om de mest besynnerliga saker. Han hade Fågelforsrekord i att kunna joja längst, han höll på hela natten och vann till slut. Tävlingen om vem som kunde äta mest senap vann han naturligtvis.

Märta, 1907-1978, var särlingen i familjen. Hon reser till Stockholm och söker lyckan. Hon gifter sig med Herman som senare visar sig vara oärlig. Han säljer gödsel, men det är bara vatten och för det får han sitta av sitt straff på Långholmen. Innan dess flödade pengarna, paret bjuder Märtas föräldrar till Stockholm och tar med dem till Berns Salonger. Herman dör ganska tidigt och Märta berättar att Herman går igen och visar sig som en liten figur i deras hem.Efter detta har Märta tre förhållanden. Yngve, Gustav och Lennart. När Märta dör på 1970-talet åker Sigurd och hans bror Erik till bodelningen i Stockholm. Sigurd får bl.a ärva väggklockan av guld.

Aina, 1913-1991, gifter sig med John Enocksson och får en dotter, Ann-Britt. Aina arbetar en tid hos Astrid på kafét.

Erik, 1918-1980, gifter sig med Eva Petersson och får tre barn, Monika, Inga-Lill och Carola. Samtidigt som Sigurd bygger huset Björkhaga på Parkvägen så bygger Erik sitt hus mittemot Sigurds. Kort efter att Sigurd med familj flyttar till Kvillehult så köper Erik en gård som heter Rydet och ligger på vägen till Virserum. Den äldsta dottern, Monika, vill se världen och hamnar i Argentina där hon bor med sin familj i många år. Inga-Lill gifter sig med Sixten Jacobsson och får en son, Mikael. Hon blir änka tidigt. Bor i Virserum. Carola gifter sig med Jan Strömberg och får tre barn. Efter Eriks död tar Carola med familj över Rydet och flyttar dit.

Fosterbror Lennart Gustafsson gifter sig med Berit och får tre barn. Familjen bor i Fågelfors och räknas som släkt hela tiden.



Ett barnbarn till Kristina, Gerd Sinnerström, har delat med sig om sina minnen om sin mormor Kristina. Både händelser som hon har fått sig berättat eller själv har upplevt.
“När Kristina i unga år var på väg till skolan en dag så skadade hon sitt knä så illa, oklart hur,  att hennes ben efter det blev krokigt och hon behövde gå med hjälp av en krycka. På äldre dagar använde hon kryckan till att lägga den böjda delen om två av sina barnbarns barn hals och hala in dem till sig medan hon sa “Kom nu ponten“. Magnus var en av dem, hans kusin Ulf den andre. Båda två var lite rädda för henne och tyckte inte om att bli inhalad på det viset.

Kristina var en stabil, rättvis kvinna som visste vad hon ville och var inte rädd av sig. När Oskar och Kristina köpte gården Hässleås utanför Fågelfors 1923, så kördes flyttlasset med häst och vagn från Hultsnäs. Det var Kristina som höll i tömmarna,men hon hade inte varit på väg mer än en kort sträcka förrän olyckan var framme. Vagnen välter och rullar ner för en liten slänt och hela lasset åker ut i naturen. Ingen blev dock skadad och när häst, vagn och bohaget var uppe på skogsvägen så fortsatte hon och hennes familj resan till nya huset.

Gerd och hennes bror Roland hälsade ofta på hos sin mormor och morfar i Hässleås. Oskar börjar tycka det är tungt att sköta gården, så äldste sonen Einar tar över gården 1937. Hans syskon får 1000 kr vardera och han blir ensam ägare. Einar gifter sig med Naima  och får två barn i samma ålder som Gerd och hennes bror. De leker ute på gårdsplan och när de blir hungriga springer de in till Kristina, som bjuder dem på hennes goda bröd, som hon har bakat i den stora bakugnen i köket. De får stora brödskivor med hemgjort smör på. Gerd minns ännu hur gott det var tillsammans med mjölk.

Åren gick och Oskars hälsa är inte den bästa, men Kristina som fortfarande tycker om att träffa folk får hjälp av dottern Astrid och hennes man Gunnar att t.ex åka och handla. Hon får också följa med dem på Filadelfias olika tältmöten vilket hon uppskattar, hon är medlem i den kyrkan. Under tiden är Oskar hos sin dotter Agda, han kan inte vara själv längre utan behöver tillsyn hela tiden.

Oskar drabbas av demens och Kristina orkar till slut  inte sköta honom själv. Dottern Astrid tar emot dem i sitt hem och det är en stor hjälp för dem båda. 
1949 dör Oskar. Astrids café och hem ligger mittemot skolan som Gerd går i. Hon tittar ut genom fönstret  och får se begravningsbilen som hämtar Oskar. Det blev en väldigt uppskakande upplevelse för henne som hon minns in i minsta detalj än i dag.
10 år senare, 1959, samlas barn och barnbarn runt Kristinas bädd. Hon har inte långt kvar att leva och alla hjälper till att vaka hos henne intill slutet. Även Gerd sitter hos henne och baddar hennes läppar. Det blev en både sorglig och fin stund som hon hade med sin mormor.









Bild

Oskar och Sigurds möten med gårdens tomtar. [Fylls i av Magnus]
Djur på gården lite senare i tiden. Sigurds arbetsliv fram till han gifter sig. Syskonens liv.

