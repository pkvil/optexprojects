% Selma Terese Åberg
% Magnus morfars mor

\person{Selma Terese Åberg}{Magnus morfars mor}
\label[STÅ]\wlabel{}

Namnet Åberg är ett vanligt namn i Fågelfors med omnejd och de flesta av dessa har nog gemensamma förfäder. Går man på kyrkogården i Fågelfors idag så hittar man namnet Åberg på många gravstenar. Åberg-släktet i Fågelfors är en smedsläkt på fyra generationer som man kan härleda tillbaka till en Nils Larsson född i Rumskulla och som sedan fick namnet Åberg. Han blev smedmästare, i betydelse av att han var fullt utlärd smed vid Fredriksfors bruk i Döderhult.

Där arbetar också hans son Lars Åberg, även han smed. Han får en son, Johan Peter Åberg, som arbetar som hjälpsmed vid Pauliström bruk, Målilla socken. När Fredriksfors järnbruk lades ner flyttades verksamheten till Fågelfors och släkten Åberg följde med dit. Johan Peter får arbete som stångjärnssmed vid Fågelfors bruk. Fjärde generationen Frans Oskar Åberg arbetar som hammarsmed vid Fågelfors bruk.  

Frans Oskar född 20 juli 1837 i Storebro, Vimmerby socken gifter sig 7 september 1861 med Johanna Gustava Jonsdotter född 5 juli 1838 i Klebo, Högsby socken.
De får elva barn däribland Selma Therese. Hon är född 19 juni 1874. 
Hennes syskon: Ida Sophia 1859, Gustava Charlotta 1862, Carl Oscar 1867-1868, Carl Oscar 1869, Emilia Maria Josephina 1871-1872, Axel Reinius 1873-1873, August Bernhard 1882-1882, tvilling Alma Emilia 1882.
Två av barnen, Gustava Charlotta och Gustaf Emil, emigrerar till Nordamerika 1887.

Frans Oscar dör 1888 och Johanna Gustava dör 1912.
