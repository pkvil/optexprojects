% Erik Magnus Nilsson och Gustava Charlotta Carlsdotter
%

\couple{Erik Magnus Nilsson}{Gustava Charlotta Carlsdotter}
\label[EMN_GCC]\wlabel{}

De gifter sig 9 november 1878 och bosätter sig i Hyltan, Fågelfors. Erik Magnus arbetar som gårdsrättare på Fågelfors Järnbruk. 
De får barnen: Edvard Manfred Severin 1888, Erik Gunnar 1890 och Elvira Kristina 1891.
 
Erik Magnus sköter sitt arbete bra, han blir befordrad till rättare på Fogelfors Bruk och familjen flyttar 1891 till en tjänstebostad i Fåglebo, utanför Fågelfors samhälle.
Två barn till: Anna Martina Charlotta 27 juni 1893 och Carl Arvid Henry 1896.

[Bild på familjen utanför huset.]
Anna Martina Charlotta föddes i Foglebo Fågelfors den 27 juni 1893. Föräldrarna var Erik Magnus Nilsson som var rättare på Fogelfors bruks gård i Foglebo och hennes mamma hette Gustava Charlotta Carlsdotter var född 3 november 1854 i Lixhult Nr.3.
Anna växte upp med sina fyra syskon Manfred född 1888, Gunnar född 1890, Elvira född 1891 och Carl född 1896. 
Under åren 1910 till 1915 bestämmer sig alla hennes syskon för att söka lyckan i Amerika och utvandrar. Anna stannar kvar och tar hand om föräldrarna. Syskonen som utvandrade fick förmodligen hjälp från äldre släktingar som utvandrat tidigare. De finner sig tillrätta i USA och får arbete, gifter sig och får barn, så man kan tänka sig att det finns fler släktingar från den här grenen i USA än här i Sverige.
De håller kontakt med Sverige genom brev och längre fram även resor för att hälsa på sina släktingar.
Anna gifter sig 1919 med Arthur Bom.

Anna är det enda barnet som stannar i Sverige och tar hand om föräldrarna. Hennes syskon emigrerar till Nordamerika mellan 1905-1915. 

[Info om golf, m.m]

[Erik f.d rättare, ny flytt.]

