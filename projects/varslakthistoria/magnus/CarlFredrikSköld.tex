% Carl Fredrik Sköld
% Magnus farmors far

\person{Carl Fredrik Sköld}{Magnus farmors far}
\label[CFS]\wlabel{}

Magnus Danielsson född 22 april 1791 i Mossäng, Kråkshult socken. Antagen till soldat år 1817 vid Kungliga Kalmar regemente, Förste Majorens kompani. Han får soldatnamnet nr.~58 Brandt och blir tilldelad soldattorpet Lamåsa Kåragård i Alseda. Han bor även på Språklaretorpet, Alseda.
Enligt en notering generalmönsterrullorna 1926 är Magnus Brandt 35 år gammal, har tjänstgjort i elva år och är 5 fot 9 tum, dvs 1,72 m lång. Gift med Stina Lisa Johansdotter född 1797 och har med henne 8 barn: Johan Magnus 1817, Jonas Peter 1820, Stina Cathrina 1823, Anna Lena 1826, Carl Fredrik 1929, Johanna Maria 1832, August 1835 och Carolina 1838.

1832 flyttar familjen till Mönsterskrivare boställe, nr.~79 Hökaryd i Åseda socken.
Magnus dör 1862 och Stina Lisa 1846.

Sonen Carl Fredrik Magnusson född 22 juli 1829 i Alseda. Antagen 9 augusti 1855 vid Kalmar Regemente, Livkompaniet, Första Korpolagskapet. Han får soldatnamnet Sköld och bor nu på nr. 3 Nyttorp, Tulunda i Mörlunda socken. Han är 6 fot 2 tum, dvs. 1,85 m lång.
Den 29 juni 1883 slutar han sin soldatbana, i generalmönsterrullorna finns en anteckning att “han har tjent utmärkt väl“, och blir gratialist. Det betyder att han får gratifikation, en pension som utbetalas två gånger om året vid uppvisande om prästbetyg och tjänsteintyg. Nu kan han försörja sin familj på en annan gård.
