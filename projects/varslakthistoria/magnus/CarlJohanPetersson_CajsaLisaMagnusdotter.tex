% Carl Johan Petersson och Cajsa Lisa Magnusdotter tillsammans
% 

\couple{Carl Johan Petersson}{Cajsa Lisa Magnusdotter}
\label[CJP_CLM]\wlabel{}

Paret gifter sig 29 juli 1860 i Mörlunda kyrka och bosätter sig på gården Hultsnäs, Näshult i Mörlunda socken. Hennes föräldrar bor och brukar gården till 1872 då Carl Johan och Cajsa Lisa tar över ägandet.
De får tio barn mellan 1861-1881, ett av barnen dör i ung ålder.
Deras barn fick namnen: Johan Peter 1861, Amanda Sofia 1862, Carl August 1865, Oskar 1867, Constans 1869, Mathilda 1872-1872, Olivia Gustava 1873, Nils Emil 1876 (emigrerade till Nordamerika), Axel Valfrid 1879 och Josefina Emilia 1881.

 Oskar bor i föräldrarhemmet i Hultsnäs hela barndomen och uppväxt. I 20-årsåldern arbetar han på Ryningsnäs Gods som arbetskarl, mest i jordbruket. Han var även med när sjön Ryningen skulle sänkas för att skapa mer åkermark. Det grävdes en ny kanal innan Blankan. Oskar var med i det arbetslaget som sprängde och grävde kanalen. Än idag kan ett inristat vattenmärke i Haklaberget ses, V M 1887. Vattnet fick inte gå över det märket för då svämmades åkrarna över. Om detta hände så öppnas några vattenluckor. 

När gäster var inbjudna till har- och tjäderjakt  av godsägaren var Oskar och andra arbetare involverade. Men det var framförallt andjakten i Ryningen som var mest eftertraktad av gästerna. Godsets arbetare fick ställa upp som roddare och upplockare av byten. De skulle också gående genom vassruggarna få änderna att flyga upp och allra helst då i riktning mot den plats där skyttarna stod. Under de första dagarna av den lovliga andjakten sköts inte mindre än 600 änder av ett inbjudet tremannalag.

Familjen bor kvar i Hultsnäs till 1890 då de flyttar till ett närliggande hus, Bygget, som troligen inte låg så långt därifrån.

Carl Johan dör 16 december 1903. Cajsa Lisa bor därefter hos sin son Oskar i Hultsnäs och dör där 6 november 1910.

