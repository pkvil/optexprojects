% Carl August Bom
% Magnus morfar

\person{Carl August Bom}{Magnus morfar}
\label[CAB]\wlabel{}

Nils Persson född 1764 i Temshult, Kråksmåla. 
Han blir antagen som soldat 18 augusti 1788 och blir placerad på Bondeberga nr. 87 i Högsby socken. Nu får han soldatnamnet Bom.

1789-1790 tjänstgör han på Sjökompaniet, men redan 1793 slutar han sin soldatkarriär.
Han flyttar till Kloster, Kolsrum i Högsby. Under årens lopp hinner han gifta sig 3 gånger.
I husförhörslängden 1818 står han som häradstjänare, dvs. vaktmästare vid tingshuset i Staby.
En annan betydelse är fångvaktare.
Nils Persson Bom dör 17 september 1830.

Sonen Peter Nilsson Bom, född 1791 i Bondeberga, som Nils fick från det första giftet med Lena Jonsdotter, bor på gården Kloster med sin hustru Lisa Kajsa Nilsdotter och deras elva barn.

Ett av barnen heter Gustav Persson Bom, född 15 juni 1833, växer upp på gården. I mitten av 1850-talet arbetar han som dräng på en gård i Sjötorp, Fågelfors. På samma gård finns en piga som heter Christina Jonsdotter, 5 november 1833. Tycke uppstår och 1858 gifter de sig och bosätter sig på Sjötorp nr 1 som är Christinas föräldrarhem. De får sju barn:
Mathilda Sofia 1859, Carl August 1862, Gustaf Emil 1866, Johan Rudolf 1870, Axel Reinius 1874, dödfött gossebarn 1877 och Pehr Alfrid Martin 1878.

Carl August bor hemma fram tills han gifter sig och ett par år framöver.

1881 dör Gustav Persson Bom av bröstlidande. Christina Jonsdotter dör 10 december 1934, 101 år gammal. Hon kallades Moster Bomma av hela släkten.

[Bild på moster Bomma]

