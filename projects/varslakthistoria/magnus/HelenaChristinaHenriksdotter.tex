% Helena Christina Henriksdotter
% Magnus farmors mor

\person{Helena Christina Henriksdotter}{Magnus farmors mor}
\label[HCH]\wlabel{}

Helena Christina Henriksdotter. Hon är född 24 juli 1829 i Tribäck, Hattemåla, torpet Lyktan, Fagerhults socken. Hennes far heter Hindrik Jonsson, född 1803, och hennes mor heter Maria Alexandersdotter, född 25 juni 1806. Föräldrarna kommer från trakterna runt Virserum och Fagerhult där deras anfäder har bott i flera generationer.

På torpet Lyckan föds också en son Jonas Magnus 1832. En tid efter det så flyttar familjen till Liskatorp, även det i Fagerhults socken.

1834 går nästa flytt till torp Eksjögle, Backstugan i Virserum socken. Fadern är nu brukare på gården Gröndahl. Fyra söner till föds där, Gustaf Peter 1836, Carl August 1839, Johan Claes 1843 och Adrian 1847.

Helena Kristina flyttar tillbaka till Fagerhult 1844 och arbetar som piga på gården Medelgärde. Återvänder till familjen i Virserum 1846. Flyttar 1853 till Åkerö i Mörlunda socken där hon arbetar som piga. På samma gård arbetar Carl Fredrik Magnusson Sköld som dräng och de blir ett par.

