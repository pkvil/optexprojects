% Sigurd Karlsson och Valborg Bom
%

\couple{Sigurd Karlsson}{Valborg Bom}
\label[SK_VB]\wlabel{}

Sigurd och Valborg träffades på ett lite märkligt sätt genom olika personer i bägges närhet.
Valborgs farmor, Selma Åberg Bom bor på Bruksgatan i Fågelfors. När Arthurs bror Henry har gift sig med sin Agda, så får de bygga sitt hus på samma tomt.
Nu kommer det märkliga. Henry är Valborgs farbror och Agda är Sigurds syster. I och med detta så umgås bägge familjerna flitigt men det dröjer flera år innan Sigurd och Valborg upptäcker varandra.
Dvs. Sigurd gör nog det med en gång men Valborg är förlovad med en annan. Fästmannen flyttar till Stockholm och det är väl meningen att hon ska komma efter men under tiden träffar han en annan flicka och förlovningen bryts.
Tiden går och en dag får Valborg ett brev från Sigurd. Han skriver om sina känslor för henne och frågar om det finns någon chans att det ska bli de två. Han vill dessutom att hon ska bränna brevet när hon har läst det. Det gör hon ju naturligtvis inte utan hon gömmer det i en kokbok. Brevet hittar vi av en slump 50 år senare när deras hem säljs och alla saker gås igenom och delas upp.
De gifter sig 24 maj 1947 i Fågelfors prästgård. Mottagning hos hennes föräldrar i det röda huset. Hela släkten är bjudna. Nils Forsblom som är gift med Valborgs faster Elin har tagit med sig fyrverkerier till bröllopet. När han kastar upp en pjäs i luften så fastnar den i en tall som börjar brinna, men allt slutar bra, alla hjälps åt att släcka.
Det är fortfarande svårt att få köpa allt som behövs, så släkten bidrar med kuponger till bl.a mat. Valborgs faster Anna, som bor i Kalmar, kommer med en stor krokan som är 1 meter hög. Den översta delen, själva kronan på verket, sparar Valborg som ett minne. Sigurds systrar Astrid och Aina har bakat smörbakelser och kringlor i sitt bageri.
Sigurd har byggt ett hus på Parkvägen i Fågelfors som får heta Björkhaga och där flyttar brudparet in och trivs riktigt bra. Snett mittemot har Sigurds bror Erik byggt ett  hus till sin familj.  .
Men Sigurd kan inte släppa sin dröm om en egen gård och efter ett par år kommer tillfället som gör att han kan förverkliga drömmen. Valborg är mer tveksam, gården med tillhörande skog är nedgången och det krävs en stor renovering för att kunna bo där. Sigurd vill så gärna ha gården och ber henne att ge det en chans, han lovar henne att om hon efter några år inte vill bo kvar så ska de flytta till ett bättre hus. När åren har gått kan Valborg inte tänka sig att bo på något annat ställe, så de stannar kvar.

Sigurd och Valborg köpte Kvillehult 1951 av Wilhelm och Gerda Jonsson för 32 000 kr. I köpet ingick hus, ladugård, skog, åkrar och en fotogenlampa som numera hänger över köksbordet här på Åsvägen. Det krävdes många och omfattande renoveringar innan familjen kunde flytta in. Det inköptes traktor av märket Ferguson, kor, grisar och höns.
I den lilla byn Blankan, c:a 1 mil utanför Högsby, ligger gården Kvillehult. Där bor Sigurd och Valborg Kvillegård med sin lilla dotter Ann-Mari 3 ½ år gammal. Alla tre går i spänd förväntan denna februarimånad 1952, Valborg är höggravid och snart ska Ann-Mari bli storasyster. 15 februari är Valborg och Ann-Mari ute och åker spark i Blankan. På natten till den 16:e vaknar Valborg av att värkarna har satt igång. Runt ½ 5 väcker hon Sigurd och Ann-Mari. De packar vad de ska ha med sig och åker upp till Fågelfors för att lämna Ann-Mari till Valborgs föräldrar, Arthur och Anna Bom, som då bodde i det lilla röda huset på vägen till bruket. Arthur sätter på kaffe och så satt de en stund och pratade och hade det trevligt. Men vid ½ 6-tiden känner Valborg att nu är det något på gång, vattnet går och hon och Sigurd åker iväg till BB i Högsby. Barnmorskan som tar emot dem tycker att de kommer lite väl tidigt på morgonen, hon har nämligen inte hunnit dricka upp sitt morgonkaffe än. Valborg blir hänvisad till ett rum i alla fall och blir undersökt av en sköterska som upptäcker att här är det bråttom. Hon tillkallar barnmorskan som det blir fart på när hon ser hur långt det är gånget. Klockan 6 den 16 februari 1952 föds pojken som sedermera får namnen Rolf Magnus Sigurd och som ska kallas Magnus.
Valborg berättade mycket om livet i Kvillehult, hur de arbetade på gården och det var nog slitsamt många gånger för dem. Men det fanns mycket glädje i det också. Sigurds dröm om en egen gård hade ju gått i uppfyllelse och varje dag var en gåva för dem. Magnus växte upp i en varm och kärleksfull miljö. Det fanns väl tillfällen då han tyckte att livet var lite orättvist, då packade han sin lilla ryggsäck och rymde ut i skogen. Innan han gick iväg så var han noga med att tala om det för sin mor och dessutom rymde han alltid till samma gran. Efter en stund gick Valborg och ”letade” , fann honom och allt var bra igen. Den sista gången hade rymde så glömde han att tala om det för sin mor, så han blev sittandes där en bra stund innan han fann det för gott att återvända hem.
Valborg sydde en sjömanskostym till Magnus som hon ville att han skulle ha på sig vid ett högtidligt tillfälle, (oklart om det var ett bröllop eller fint kalas). Under protest gick han med på detta men så fort de kom hem igen så sa han”Ta på mig kläder så jag kan leva”.
En vinterdag är Magnus och en lekkamrat ute och leker i snön. Plötsligt är de försvunna och Valborg ger sig ut att leta efter dem. Hon följer deras fotspår i snön och de leder henne till Emåns kant. Ut på isen fortsätter spåren och hon blir alldeles kall av fasa för vad som kan ha hänt dem. Hon ropar, men hon får inget svar. Gråtande springer hon hem igen för att hämta hjälp. Väl tillbaka får hon se bägge barnen välbehållna på gårdsplan. De hade gått ut på isen men inte gått tillbaka samma väg som de kom utan gått i en båge hem igen.
Bo på en bondgård innebär att alla måste hjälpa till med det de kan. Magnus gjorde vad han kunde bl.a mata tuppen och hönsen. Men det fanns ju även tid till lek. Magnus spelar kula på gårdsplan och tuppen tror givetvis att det är mat till honom och försöker ta stenkulorna. Magnus hinner dock före honom varje gång. Tuppen reagerar med att utveckla ett oresonligt hat till Magnus med den påföljden att han jagar och försöker hoppa på honom så fort Magnus visar sig utomhus. Det går t.o.m så långt att tuppen lär sig vilken tid på eftermiddagen som skoltaxin kommer och möter Magnus uppe vid vägen. Efter en tid av tuppförföljelse så tog Magnus ett beslut som gjorde slut på eländet. Nästa gång tuppen mötte honom på väg hem från taxin hade Magnus beväpnat sig med en grov gren och med ett välriktat slag så var den tuppens liv över. En annan djurhistoria handlar om grävlingen i hönsgården, som var inhängnad, bakom ladugården, . Sigurd upptäckte att ett djur hade grävt sig under stängslet och tagit höns. En kväll kom Sigurd in i huset och sa till Magnus ” Nu är det en grävling som ska ta en höna, du får skjuta den ”. Han tar fram geväret och de smyger ut. Mycket riktigt, de får syn på en grävling på väg in i hönsgården. Sigurd osäkrar geväret och ger det till Magnus som drar av ett skott rakt på djuret som stendör. Magnus är 7-8 år när detta händer, långt efteråt inser han att grävlingen redan var död, skjuten av sin far, innan han själv skjuter sitt skott. Men just då var det ett stort äventyr.

Även en räv var intresserad av de fina hönsen och återigen tycker Sigurd att Magnus ska få den äran att skjuta. Innan far och son går ut så kommer Sigurd och Valborg överens om att när hon hör skottet så ska hon öppna köksdörren och släppa ut hunden. Visst är det en räv som står på bergknallen bakom ladugården, Magnus lägger an salongsgeväret och skjuter. I ett huj kommer hunden farande och hugger räven i strupen och skäcker den fram och tillbaka. Detta gör ju att Magnus inte märker att allt är arrangerat av sin far som är mycket stolt och berömmer sin son. Alla är nöjda, hönsen med.

Några år senare. Magnus är ensam hemma, han sitter framför elden i den öppna spisen. Han hittar en patron till salongsgeväret, börjar så sakta fundera på vad som skulle hända om han slänger in den i elden. Mest om det smäller. Sagt och gjort, han slänger in patronen och visst smäller det. Han ägnar sen en lång stund med att plocka glöd som har spritt sig i hela rummet. Han gjorde aldrig om det.

I dammen uppe vid dammluckorna är Magnus ute och ror i sin eka. Tankarna börjar vandra iväg att det kanske skulle gå att bygga om båten så att han slapp att ro. Han börjar med att spika ihop bräder till två skovelhjul som han fäster vid en axel för att sedan sätta fast dem på ekan. En gammal motorsåg, med frikoppling, vars svärd plockas bort sätter han dit i aktern på båten. Premiärturen blir inte så lyckad, skovelhjulen sitter för långt ner vilket gör att de gräver sig ner i vattnet. Efter vissa justeringar görs ett nytt försök men då går de för fort så det sprutar kaskader av vatten. Farkosten är dessutom väldigt vinglig även om Magnus sitter så stilla han kan. Denna uppfinning läggs åt sidan.

När man är i 12-års-åldern är det väldigt kul att hitta på bus, det tycker också Magnus och hans kompis Kennet Gustavsson. De har fått tag på en bit harts och bestämmer sig en kväll för att spela hartsfiol hos någon i Blankan. Det finns inte så många hus i denna by, de väljer Hilma och Josefs hem. Beslutet syns vara ganska lättaget eftersom de är gamla och vid eventuell upptäckt så kan pojkarna springa ifrån dem. Spela hartsfiol går till så här: en björntråd sätts fast i en fönsterkarm, där telefonlinjen går in, tråden spänns och sedan gnider man hartsen fram och tillbaka. Det låter alldeles förfärligt inne i huset. Hilma och Josef tror att det är en inbrottstjuv som håller på att såga av telefonledningen så de ringer till polisen som kommer snabbt som ögat. När pojkarna ser polisens billjus så finner de för gott att springa därifrån. Polisen jagar dem ända bort till Pingelhagen, där gömmer sig Magnus och Kennet. Men det finns inte så många pojkar i de åldern i byn så polisen uppsöker föräldrarna och berättar om busstrecket. Straffet för de båda blev att gå till Hilma och Josef och be om förlåtelse. Det måste ha varit snälla människor för när allt var förlåtet så satte sig alla inblandade ner och åt tårta.
På 50-talet heter familjen Karlsson, men det bor också en annan man som heter Sigurd Karlsson i Blankan och när posten har kommit fel för många gånger bestämmer de sig för att byta efternamn. Eftersom deras gård heter Kvillehult så blir det naturligt att ta efternamnet Kvillegård. Detta sker i början på 60-talet.
Bredvid Kvillehult ligger ett sågverk och bakom dess byggnader rinner en liten bäck som också går genom gårdens marker. I denna bäck rinner det ner ett giftigt ämne, Fenol, som förstör vattnet där och värre än så, dricksvattnet till Kvillehult. En process startas med anmälan om vattenförorening och tagning av vattenprover i brunnen. Det grävs en ny brunn men även den förorenas. Sågverksägaren sitter som ordförande i kommunstyrelsen med många trådar att dra i för att komma undan sitt ansvar. Han lyckas bra och det drar ut på tiden med ett beslut om vem som är ansvarig och ersättningsskyldig. Det enda som är säkert är att resultatet av vattenproverna visar att det inte får användas som dricksvatten. I skogen finns en källa med rent och klart vatten. Dit åker Sigurd och Magnus för att hämta hem vatten i dunkar till familjen.

Magnus går de första åren i Drageryds skola, några år i Forsaryds skola för att sen avsluta mellanstadieåren återigen i Drageryd. Högstadiet finns i Högsby och där upptäcker lärarna Magnus kapacitet och råder honom att söka naturvetenskaplig linje i Oskarshamn. Han blir antagen och nu går första flytten för hans del. Han hyr ett rum hos Malin Helander.

Skåpet och skänken som står i vårt sovrum har Magnus ropat in på en auktion i sin ungdom. Skänken ropade han in för 5 kr. på Abrahams auktion på 1960-talet i Blankan. Det tillhörde en spegel och två utdragslådor, dessa sålde han till sin morbror Mauritz för 10 kr.
De två väggskåpen som hänger i tv-rummet är gjorda av Josef i Blankan, från början fanns de i lekstugan i Kvillehult. Ekbokhyllan ropade vi in på en auktion i Sinnerbo på 1970-talet för 60 kr.
Taklampan med marmorhållare kommer från Nora Dahlsäters auktion i Blankan 1982. Vi fick den för 40 kr.


Arvegods från Sigurd och Valborg: det blå stora skåpet och väggklockan kommer från Sigurds föräldrahem. Vid bodelningen efter hans mor,1959, så var det ingen som ville ha dessa saker. Efter en stund säger någon av syskonen ” Ge dem till pojken ” och så blev det, Magnus fick dem. Örnen högst upp på klockan saknades. Många år senare så snidade Lindström i Fågelfors en ny. Magnus fick igång klockan och den gick så fint tills en natt då vi vaknade av att den slog och slog. När vi hade räknat till över 60 slag så fick det vara nog och Magnus stannade den. Sen dess har vi inte fått igång den mer. Det blå skåpet stod en gång i tiden i Hultsnäs ( Sigurds barndomshem ) innan det följde med till Hässlås.
Guldklockan kommer från Sigurds syster Märta. Skrivbordet som Magnus har nu var Sigurd och Valborgs skrivbord i Kvillehult. Fia-spelet köpte Valborg på en marknad i Virserum, på spelet har Pälle Näver skrivit en dikt, ”Gick här och frös, då kom en tös, solig och grann, och kylan försvann”.
Den blå mjölkpallen och mjölkkannan har varit Valborgs farmor. Den bruna kaffekvarnen är troligen hennes också, samt den lilla kaffekokaren av koppar. I Kvillehult i hallen uppe står en grön sammetssoffa som stod i Fågelfors. Selma Åberg köpte den i Vetlanda. Den gröna kaffekvarnen kommer från Sigurds föräldrar.
Tavlan med fåglarna
Magnus fick ett fickur av sin pappa när han fyllde 25 år. När Peter fyllde 25 år fick han den av Magnus.

