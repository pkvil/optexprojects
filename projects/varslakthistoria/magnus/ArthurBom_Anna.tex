% Arthur Bom och Anna
%

\couple{Arthur Bom}{Anna ?}
\label[AB_A]\wlabel{}

Efter giftemålet 1919 bor familjen de första åren på övervåningen i ett gult hus vid mejeriet på Hyltan i Fågelfors.
Första barnet föddes 18 december 1921 och fick namnet Alfa Valborg Martina.
1926 föds Valborgs syster Elin och 1930 får hon en lillebror Mauritz.

Valborg går sex år i skolan. 1937 har hon gått en sex veckor lång kurs i Kalmar Läns Södra Landstings skolkök. I kursen ingår enklare matlagning, bakning, kortfattad födoämneslära samt ordning och renlighet. Från sommaren 1939 arbetar hon som köksbiträde i två månader och som barnhusa i ett år och sex månader hos familjen Ekströmer på deras herrgård i Fågelfors. Tiden på herrgården är Valborg väldigt stolt över hela sitt liv. Det var nämligen bara flickor som kom från ”fina” familjer som fick arbete där. 1941 tjänstgör hon på barnkolonien Walldahemmet i Halland mellan 15 juni och 15 augusti. 15 juni 1942 och 15 januari 1943 arbetar hon som hembiträde hos Selma Henriksson i Djursholm. 1946 går hon en kurs i vävning i regi av Södra Kalmar Läns Hembygdsförening.

Barndomsminne: Elin och Valborg sover i en utdragssoffa. En kväll blir de törstiga och stiger upp för att dricka vatten i köket. Direkt efter att de är ur bädden så trillar väggklockan ner och slår upp ett stort märke i soffan och hamnar precis där flickorna har legat. Där var nog en änglavakt inblandad.
Kökssoffan står nu i Kvillehult och klockan hänger på väggen här i köket på Åsvägen. Klockan har en egen historia, släkt från Amerika har haft den med sig vid ett Sverigebesök. Det fattas en målad bild i ”glasfönstret”. Den fick några släktingar som minne av dem som hade skickat klockan en gång i tiden.

Elin, 1926, gifter sig med Ruben Franzén 1961 i prästgården i Högsby.  Elin bär en vit, lång klänning, brudkrona och har röda rosor i sin brudbukett. Bröllopsmiddag intas på Hotell Morén. De träffades redan 1945, men av olika anledningar dröjde bröllopet. Elin utbildar sig till hemvårdarinna. Innan giftermålet bor och arbetar  hon i Mönsterås och Kalmar. Efter vigseln köper paret ett hus i Nybro. Elins sista arbete blir på Madesjö skola. Ruben är ingenjör och vägmästare.

Mauritz, 1930, gifter sig med Waldy Johansson i Kalmar slottskyrka. Mottagning efteråt i deras bostad på Storgatan i Högsby. De får två barn, Anders och Birgitta.
Ett starkt minne från den 28 januari 1935: Mauritz sitter på sin mammas arm och de står och tittar på den stora branden vid Fogelfors Bruk. Han minns att hans pappa var med och försökte släcka elden och hur han var klädd i en stor tjock päls som av kyla och vattensläckning står för sig själv, när han tar av sig den.

Arthur skötte sitt jobb på bruket så väl att han blev befordrad till brädgårsförman, men han hade sina rötter bland arbetarna och visade sitt ställningstagande genom att bli medlem i det socialdemokratiska partiet.
Anna och Arthur engagerade sig i den lokala Missionsförsamlingen i Fågelfors där de var medlemmar.

1936 flyttar Arthur och Anna med barnen till den nybyggda skogvaktarbostaden på Klevstigen. Arthur hade nu fått tjänst som skogsförvaltare och fritiden ägnades åt den välskötta trädgården och turer i skogen med bössan.
Anna stöttade Arthur i hans samhällsengagemang och han tog på sig nya politiska uppdrag, han valdes till ordförande i barnavårdsnämnden, pensionsnämnden och vice skogsbrandsfogde.
Ett av hans uppdrag var att leda arbetet med att utforma en kommunal sjukkassa.

Det berättas att Arthur hade förkärlek till bröd, och när han kom hem från någon finare bjudning var det inte maträtterna han beskrev utan hur många sorters bröd det hade funnits. Sitt sociala engagemang visade Anna och Arthur genom att baka bröd och ge till de barnfamiljer som hade det sämre ställt. De gjorde även smörgåsar som delades ut till traktens luffare som samlades på den så kallade luffarkammaren på bruket.
När Fågelfors bildade storkommun mad Högsby blev Arthur skolstyrelsens ordförande och ledamot i barnavårsnämnd och kommunalnämnden.
När Anna och Arthur gick i pension köper de Arthurs föräldrahem på Bruksgatan 30 och renoverar och bygger ut huset i vilket de sedan flyttar in.

