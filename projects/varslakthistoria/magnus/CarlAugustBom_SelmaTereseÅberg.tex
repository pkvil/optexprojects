% Carl August Bom och Selma Terese Åberg
%

\couple{Carl August Bom}{Selma Terese Åberg}
\label[CAB_STÅ]\wlabel{}

I Fredrik Lobergs bok “Samhällsbyggarna” kan man läsa vad han har skrivit om Selma och Carl August Bom:
“Selma Åberg var 16 år, 21 juni 1890, när hon gifte sig med den tolv år äldre Carl August Bom. Eftersom Selma var minderårig hade familjen skrivit till Kungens befallningshavare med en önskan att hon fick gifta sig med Carl August. Kung Oscar den II:s underlydande gav klartecken med vändande post.

Hon flyttade med sin make till Sjötorp nr.1. I det lilla torpet bodde även Carl Augusts syskon och hans mamma Kristina Jonsdotter, hon kallades för “Moster Bomma” och var änka efter Gustav Bom.
Två månader efter giftermålet föddes sonen Artur 12 augusti 1890.

Den 12 november 1892 flyttar den lilla familjen till Hyltan i Fågelfors.  Snart kom flera syskon Gunnar, Anna, Elin och Henry. 1901 flyttar familjen till Hemgården på Bruksgatan även det i Fågelfors.  I Juni 1903 när Henry var knappt ett år dör Carl August.Han har varit och flottat timmer på bruksdammen och trots feber och hosta fortsatte han att arbeta. Efter arbetsdagen kördes Carl August skakande och kraftlös med häst och vagn till sitt hem på Bruksgatan men han tillfrisknade aldrig. Dödsorsaken den 11 juni 1903 fastställdes till lunginflammation. Selma står nu ensam med 5 barn, endast 29 år. I början av 1890-talet hade familjen flyttat till ett hus i Fågelfors som ägdes av bruket och nu när Carl August var död meddelade brukspatron Ekströmer att de var tvungna att flytta före ett visst datum om de inte kunde köpa huset. Selma vädjar till brukspatron att han skulle ge henne en chans att hon och barnen skulle försöka tjäna ihop till de summa som krävdes för att kunna bo kvar. Han lovade om hon fick ihop pengarna så skulle hon få köpa Hemgården nr.5 på Bruksgatan 30 i Fågelfors.
 
Arthur och hans bror Gunnar fick redan vid 13 och 12 års ålder börja arbeta vid Fogelfors Bruk, här kan vi se utdrag ur löneboken för några månader under 1903. Vi kan se att Arthur tjänade 65 öre om dagen och Gunnar 10 öre för varje kväll efter skolan.
13-årige Artur och hans ett år yngre bror Gunnar sökte arbete på snickerifabriken. Artur tjänade 7,97 kronor varje månad, som han lämnade till sin mor. Gunnar arbetade kvällspass och tjänade 10 öre i timmen. Före och efter arbetet tog bröderna på sig de finaste tröjorna som Selma stickat. Vid fabriksporten utanför bruksentren sålde pojkarna kringlor som Selma bakat medan hon passade de yngsta barnen i hemmet. Detta gav ytterligare ett litet tillskott till hushållskassan. Ryktet om de välsmakande kringlorna spred sig snabbt i brukssamhället. Selma började också sticka vantar, sockor och tröjor på beställning. För detta tjänade hon några kronor varje månad.

Selma skrev också till sina släktingar, syskon till både henne och Carl August, som emigrerat till Amerika om hjälp. De hade inte glömt vardagsslitet i sin hembygd och med jämna mellanrum skickade de hem bidrag till familjen.
När 6 år har passerat efter Carl Augusts död hade Selma och hennes barn lyckats spara ihop 72 kronor.
 Selma kunde den 14 mars 1908 stolt stega in till Brukspatronen och meddela att hon hade pengar till att köpa loss huset om nu Brukspatron stod vid sitt ord, vilket han gjorde.

Så här såg köpekontraktet ut.

Två år senare den 27 maj 1910, anslöt sig Selmas 19-årige son Gunnar till den långa strömmen av emigranter västerut. Han färdades mot Moline i Illinois, Rock Island County. Där hade Selma syster Alma och flera arbetarfamiljer bosatt sig. Den dramatiska resan tog 3 ½ vecka, den 20 juni kunde andas ut i sin nya tillvaro i Nordamerika. Han skrev ett långt brev hem till mamma Selma och syskonen. Originalbrevet är lite svårt att tyda, så här kommer en avskrift:

\startblockquote
\leavevmode\hfill Moline den {\setff{+frac}\currvar 20/6} 1910.\par
\vskip\baselineskip

\noindent Älskade moder och syskon jag får nu sätta mig ner och fatta pennan för första gången i det nya landet. jag får då först tala om att vi allesammans här äger en god hälsa hvilket jag även hoppas ni äger därhemma hvilket är det bästa av allt på denna jorden. jag får väl tala om lite huru jag hade det på resan den har gått riktigt bra, när jag kom till Nässjö och hade bytt om tåg där då fick jag redan sällskap med två syskon en Flicka och en pojk ifrån Sandsjö socken Jönköpings län. Vi kom till Göteborg klockan 7 på Torsdags morgon när vi då kom ut ur stationshuset då stod där 15-20 st. hotellmästare och alla så ville de ha oss med sig bara för att de skulle kunna ta många pängar men det gick bra ändå. Vi fick det så bra på ett hotäll så vi kunde inte bätter få det.vi fick ge 150 för vi låg där om natten. sedan klockan 9 på fredagskvällen fick vi gå ombord och 10 gick båten från Göteborg. Ni skulle bara ha varit där och sett så mycke folk där var på kajen där var då många tusen personer och allihop så sjöng de den ena biten efter den andra ända tills båten började på att lägga ut ifrån land. då stod de där och vinkade en sista hälsning hvilket även vi göra som var på båten ett sista farväl till fädernelandet. Somliga grät och somliga skrattade. vi hade fint väder på nordsjön hela tiden så där var riktigt roligt. vi kom till Hull på söndagen middag sedan gick vi från Hull kl.10 på måndagen och var i Liverpool kl.2. sedan fick vi gå på den stora båten kl.10 på tisdagen och var i atlanten i 9 dygn. vi hade nog kommit över fortare men båten blev så hårdt lastad när vi kom ett litet stycke. det var så att de körde på grund med en annan amerikaångare Kristiania så alla dom passagerarna fick komma på Saxonia så vi var omkring 3-4 tusen man på båten. men det var icke nog med det sedan på natten mellan Tisdag och onsdag var det så hård dimma så dom tordes inte köra mer än mäd half fart utan di fick bara böle och blåse hela natten. di visst inte 100 meter från Irland utan dom fick stanna där och mäta djupet innan de vågade sig längre men sedan framåt middagen började dimman skingra sig. men sedan på torsdagen var det sådan sjögång så båten nära stod på ände så då var vi sjuka hvarenda en och spydde gjorde vi också så vi kunde inte vara oppe på däck för några kastade opp sig ända uppå däck. så där kom vågor som var så stora som vår stuva därhemma i Sweden. så då var det inte roligt längre. sedan har det det blåst hela tiden på sjön och varit hårdt kallt ni må tro att det var nog inte så som det stod uppmåladt i boken vi fick. Inte häller så bra mat som det stod utan det var nätt och jämnt vi kunde lifnära oss. vi  fick gå igenom 7 doktorer och vi fick även vaccineras men det var några som slapp undan så jag böhövde ej för det syntes på mig så bra. vi kom till Boston natten mellan onsdagen och torsdagen så när vi vaknade på morgonen då var vi inne i Bostons hamn. Sedan gick vi från Boston klockan 3 på eftermiddagen och var i Chikago kl halv 8 på lördagsmorgonen men sen gick jag från Chikago kl halv 2 och var i Moline kl. halv 10 på lördagskvällen. när  jag då steg av tåget så var det ju litet folk men det stod en karl han tittade på mig och hvart jag gick så gick han efter. då tänkte ja att ta opp adresslappen och visa den men innan jag han det så kom han till mig och frågade om jag kunde språka engelska, nå sa jag kan ni svenska jes va heter ni frågade han, då blev jag jämt lite het då sade jag hvad jag hette då räckte han mig handen och jag frågade om han var Morbror Oskar jes sa han då och så skakede vi hvarandra en stund och sen så tog han en kappsäck och sen gick vi hemåt. vi åkte på spårvagn en bit också för det var en bra bit hem sade han. sedan hälsede på Alfhild hon var på ett ställe och arbetade och sen kom Moster Ida och Ruben-Torsten-Paul och mötte oss, ni må tro att det inte var så ledsamt då sen gick vi hem och jag fick mat och sedan var det ju till att börja på och språka på och språka om det gamla landet. Axel Hofstedt de var och skulle mig vid en annan station litet längre bort men dom hittade mig inte utan vid halv 12 då kom Axel till baka och sade att di inte hade sitt och då när han tittade in i rummet då satt jag och pojka där och då blev det glädje igen när jag då fick hälse på honom. Sedan satt vi oppe till 2 på natten och språkade ni kan tänka att vi hade mycke att språka om. sedan på söndagsmorn då gick Axel bort till Hofstedts och talade om för dom att jag var kommen och Morbror Oskar och jag gick bort till Classons och hälsade på dom så där fick jag bada. di har ett har ett rum som dom badar i sedan kom Axel och Hofstedt dit också och sedan följdes vi åt till en skoaffär och köpte mig ett par skor sedan på eftermiddan gick Axel och bort och hälsade på Alma och Hofstedt di satt ute i det gröna när vi kom. som vi kom så kände jag igen Moster Alma och kände mig också fast hon har växt och blivit stor nu sedan var vi inne och fick saft där och vi hade det så roligt sen på kvällen kom di bort till oss och vi satt ute här och pratade till kl 11. Elin och Signe har jag inte sett ännu för de är uti staden och arbetar. jag har också hälsat på Otto Bäck och Olga och jag kände igen dem båda två så både Olga och Alma var så frågevisa på allting så jag har fått tala om hur det är i Sveden. här är grönt nu och sommar i Amerika jag tänker det är det också hemma nu ja nu får jag sluta mitt slarv för denna gång undrar hur det gick med Annas fot hoppas ni inte gråter nu det gör inte jag. Alma och Olga hälsar till er Alina skall skriva, medens di hälsar till er allesammans härifrån di har att så fint och bra allihop Axel och jag ligger ihop jag har det riktigt bra här jag ska skriva mer nästa gång. hälsa Mormor och Kusin Rudolf från mig samt allesammans därhemma även farmor vänligen G.G. gudd baaj skriv snart tack.
\stopblockquote

Efter att Gunnar emigrerat låg ansvaret för familjens försörjning i ännu högre grad på storebror Arturs axlar. Redan i unga år bestämde sig Artur för att leva ett nyktert liv och vid 19 års ålder, startade han tillsammans med kyrkovaktmästaren Linus Johansson Nationalgodtemplarordens, NTO:s Logen 1343 Liljan i Fågelfors 1907.
1912 när Artur var 22 år valdes han till NTO:s distriktsekreterare. 

1909 träffar Artur sin blivande fru Anna, de förälskar sig i varandra men giftermål får vänta några år. Båda två måste ta hand om sina föräldrar. Arturs syskon är minderåriga och Annas syskon har alla emigrerat.
1919 kan de äntligen gifta sig.

Arthurs syster Anna fick jobb som hembiträde hos Konsul Jansson i Kalmar. Hon förblev ogift och man kan anta att hon betydde mycket för honom eftersom hon fanns med i hans testamente och fick för den tiden en ganska stor summa pengar.
Plats för testamente

Arthurs andra syster Elin flyttade till Stockholm och gifte sig. De kom oftast varje sommar och hälsade på släkten i Fågelfors.

Den sista i barnaskaran Henry blev Fågelfors trogen, arbetade på bruket, gifte sig med Agda Karlsson (syster till Sigurd) och byggde hus på samma tomt som Selmas hus stod på. 



[Bild på Arthur]

[Bild på Carl August]

[Bild på Selma]

