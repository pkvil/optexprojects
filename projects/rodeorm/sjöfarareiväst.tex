%
% P R O L O G
%

\kapitel{PROLOG}{Hur de rakade hade det i Skåne i kung Harald Blåtands tid.}

\initial Många oroliga män foro bort från Skåne med Bue och Vagn och hade ingen lycka i Hjörungavåg; andra följde Styrbjörn till Uppsala och föllo med honom. När det spordes hemma att få voro att vänta tillbaka, blevo sorgekväden framsagda och minnesstenar resta, varpå allt förståndigt folk var överens att det var bäst som skett, i det man nu kunde hoppas på mera stillhet än förr och färre ägoskiften med eggjärn. Det blev nu ymniga år, både med råg och sill, och de flesta trivdes gott; men de som tyckte att grödorna kommo långsamt foro till England och Irland och hade god härnad, och många stannade därute.

\initial Nu hade rakade män börjat komma till Skåne, både från saxarnas land och från England, för att predika den kristna läran. De hade mycket att tala om, och folk var först nyfiket och lyssnade gärna; och kvinnor funno det nöjsamt att doppas av främlingarna och få en vit särk till skänks. Men snart hade främlingarna ont om särkar; och folk upphörde att lyssna till deras predikningar, som föreföllo tröttsamma och föga trovärdiga; dessutom talade de ett hackigt tungomål, som de lärt sig i Hedeby eller på de västra öarna, och tedde sig därför barnsliga till förståndet.

\initial Det gick därför smått med kristnandet; och de rakade, som talade mycket om frid och som över allt annat voro upptända av våldsamhet mot gudarna, grepos stundom av religiösa män och hängdes upp i heliga askträd och fingo pilar i sig och gåvos åt Odins fåglar. Men andra, som nått norrut till göingarnas skogar, där föga religion fanns, hälsades med glädje och leddes bundna till marknader i Småland och byttes bort mot oxar och bäverskinn. Som trälar hos smålänningarna läto några håret växa och kände sig missnöjda med Jehovah och gjorde gott skäl för sig; men de flesta fortsatte att vilja störta gudarna och doppa kvinnor och barn, hellre än att bryta sten och mala korn, och vållade sina husbönder så stor förtret, att göingarna snart inte kunde få ett par treårs smålandsoxar för en fullgod präst utan en mellangift av salt eller vadmal. Då rådde dåligt humör mot de rakade i gränstrakterna.

\initial En sommar hade bud gått runt hela Danavälde att kung Harald Blåtand anammat den nya läran. I sina unga dagar hade han gjort ett försök och hastigt ångrat sig; men nu var han på allvar gången över. Ty kung Harald var nu gammal och hade länge plågats av svår värk i sin rygg, så att han haft ringa glädje av sitt öl och sina kvinnor; och kloka biskopar, som kejsaren sänt, hade nu gnidit honom med björnister, som stärkts med apostlanamn, och svept honom i fårskinn och givit honom signat örtvatten i stället för öl och tecknat kors mellan hans axlar och läst många djävlar ur honom, tills värken gått bort och kungen blivit kristen.

\initial Gudsmännen hade därvid lovat att värre elände skulle drabba honom, om han åter hemfölle åt blot eller visade sig ljum i tron. Därför befallde kung Harald, sedan han blivit rörlig igen och kunnat taga till sig en ung morisk slavinna, som Olof med Ädelstenarna, konung av Cork, sänt honom som vängåva, att allt folket skulle låta sig kristnas; och ehuru sådant tal tycktes sällsamt från den som själv härstammade från Odin, lydde många hans påbud, ty han hade styrt länge och med lycka och hade därför mycket att säga i landet. Han lade de hårdaste straff på dem som buro hand på präster; och i Skåne tilltogo nu dessa i antal, och kyrkor byggdes på slätten; och de gamla gudarna började komma ur bruk utom i sjönöd och vid kreaturssjukdom.

\initial Men i Göinge skrattades mycket åt allt detta. Ty folket i gränsskogarna hade lättare för att skratta än det förståndiga folket på leran, och åt kungars befallningar skrattade de mest. I de trakterna nådde få mäns makt längre än deras högra arm, och från Jellinge till Göinge var lång väg även för de största kungar. I gamla dagar, på Harald Hildetands och Ivar Vidfamnes tid och dessförinnan, hade kungar brukat komma till Göinge för att jaga vildoxar i de stora skogarna, men sällan i andra ärenden. Sedan dess hade vildoxarna tagit slut, och kungars besök med dem; och om nu någon kung retade upp sig över ohörsamhet eller mager skatt och hotade att komma dit, brukade han få det svaret, att inga vildoxar synts till i trakterna, men att man skulle underrätta om så skedde, och taga vänligt emot honom då. Därför var det sedan länge ett stäv hos gränsborna, att bland dem skulle ingen kung komma förrän vildoxarna kommo tillbaka.

\initial Så förblev allt som det varit i Göinge, och ingen kristendom kom i gång där. De präster som försökte sig dit såldes fortfarande över gränsen; men somliga göingar tyckte att man rätteligen borde slå ihjäl dem på fläcken och börja krig mot det näriga folket i Sunnerbo och Allbo, emedan smålänningarnas pris inte gav skälig förtjänst på handeln.

%
% F Ö R S T A   K A P I T L E T
%

\kapitel{FÖRSTA KAPITLET}{Om bonden Toste och hans hushåll.}

\initial Vid kusten bodde folket i byar, för näringens skull och till ökad trygghet; ty strandhugg försöktes ofta från skepp som rundade Skåne, både på våren, av män på utfärd som önskade förse sig med billig färskmat, och på hösten, av dem som tomhänta voro på hemfärd från felslagen härnad. Horn blåstes i natten, när landstigna flockar förnummos, och kallade grannar till hjälp; och hemmafolk i en god by kunde stundom själv taga ett skepp eller två, från främlingar som voro oförsiktiga, och ha vackert byte att visa byns utfarare när långskeppen kommo hem till vintervila.

\initial Men rika och stolta män, som hade eget skepp, funno det svårt att ha grannar nära och bodde helst var för sig; ty även när de lågo på sjön, höllo de sina gårdar försvarade med goda män som sutto hemma. I Kullabygden funnos många sådana stormän; där hade de rika bönderna rykte om sig att vara högmodigare än på andra håll. När de voro hemma, kivades de gärna med varandra, fast det fanns gott utrymme mellan gårdarna; men de voro ofta borta, ty från barndomen sågo de ut över havet och höllo det för sin egen utmark, där alla som mötte dem fingo skylla sig själva.

\initial Där bodde en bonde vid namn Toste, en aktad man och stor sjöfarare; fast han var till åren, förde han alltjämt sitt skepp och for utomlands alla somrar. Han hade släktingar i Limerick på Irland, bland de vikingar som blivit bofasta där, och han brukade fara dit för att driva handel och för att hjälpa hövdingen, som var av Lodbroks blod, att taga skatt av irerna och av deras kloster och kyrkor. Nu hade de goda tiderna för vikingar börjat taga av på Irland, alltsedan Muirkjartach med Läderkapporna, konung av Connacht, gått sin rundgång kring ön med sköldsidan mot havet; ty de infödda värjde sig nu bättre än förr och följde villigare sina kungar, så att det kostade mycken möda att taga skatt av dem; och till och med kloster och kyrkor, som förr varit lätta att plundra, hade nu byggt höga stentorn, där prästerna stucko sig undan med sina skatter och varken kunde nås med eld eller vapen. Därför tyckte nu många bland Tostes män att det skulle varit bättre att lägga färderna till England eller Frankrike, där tiderna voro goda och mera kunde vinnas med mindre besvär; men Toste trivdes bäst som han var van och tyckte sig för gammal att pröva länder där han inte var hemmastadd.

\initial Hans hustru hette Åsa och var bördig från skogsbygden. Hon hade en talande tunga och var något strid till humöret, och Toste sade ibland att han inte kunde märka stort tecken till att hon mildrades med åren, såsom brukade ske med män; men hon var en duktig husfru och skötte gården bra när Toste var borta. Hon hade fött honom fem söner och tre döttrar, men med sönerna hade lyckan inte varit den bästa. Den äldste hade som yngling omkommit på ett bröllop, när han var munter av öl och ville visa att han kunde rida på en tjur; den näste hade spolats över bord i en storm på sin första resa. Men värst hade olyckan varit med den fjärde, som hette Are; ty en sommar, när han var nitton år, hade han gjort tvenne grannars hustrur med barn, medan deras män voro utomlands, och därmed vållat svår oreda och mycket gyckel, jämte stora utgifter för Toste när männen kommo hem. Han hade blivit missmodig och människoskygg av detta och dräpt en man, som skämtat för mycket om hans raskhet, och därpå rymt ur landet. Det spordes att han sällat sig till svenska handelsmän och farit österut med dem, för att slippa möta folk som kände till hans ledsamheter; men därefter hade ingenting hörts om honom. Åsa hade drömt om en svart häst med blod över bogarna, och därmed visste hon att han var död.

\initial Två söner var därefter vad Åsa och Toste hade kvar. Den äldre hette Odd; han var en kortvuxen man, grovt byggd och hjulbent, stark och hårdhänt och betänksam i sitt tal; han var tidigt med på Tostes färder och hade god hand med skepp och vapen. Hemma blev han fort tvär, ty det var svårt för honom att få vintrarna att gå; och det skar sig då ofta mellan honom och Åsa. Han brukade säga att det nu en gång var så med honom, att härsken saltmat på skeppet smakade honom bättre än julstekar i land; men Åsa sade sig aldrig kunna märka att han var den som tog minst av det bästa hon satte fram. Han sov så mycket varje dag att han ofta klagade över dålig nattsömn; även när han tog en piga med sig i sänghalmen, sade han, blev bättringen inte stor. Åsa tyckte illa om att han låg med hennes pigor: de kunde lätt bli inbilska av sådant och sturska mot sin matmor, och Odd borde hellre gifta sig. Men Odd sade att därmed brådskade det inte: de kvinnor han trivts bäst med hade han funnit på Irland, och av den sorten kunde han inte gärna taga någon med sig hem; ty då, kunde han förmoda, skulle snart både näbbar och klor vara i gång mellan Åsa och henne. Åsa blev då retad och undrade om han satt där och önskade att hon måtte dö. Odd kunde då svara att därmed finge hon göra som hon själv tyckte bäst: han ville inte ge henne några råd i den saken men skulle stå ut med vad som kunde bli av.

\initial Fast han var långsam i sitt tal, hade Åsa inte alltid lätt att få sista ordet mot honom; och hon brukade säga att det i sanning var hårt för henne att ha mistat tre goda söner och fått den lämnad kvar som hon lättast kunnat undvara.

\initial Med Toste kom Odd bättre överens; och så snart det blev vår och började lukta tjära kring båthus och skeppsbryggor, brukade hans lynne bli lättare. Ibland försökte han då till och med sätta samman visor, fast det gick smått för honom med sådant: om hur alkfågelns vång nu var redo att plöjas; eller om hur havets hästar nu snart skulle bära honom till sommarlandet.

\initial Men stort namn som skald vann han aldrig, och minst hos traktens giftasvuxna bonddöttrar. Han sågs sällan vända sig om när han seglade bort.

\initial Hans broder var yngst bland alla Tostes barn och sin moders ögonsten; han hette Orm. Han växte fort och blev lång och skranglig, och Åsa jämrade sig mycket över hans magra hull; så snart han inte åt en god del mera än någon av de vuxna, trodde hon alltid att hon skulle mista honom och sade att hans dåliga matlust skulle bli hans fördärv. Orm tyckte om mat och klagade sällan över sin moders omsorger om hans föda, men Toste och Odd grumsade stundom rörande godbitar som han skulle ha. Han hade som barn varit sjuk ett par gånger; och därefter kunde Åsa aldrig tro att hans hälsa var god, utan var ständigt omkring honom med ängslan och förmaningar och fick honom ibland att känna sig full av farliga krämpor och i stort behov av helig lök, läkedomsgaldrar och värmda lerfat, när hans svåraste ohälsa var att han förätit sig på korngröt och fläsk.

\initial När han började bli vuxen, blevo Åsas bekymmer flera. Det var hennes hopp att han skulle bli en märklig man och en hövding; och hon framhöll ofta belåtet för Toste att han artade sig till att bli stor och stark och var så klok i sitt tal att han i allo syntes brås på sitt möderne; men hon var full av rädsla för all farlighet som skulle möta honom på mäns vägar. Hon talade ofta med honom om hans bröders olyckor och hade honom att lova att akta sig för tjurar, vara försiktig på skepp och aldrig ligga med andra mäns kvinnor; men utom detta fanns det så mycket annat han kunde råka ut för, att hon inte visste sig någon råd. När han var sexton år och skulle fara ut med de andra, förbjöd Åsa detta, emedan han ännu vore alltför ung och alltför ömtålig till hälsan; och när Toste då undrade om hon allt framgent tänkte fostra honom till kökshövding och käringhjälte, brast hon ut i sådant raseri att Toste blev rädd och lät henne hållas och var glad att själv komma åstad så fort han kunde.

\initial Den hösten kommo Toste och Odd sent tillbaka och hade förlorat så mycket manskap att de knappt hade folk till årorna; men de voro likväl belåtna och hade mycket att berätta. I Limerick hade förtjänsten varit liten, ty de iriska kungarna i Munster hade nu blivit så mäktiga att vikingarna där hade mest att göra med att försvara sig själva; men vänner till Toste, som legat där med sina skepp, hade frågat honom om han ville vara med på ett försök mot en stor midsommarmarknad, som hölls i Merioneth i Wales, på en plats där vikingar aldrig varit men dit man nu kunde komma med hjälp av ett par säkra vägvisare, som Tostes vänner hade på hand. Odd hade övertalat Toste att bli med, och deras skeppsfolk hade också varit hågat; och med sju skepp hade de alla landat i Merioneth och tågat en besvärlig väg inåt land och oförmärkt kommit över marknaden. Det hade blivit skarp strid och en del manspillan, och vikingarna hade segrat och vunnit stort byte både i gods och fångar. Därpå hade de seglat över till Cork och sålt sina fångar, ty i Cork samlades av ålder slavhandlare från världens alla hörn för att välja bland fångster som vikingar bragte dit; och kungen där, Olof med Ädelstenarna, som var kristen och mycket gammal och vis, brukade själv köpa dem han fann lämpliga, för att sedan låta deras anhöriga lösa ut dem, med god förtjänst för honom. Från Cork hade de seglat hem i starkt sällskap, för att inte råka ut för sjöröveri när de hade ringa lust att slåss, med svag besättning och stor rikedom ombord, och hade så kommit oskadda runt Skagen, där vikbor och vestfoldingar brukade ligga på lur efter lönande skepp på hemfärd.

\initial Sedan allt skeppsfolket fått sin del av bytet, blev mycket kvar åt Toste; och när han vägt upp sitt silver inne i sin kammare, sade han att en resa som denna kunde vara ett gott slut på hans färder och att han framdeles tänkte stanna hemma, så mycket hellre som han började bli styv i kroppen och Odd nu kunde sköta allt lika bra som han själv och dessutom skulle ha Orm till hjälp. Odd sade att han tyckte detta var klokt talat. Men Åsa sade genast att det alls inte var klokt; ty visserligen var det mycket silver som vunnits, men med så mycket folk som de hade i maten alla vintrar skulle det inte räcka länge; och hur skulle man kunna lita på att Odd inte gjorde av med hela förtjänsten på sina kvinnor på Irland, om han ens brydde sig om att komma hem; och Toste borde förstå att han fick sin styvhet i ryggen av att sitta sysslolös vid elden om vintrarna, inte av att vara till sjöss; och för henne var det alldeles nog att snubbla över hans ben halva året. Hon kunde inte förstå, sade hon, hur det var fatt med män nuförtiden; ty hennes egen morfars bror, Sven Råttnos, en väldig hjälte bland göingar, hade fallit som en man i fejd med smålänningar tre år efter det han druckit alla under bordet på sin äldste sonsons bröllop; och nu fick man höra prat om krämpor från män som voro i sina bästa år och som inte skämdes för att vilja dö på halmen som kor. Men nu skulle Toste och Odd och alla de hemkomna få gott öl till välkomst, av en brygd som skulle smaka dem; och Toste skulle slå sina griller ur huvudet och dricka för en lika god resa nästa år; och sedan skulle de ha en god vinter samman, bara ingen förargade henne med sådant prat.

\initial När hon gått för att ordna med ölet, sade Odd att Sven Råttnos kanske valt smålänningarna som den bättre delen, ifall alla kvinnor i den släkten väsnades lika mycket som hon; och Toste sade att han inte helt ville säga emot i den saken, men att hon vore en duktig kvinna på många sätt och att han inte ville reta henne mer än som var tvunget, och det borde inte heller Odd göra.

\initial Den vintern lade alla märke till att Åsa stundom gick blek och betryckt vid sina sysslor och att hennes tunga löpte mindre rappt än vanligt; hon var mera mån om Orm än någonsin och blev ibland stående och såg på honom, som om hon sett en syn. Orm hade nu vuxit sig stor och kunde i styrka tävla med alla jämnåriga och många äldre. Han var rödhårig och mjäll i hyn, bred mellan ögonen, stubbnäst och stormynt, långärmad och något kutryggig; han var rörlig och snabb, och med båge och i spjutkast var han säkrare än de flesta. Han kom lätt i raseri och kunde då rusa blint på den som retat honom; och till och med Odd, som tidigare haft sin lust i att få honom blek och rasande, hade nu blivit försiktig med honom, sedan hans styrka börjat göra honom farlig. Men eljest var han lugn och foglig och alltjämt van att i allt rätta sig efter Åsa, fast han stundom munhöggs med henne när hennes omsorger tycktes honom besvärliga.

\initial Toste gav honom nu mäns vapen, svärd och bredyxa och en god hjälm, och Orm gjorde själv en sköld; men med ringskjorta var det värre, ty ingen i huset passade honom, och det var numera ont om goda brynjesmeder i landet, emedan de flesta farit utomlands, till England eller jarlen i Rouen, där de fingo bättre betalning. Toste tyckte att Orm kunde nöja sig med lädertröja så länge, tills han kunde skaffa sig en god skjorta på Irland: där funnos alltid döda mäns vapenkläder billigt att få i alla hamnar.

\initial När de en dag sutto vid maten och talade om detta, lade Åsa ansiktet mot armarna och började gråta. Alla tystnade och sågo på henne, ty det var inte ofta hon fällde tårar; Odd frågade om hon hade tandvärk. Åsa torkade sig i ansiktet och vände sig till Toste; hon sade att talet om döda mäns kläder syntes henne vara ett ont varsel, och hon var redan säker på att Orm skulle omkomma så snart han följde med till sjöss; ty tre gånger hade hon nu i sina drömmar sett honom ligga blödande vid en skeppsbänk, och alla visste att hennes drömmar voro att lita på. Därför ville hon nu bedja Toste att vara god mot henne och inte göra av med Orms liv i onödan, utan låta honom bli hemma denna sommar; ty hon trodde att faran hotade honom nu snart; och kunde han överleva den, skulle han kanske vara mindre utsatt sedan.

\initial Orm frågade om hon kunnat se i drömmen var han var sårad. Åsa sade att hon varje gång genast vaknat i förskräckelse vid synen; men hon hade sett hans hår blodigt och hans ansikte mycket blekt; och denna dröm hade tyngt henne mycket, och mera för varje gång den återkom, fast hon inte velat säga något förrän nu.

\initial Toste satt fundersam och sade därpå att han inte visste mycket om drömmar och aldrig ängslat sig för sådant.

– Ty de gamla brukade säga, att som Spinnerskorna spinna, så måste det bli. Men när nu du, Åsa, drömt samma dröm tre gånger, då är det kanske likväl en varning; och vi ha redan mistat söner nog. Därför skall jag inte säga dig emot i detta; och Orm kan bli hemma i sommar, om han själv så tycker. För mig känns det nu som om jag gott kunde fara ut en gång till; och på så sätt blir det kanske bäst för alla.

\initial Odd höll med Toste, ty han hade flera gånger märkt att Åsas drömmar slogo in. Orm var inte nöjd med vad som bestämts, men han var van att rätta sig efter Åsa i viktiga ting; och därefter talades inte mer om detta.

\initial När våren kommit och tillräckligt många män från inlandsbygderna gjort upp med Toste om plats bland hans skeppsfolk, seglade han och Odd åstad som vanligt och Orm blev kvar hemma. Han var misslynt mot Åsa, och ibland låtsade han sig vara sjuk för att skrämma henne; men när hon då fick brått med omvårdnad och läkedom, började han själv tro på vad han låtsat och hade föga glädje av sin lek. Åsa kunde inte glömma sin dröm, och trots alla bekymmer han vållade var hon glad att ha honom hemma.

\initial Men den sommaren kom han likväl på sin första resa, utan att Åsa blev frågad om lov.

%
% ANDRA KAPITLET
%

\kapitel{ANDRA KAPITLET}{Om Kroks utfärd och hur Orm kom på sin första resa.}

\initial I det fyrtionde året av kung Harald Blåtands herradöme, sex somrar före Jomsvikingarnas härfärd till Norge, stucko tre skepp ut från Listerlandet, med nya segel och välbemannade, och höllo söderut för att plundra bland venderna. De fördes av en hövding som hette Krok. Han var en mörklagd man, lång och skranglig och mycket stark; han hade stort anseende i bygden, ty han hade lätt att tänka ut djärva planer och brukade gyckla med folk som det gått illa för och förklara hur han själv skulle rett sig bättre i deras ställe. Själv hade han aldrig gjort mycket, utan trivts bäst med att berätta om vad han snart tänkte göra; men nu hade han så länge eggat bygdens ungdom med tal om det byte som raska män kunde vinna på en kort färd till venderna, att manskap kommit samman och skepp utrustats och han själv valts till hövding för färden. Det fanns mycket att vinna bland venderna, hade han sagt; främst kunde man vara förvissad om god fångst av silver, bärnsten och slavar.

\initial Krok och hans män kommo till vendernas kust och nådde en flodmynning och rodde uppför den mot strid ström, tills de kommo till en träborg med pålverk tvärs över floden. Här stego de i land i en tidig gryning och gingo fram mot venderna runt deras träverk. Men venderna voro manstarka och sköto flitigt med pilar, och Kroks män voro trötta av sträng rodd; det blev en hård strid innan venderna flydde. Krok hade då förlorat goda män; och när bytet sammanräknades, visade det sig bestå av ett par järngrytor och några fårskinnspälsar. De rodde tillbaka nedför floden och försökte på ett annat ställe längre västerut; men också denna by var väl försvarad; och efter skarp strid, där de åter hade förlust, vunno Kroks män ett par rökta fläsksidor, en söndrig pansarskjorta och ett halsband av små slitna silvermynt.

\initial De begrovo sina döda vid stranden och höllo rådslag; och Krok hade besvär med att förklara varför färden inte artat sig så som han sagt. Men han lyckades lugna sina män med kloka ord. Tillfälligheter och otur, sade han, finge man alltid vara beredd på; den sanne vikingen gåve inte tappt för en och annan småsak; venderna började bli mera hårdfjällade än förr, och han ville nu lägga fram ett gott förslag, som skulle lända till allas båtnad. Man borde göra ett försök på Bornholm, ty dess rikedomar voro välkända för alla, och ön var fattig på stridbart folk, emedan många därifrån på sistone farit till England. Ett strandhugg där skulle med ringa möda lämna riklig vinst såväl av guld som av bonader och vackra vapen.

\initial De funno detta vara väl talat och återfingo sin tillförsikt; de satte segel och styrde mot Bornholm och kommo dit tidigt en morgon och rodde upp längs östra kusten i vindstilla och lyftande dimma, för att söka en god landningsplats. De rodde tätt tillsammans och voro vid gott lynne men höllo sig tysta för att försöka komma obemärkt i land. Då hörde de framför sig ljudet av årklykor och jämnt doppade årblad och skönjde genom dimman ett ensamt långskepp som kom fram runt en udde och höll rakt emot dem utan att sakta sin rodd. Alla spände ögonen i skeppet, som var stort och vackert, med rött drakhuvud och tjugo par åror, och gladdes att det var ensamt; och Krok befallde att alla män som inte sutto vid årorna skulle taga sina vapen och stå redo, ty här kunde mycket vinnas. Men det ensamma skeppet kom närmare, som om det ingenting märkt; och en fetlagd man i dess förstam, med ett brett skägg under en bucklad hjälm, satte handen vid munnen, när de kommit varandra nära, och ropade med skrovlig röst:

– Håll undan eller slåss!

Krok skrattade, och hans män med honom; och han ropade tillbaka:

– Har du sett tre skepp väja för ett?

– Jag har sett mer än så, skrek den tjocke stambon otåligt; ty de flesta väja för Styrbjörn. Men välj nu fort hur du vill!

\initial Då sade Krok ingenting mera, utan höll undan och låg stilla på årorna, medan det främmande skeppet rodde förbi; och intet svärd hölls blottat på något av Kroks skepp. De sågo en högväxt ung man i blå kappa och med ljust fjun kring hakan, som rest sig från sin viloplats bredvid rorsmannen på det främmande skeppet, stå med ett spjut i handen och betrakta dem med kisande ögon och därpå gäspa stort; sedan ställde han spjutet ifrån sig och lade sig åter till vila; och Kroks män förstodo att detta var Björn Olofsson, kallad Styrbjörn, Uppsalakonungens fördrivne brorson, som sällan skydde storm och aldrig strid och som få män mötte villigt på havet. Hans skepp fortsatte sin färd och försvann söderut i diset; men hos Krok och hans män hade det goda lynnet svårt att komma tillbaka.

\initial De rodde ut till de östra skären, där inga människor funnos, och gingo i land där och kokade föda och höllo ett långt rådslag; många tyckte nu att man helst borde segla hem igen, sedan oturen följt med även till Bornholm. Ty med Styrbjörn i dessa farvatten var ön säkerligen full av Jomsvikingar, och ingenting fanns då att uträtta för andra. Några sade, att de kanske inte hade mycket att göra på havet utan att äga en hövding av samma sort som Styrbjörn, som inte höll undan i onödan.

\initial Krok var till en början mera fåmäld än vanligt; men han lät taga i land öl åt alla; och sedan de druckit, började han uppmuntra dem. Det var på ett sätt otur att de mött Styrbjörn, det ville han erkänna; men på ett annat sätt var det stor tur att de mött honom så som skett; ty om de hunnit i land och där råkat ut för hans folk eller andra Jomsvikingar, skulle det länt dem till svår skada. Alla Jomsvikingar, och särskilt Styrbjörns egna män, voro halvt bärsärkar, stundom hårda mot järn, och höggo med båda händer fullt ut lika bra som de bästa kämpar från Lister. Att han inte velat anfalla Styrbjörns skepp kunde kanske te sig underligt för den tanklöse; men han ansåg sig ha haft gott skäl för sin avhållsamhet, och det var tur att han besinnat sig i tid. Ty en landlös sjörövare kunde inte gärna ha mycket hopsparat som vore värt hård strid; själva hade de inte gett sig till sjöss för att vinna tom ära utan för att vinna byte; därför hade han ansett det riktigare att tänka på allas bästa än på det egna anseendet, och eftertanke skulle säga dem alla att han däri handlat som en hövding borde.

\initial Sedan Krok på detta sätt börjat skingra sina mäns missmod, kände han sig också själv styrkt av sina ord; och han fortsatte med att avråda dem från att nu fara hem. Ty hemmafolket i Lister, sade han, var ett vasstungat släkte; och särskilt kvinnorna skulle visa sig besvärliga att återse, med många frågor rörande deras bedrifter, deras byte och deras snabba återkomst. Ingen man med anseende kunde vilja utsätta sig för sådant sladder, och med hemfärd borde det därför anstå tills någonting vunnits som vore värt att komma hem med. Vad det nu gällde vore att komma överens om att hålla samman och visa ihärdighet samt att finna något bra mål för vidare färd; och innan han själv sade något mera, ville han gärna höra förståndiga mäns åsikt i saken.

\initial En föreslog att man skulle fara till kurernas och livernas land, där lönande byte skulle finnas; men detta vann intet medhåll, ty män med bättre besked visste att svearna varje sommar i stora flockar plundrade i dessa länder och ogärna sågo främlingar komma i samma ärende. En annan hade hört att det mesta silvret i hela världen skulle finnas på Gotland, och tyckte att man kunde försöka där; men andra, som visste bättre, sade att gutarna numera, sedan de blivit rika, bodde i starka byar, som endast kunde tagas med stor härsmakt.

\initial En tredje tog därpå till orda, en man vid namn Berse, som talade betänksamt och var allmänt skattad för sin klokhet. Han sade att det började bli trångt och snålt i Östersjön, där nu alltför många lågo ute på plundring och där till och med venderna börjat förstå att sätta sig till motvärn. Eftersom man inte gärna kunde fara hem – ty därvidlag tyckte han som Krok – vore det att överväga om man inte borde segla ut i västerled. Själv hade han aldrig varit åt det hållet; men män från Skåne, som han talat med på en marknad förliden sommar, hade varit i England och Bretland med Toke Gormsson och Sigvalde Jarl, och de hade haft mycket att säga till de färdernas beröm. De hade prunkat i guldringar och dyrbara kläder; och enligt deras utsago hade de vikingar, som satt sig fast vid franska flodmynningar för längre tid för att plundra inåt landet, ofta grevedöttrar till gamman i sängen samt borgmästare och abboter till drängar. Huruvida hans skånska sagesman strängt hållit sig till sanningen med dessa påståenden kunde han inte veta, men med hänsyn till skåningars allmänna trovärdighet vore det kanske klokt att draga av vid pass hälften. Säkert var emellertid, att dessa hemkomna män gjort ett intryck av den största välmåga, så att de till och med bjudit honom, en blekingsk främling, på en myckenhet starköl, utan att stjäla hans tillhörigheter sedan han somnat; och därför kunde inte allt vara lögn, vilket man ju också hade god reda på från andra håll. Där skåningar stått sig så bra, torde också blekingar kunna trivas; och därför, slöt Berse, ville han gärna pröva en färd västerut, om de flesta tyckte som han.

\initial Många ropade bifall till detta; men andra sade att matförrådet var för knappt för att räcka tills de kommo till de feta länderna i väster.
Då tog Krok åter till orda och sade att Berse kommit med just det förslag som han själv tänkt framföra. Till vad Berse sagt om grevedöttrar och rika abboter, för vilka stora lösesummor brukade erhållas, ville han tillägga en sak som var allmänt bekant bland beresta män, nämligen att det på Irland funnes ej mindre än etthundrasextio konungar sammanlagt, större och mindre, vilka alla hade skatter och sköna kvinnor och vilkas krigare stredo klädda endast i linnekläder och alltså inte kunde vara svåra att komma till rätta med. Den enda svårigheten vore att komma upp genom Öresund, där man ofta kunde stöta på närgånget folk. Men tre välbemannade skepp, som själve Styrbjörn inte vågat anfalla, skulle ha respekt med sig till och med där; vidare voro de flesta vikingar i de trakterna vid denna årstid redan åstad västerut; dessutom hade man nu månlösa nätter framför sig. Vad som brast i föda kunde med lätthet anskaffas så snart man kommit lyckligen genom sundet.

\initial Alla voro nu åter vid gott lynne och tyckte att planen var god och att Krok var den bäste både till förstånd och kunskaper; och alla voro stolta över att finna sig oförvägna nog till en färd i västerled, ty från deras trakt hade i levande mäns minne intet skepp försökt en sådan resa.

\initial De satte segel och kommo till Möen och lågo där en dag och en natt och höllo god utkik och väntade på gynnsam vind. Därpå höllo de i stormigt väder upp genom sundet och kommo på kvällen igenom dess hals utan att möta fiender; fram på natten nådde de lä under Kullen och beslöto att se sig om efter förning. Tre flockar gingo upp i land, var på sitt håll: Kroks flock hade tur med sig och kom till en fårfålla nära en stor gård och lyckades döda fårherden och hans hund innan de hunnit göra larm. Därpå fångade de fåren och skuro halsen av dem, så många de kunde bära med sig; men det blev då mycket bräkande, och Krok befallde sina män att skynda sig.

\initial De begåvo sig tillbaka ned mot skeppen längs den stig de kommit, envar med ett får över axeln, och skyndade sig så mycket de kunde. De hörde bakom sig rop av folk som vaknat i gården, och snart kom grovt skall av hundar som släppts på deras spår. Därpå ljöd längre borta en kvinnoröst, som skar genom larmet av hundar och män och skrek »Vänta! Stanna!» och ropade »Orm!» flera gånger och därpå skrek »Vänta!» mycket gällt och förtvivlat. Kroks män hade svårt att gå fort med sina bördor, ty stigen var stenig och stupade brant, och natten var molnig och ännu nästan mörk. Krok själv gick sist i raden och bar sitt får över axeln och hade en yxa i den andra handen. Han ville helst slippa att slåss för fårens skull, emedan det var föga lönt att satsa liv och lemmar för så litet; han drev på sina män med hårda ord när de snubblade eller gingo långsamt.

\initial Skeppen lågo invid ett par flata stenhällar och höllos ifrån dem med åror; de voro färdiga att lägga ut vid Kroks återkomst, ty de andra flockarna hade redan återvänt utan att ha funnit något; några män voro i land för att hjälpa Krok om det skulle behövas. Det var få steg kvar till skeppen, när två stora hundar kommo rusande ned längs stigen. Den ene hoppade mot Krok, men han vek åt sidan och träffade den med yxan; den andre for förbi honom i ett stort språng och kom på den man som gick närmast framför och välte honom av farten och högg honom i halsen. Ett par av de väntande skyndade fram och dödade hunden; men när de och Krok böjde sig över den bitne, sågo de att mycket var sönder i hans hals och att han snabbt förblödde.

\initial I detsamma kom ett spjut och strök förbi Krok, och två män kommo nedför branten och ut på stenhällarna; de hade sprungit så häftigt att de lämnat alla av sitt följe efter sig. Den främste, som var barhuvad och utan sköld, med ett kortsvärd i handen, snävade och föll framstupa på stenhällarna; två spjut gingo över honom och träffade hans följeslagare, som föll och blev liggande. Men den barhuvade var genast på fötter igen och tjöt som en varg; han for på en man, som sprungit fram med lyftat svärd när han föll, och högg honom vid tinningen och fällde honom. Därpå sprang han mot Krok, som stod närmast; och allt detta gick mycket fort. Han högg mot Krok snabbt och hårt; men Krok bar ännu sitt får och fick det emellan; och i detsamma slog han själv till med omvänd yxa och träffade sin motståndare över pannan, så att denne föll sanslös. Krok böjde sig över honom och kunde se att han var en yngling, rödhårig och stubbnäst och blek i hyn; han kände efter med fingrarna där yxhammaren träffat och fann skallen hel.

– Jag tar med mig både får och kalv, sade Krok; han kan få ro för den han dräpte.

\initial Därmed lyfte han upp honom och bar honom ombord och lade honom under en roddarbänk; och med allt ombord utom de två män som lämnats döda, lade skeppen ut, i samma stund som en stor flock förföljare nådde ned till stranden. Det hade nu börjat ljusna, och några spjut slungades ut mot skeppen men gjorde ingen skada. Roddarna förde sina åror med kraft, glada att ha färskmat ombord; och de hade redan nått ett gott stycke från land, när de bland skepnaderna på stranden fingo se en kvinna, i lång blå särk och med håret flygande, som rusade ut till hällarnas rand och sträckte armarna mot skeppen och ropade. Hennes rop nådde dem endast som en tunn ton över vattnet, men hon stod kvar länge efter det intet kunde höras.

\initial På så sätt kom Orm, Tostes son, som med tiden blev känd såsom Röde Orm eller Orm den Vittfarne, åstad på sin första resa.

%
% T R E D J E   K A P I T L E T
%

\kapitel{TREDJE KAPITLET}{Hur de seglade söderut och hur en god vägvisare blev funnen.}

\initial Kroks män nådde hungriga upp till Väderön, sedan de fått ro hela vägen dit, och där lade de till och gingo i land för att samla bränsle och göra sig ett gott mål; där funnos endast några gamla fiskare, som på grund av sin fattigdom voro trygga för plundrare. När de styckade fåren, berömde de deras fetma och det goda vårbete som måtte finnas på Kullen; de satte styckena på spjuten och höllo dem i elden och smackade när fettet började fräsa, ty det var länge sedan de känt så god lukt. Många berättade för varandra om när de sist varit med om så läckra bitar, och alla voro överens om att färden i västerled artade sig bra. Därpå började de äta så att det dröp om skäggen.

\initial Orm hade nu återfått sansen, men det var inte mycket med honom; när han kom i land med de andra, hade han svårt för att stå på benen. Han satte sig ned och höll huvudet mellan sina händer och svarade ingenting på tilltal. Men efter en stund, sedan han spytt och druckit vatten, blev han bättre; och när stekoset började kännas, lyfte han på huvudet likt en nyvaknad och såg på männen omkring sig. Den som satt närmast honom grinade vänligt mot honom och skar en bit av sitt köttstycke och räckte honom.

– Tag och ät, sade han. Bättre stek har du aldrig smakat.

– Det vet jag, svarade Orm. Det är jag själv som består den.

Han tog biten och höll den i nypan utan att äta; han såg sig noga om i kretsen, på man efter man, och sade därpå:

– Var är den jag högg? Dog han?

– Han dog, svarade hans granne; men ingen här har hämnd efter honom, och du skall ro i hans ställe. Hans åra är närmast framför min, och därför kan det vara bäst att du och jag blir vänner. Jag heter Toke, och vad heter du?

Orm sade sitt namn och frågade därpå:

– Var det en ansedd man jag fällde?

– Han var något långsam av sig, som du märkte, sade Toke, och med vapen var han inte så bra som jag; men det hade varit mycket begärt, ty jag är en av de bästa här. Men han var en stark och säker man och hade gott anseende; han hette Åle, och hans fader sår tolv tunnor råg, och han hade legat på sjöfärd två gånger redan. Och ror du så bra som han, så ror du inte dåligt.

\initial När Orm hört detta, syntes han bli bättre till mods och började äta. Men efter en stund frågade han:

– Vem var det som fällde mig?

Krok satt ett stycke därifrån och hörde hans fråga. Han skrattade och lyfte sin yxa och tuggade slut och sade:

– Det var hon här som kysste dig; om hon bitits hade du inte frågat.

Orm såg på Krok med stora ögon som det inte kom någon blinkning i; därpå suckade han och sade:

– Jag var utan hjälm och andfådd; eljest kunde det kanske gått på annat sätt.

– Du är morsk, skåning, sade Krok, och tror dig redan vara en kämpe. Men du är för ung ännu och har inte en krigares förstånd. Ty ingen förståndig man rusar ut hjälmlös för några fårs skull; nej inte ens om det vore hans egen hustru som snattats. Men min tro är att du är en man med lycka, och det kan hända att du har lycka med åt oss alla. Din egen lycka ha vi redan sett på tre sätt. Du snavade på stenarna när två spjut flögo mot dig; och efter Åle, som du dräpte, har ingen bland oss hämnd; och av mig blev du inte huggen, emedan jag ville ha en roddare i hans ställe. Därför tror jag att din lycka är stor och kan bli oss nyttig; och jag ger dig nu fred för oss alla, så länge du sköter Åles åra.

\initial Alla männen tyckte att detta var bra talat av Krok. Orm åt och betänkte sig; därpå sade han:

– Jag tar emot din fred; och jag tror inte att jag behöver skämmas för den saken fast ni stulit får från mig. Men som en slav vill jag inte ro, ty jag är av god ätt; och fast jag är ung räknar jag mig som en man med anseende, sedan jag fällt en så god man som Åle. Därför vill jag ha mitt svärd tillbaka.

\initial Det blev nu mycket tal om detta. Några tyckte att Orms begäran var föga rimlig och att han kunde vara nöjd med att ha fått livet till skänks; men andra sade att morskhet inte vore något fel hos en ungdom och att man borde visa hänsyn mot en man med lycka; och Toke skrattade och undrade hur många de män voro som bland tre skeppsbesättningar kunde känna sig ängsliga för en ung man med svärd. En man vid namn Kalv, som talat emot Orms begäran, ville slåss med Toke för dennes yttrande; och Toke förklarade sig vara villig så snart han slutat det goda njurstycke han höll på med. Men Krok förbjöd allt slagsmål för en sådan saks skull; och slutet blev att Orm fick tillbaka sitt svärd och att hans uppförande finge avgöra om han framdeles skulle anses som fånge eller kamrat. Men för svärdet, som var ett gott vapen, skulle Orm ge betalning till Krok, så snart sådan vunnits på färden.

\initial Det blåste nu en lätt bris, och Krok sade att det var tid att sticka åstad med den. Alla gingo ombord, och skeppen höllo upp i Kattegatt med seglen fyllda. Orm såg sig tillbaka över sjön och sade att det var stor tur för Krok att det var ont om hemmaskepp här i trakterna vid denna årstid; ty kände han sin moder rätt, skulle hon eljest vid detta laget legat ute efter dem med halva Kullen ombord.

\initial Därpå tvättade han sitt sår i pannan och sköljde det levrade blodet ur sitt hår; och Krok sade att ärret i pannan skulle bli värt att visa bland kvinnor. Därpå kom Toke med en gammal läderhjälm med järnskenor; han sade att den inte var mycket till hjälm i dessa tider, men att han hittat den bland venderna och inte hade bättre att undvara. Mot en yxa, sade han, dugde den föga men vore dock bättre än ingenting. Orm provade den, och det befanns att den skulle passa när bulnaden gått tillbaka. Orm tackade Toke; och de visste nu båda att de skulle bli vänner.

\initial De kommo runt Skagen med god vind och blotade där enligt gammal sed till Ägir och hela hans släkt, både fårkött och fläsk och dricka, och följdes länge av skrikande måsar, vilket hölls för ett gott tecken. De höllo ned längs den jutska kusten, där landet var öde och revbenen av brutna skepp ofta syntes i sanden; längre åt syd voro de i land på ett par småöar och funno vatten och föda men eljest intet. De fortsatte nedåt längs kusten; och mestadels hade de tur med vinden, så att männen förblevo vid gott lynne vid att slippa mödan med mycken rodd. Toke sade att Orm kanske också hade väderlycka, till allt annat: väderlycka vore bland det bästa en man kunde ha, och Orm kunde då i sanning hoppas på en god framtid. Orm trodde att Toke nog kunde ha rätt; men Krok ville inte hålla med om den saken.

– Väderlyckan är min, sade han; ty vi ha haft den bästa tur med väder och vind alltifrån början, långt innan Orm kom med; och om jag inte litat på min väderlycka, skulle jag aldrig vågat mig ut på denna färd. Men Orms lycka är god, även om den inte är som min; och ju flera lyckomän vi ha ombord, desto bättre är det för oss alla.

\initial Den kloke Berse höll med om detta och sade att män utan lycka hade det svårast av alla:

– Ty mot män vinna män, och mot vapen vapen; åt gudarna finns det blot, och mot trolldom sejd; men mot dålig lycka finns ingenting att ställa.

\initial Toke sade att han inte visste om han för sin del hade stor lycka, utom på så sätt att hans fiskelycka alltid varit god. Mot män som han haft något otalt med hade han alltid rett sig bra, men det kunde mera bero på styrka och skicklighet än på lycka.

– Men nu, sade han, är jag nyfiken på om jag under denna resa skall ha god guldlycka och kvinnolycka; ty jag har hört mycket om allt vackert som skall finnas här västerut, och det börjar bli länge sedan jag höll i en guldring och en kvinna. Och även om det mesta blir silver i stället för guld, och jag inte hittar någon grevedotter, såsom Berse tänker göra, utan endast vanliga frankiska ungmör, skall jag inte klaga; ty jag är ingen högmodig man.

\initial Krok sade att Toke finge ge sig till tåls ännu någon tid, hur glupsk han än kunde känna sig efter det ena och det andra, och Toke höll detta för att vara det troliga; ty det såg inte ut, tyckte han, som om guld och kvinnor förekommo allmänt i dessa nejder.

\initial De foro längs låga kuster, där intet sågs utom sand och kärr och en och annan fiskarhydda; de kommo förbi uddar där höga kors stodo uppresta, och de förstodo då att de nått till de kristnas land och till frankiska kuster. Ty de kunniga ombord kände till att dessa kors först satts upp av den store kejsaren Karl, alla kejsares stamfader, för att hålla nordiska sjöfarare borta från landet; men nordmännens gudar hade varit starkare än hans. De stucko in i sund och vikar vid hotande stormbyar och för att övernatta och sågo vattnen, saltare och grönare än dem de hittills känt, häva sig och sjunka vid ebb och flod. Inga skepp syntes till och inga människor, men stundom märken efter gammal bygd; ty fordom hade här funnits gott om byar, innan nordmännen kommo. Men allt var sedan länge utplundrat och lagt öde, och först långt söderut kunde sjöfarare numera räkna på vinning.

\initial De kommo ned där havet smalnade mellan England och fastlandet; och det blev tal om att sticka över till England. Ty de kände till att konung Edgar nyligen dött och efterträtts av omyndiga söner; detta hade bringat landet i gott rykte bland vikingar. Men Krok och Berse och andra bland de klokaste höllo före att frankernas land alltjämt var det bästa, om man bara komme långt nog åt söder; ty kungen av Frankrike och kejsaren av Tyskland lågo i strid med varandra om sina gränsmarker, och när sådant var i gång brukade av ålder kusttrakterna vara ypperlig mark för nordmän.

\initial Därför förblevo de på den frankiska sidan; men de lågo nu längre ut till havs än tidigare och höllo skarp utkik åt alla håll; ty de hade nu nått till det land som nordmän vunnit från kungen av Frankrike; och visserligen syntes här alltjämt ett och annat gammalt kors vid uddar och flodmynningar, men oftare pålar på vilka skäggiga huvuden voro uppsatta, till ett tecken att landets herrar ogärna sågo sjöfarare från hemlandet vid dessa kuster. Krok och hans män tyckte att sådan oginhet mot stamfränder vore en stor skam för de män som nu sutto med rikedomar i detta land; men det var vad man kunde vänta, sade de, av folk som kommit från Skåne och Själland; och de frågade Orm om han hade släkt här i landet. Orm sade att han inte trodde det, enär hans släkt alltid seglat på Irland; men uppsättandet av huvuden på pålar vore en sak som han skulle komma ihåg när han återvänt hem, ty sådant kunde visa sig gagneligt för fårskötseln. Åt detta skrattade alla och tyckte att han kunde svara bra för sig.

\initial De lade sig i försåt vid en flodmynning och togo några fiskebåtar men funno föga av värde i dem, och av männen i båtarna kunde de inte få ut något svar på sina frågor om var rika byar funnos i grannskapet; sedan de dräpt ett par av dem och likväl de övriga inte kunde svara begripligt, släppte de dem med livet, emedan de sågo skröpliga ut och varken dugde till roddare eller till försäljning. Mer än en gång voro de nattetid i land men vunno inte mycket; ty folket bodde i alltför stora och välförsvarade byar, och de fingo skynda tillbaka till skeppen för att inte bli kringrända av övermakt. De hoppades att det land där nordmän rådde snart måtte taga slut.

\initial En kväll mötte de fyra långskepp, som kommo roende upp från söder; de sågo ut att ha tung last, och Krok lät sina skepp gå nära dem för att se hur de voro bemannade. Det var en lugn afton, och de kommo långsamt varandra närmare; främlingarna satte upp en långsköld på främsta skeppets mast, med spetsen vänd uppåt, som tecken på att de nalkades som vänner; och Kroks män talade med dem på ett spjutkasts avstånd, under det de försökte räkna varandras styrka. Främlingarna sade sig vara från Jutland och på hemfärd från en långvarig resa. Med sju skepp hade de plundrat i Bretagne förra sommaren och därpå långt söderut; sedan hade de övervintrat på en ö utanför Loires mynning och varit uppåt floden; men därpå hade en sträng farsot kommit ibland dem, och nu voro de på väg hem med de skepp de räckt till att bemanna. Rörande sitt byte svarade de att den kloke sjöfararen aldrig prisade sin vinst innan han bragt den tryggt i land, men så mycket kunde de säga, eftersom de vid detta möte funno sig starka nog för att skydda vad de vunnit, att de för sin del inte ville klaga. Dåliga tider, jämfört med förr, finge man alltid räkna med, hur långt ut man än komme; men den som hade turen att hitta en oskövlad trakt i Bretagne eller längre åt söder, kunde alltjämt få lön för sin möda.

\initial Krok frågade om de hade vin eller gott öl att byta bort mot fläsk och torkad fisk; och på samma gång sökte han komma dem närmare, ty han var svårt frestad att göra ett försök mot dem och därmed på en gång få god lön för hela resan. Men jutarnas hövding lade genast sina skepp upp i bredd, med fören mot Krok, och svarade att de tänkte behålla sitt vin och sitt öl för eget bruk.

– Men du är välkommen närmare, sade han till Krok, om det finns någonting annat du vill smaka.

\initial Krok vägde ett spjut i handen och syntes oviss om vad han borde göra; men i detsamma blev det oro på ett av de främmande skeppen. Två män sågos i brottning med varandra vid relingen och tumlade därpå i vattnet, slutna i varandras armar. De sjönko båda, och den ene syntes inte mera; men den andre kom upp ett stycke från skeppen och dök åter när ett spjut slungades efter honom från dem han lämnat. Det ropades en del på de jutska skeppen, men när Kroks män frågade vad som stod på, fingo de intet svar. Skymningen började nu falla, och efter en stunds ordväxling fortsatte främlingarna sin rodd utan att Krok kunnat besluta sig för strid. Toke, som satt vid sin babordsåra närmast intill Orm på Kroks eget skepp, ropade nu till Krok:
– Kom hit och se! Min fiskelycka blir bättre och bättre.

\initial En hand höll i Tokes åra och en i Orms, och ett ansikte låg i vattenytan mellan händerna och såg upp mot skeppet. Det var storögt och mycket blekt, med svart hår och svart skägg.

– Detta må vara en rask man och en god simmare, sade en av männen; han har dykt under skeppet för att komma undan jutarna.

– Han må också vara en klok man, sade en annan, eftersom han tyr sig till oss såsom bättre män än de.

– Han är svart som ett troll och gulvit som en döding, sade en tredje, och ser knappt ut att kunna ha lycka med sig; det kan vara farligt att taga upp en sådan.

\initial Det talades nu för och emot om detta, och några ropade frågor till mannen i vattnet; men han låg orörlig och höll sig fast i årorna och klippte med ögonen och svajade för sjön. Till sist befallde Krok att han skulle tagas ombord; man kunde ju dräpa honom sedan, sade han till de motvilliga, om det skulle visa sig vara bäst så.

\initial Toke och Orm drogo in sina åror och fingo mannen ombord; han var gul i skinnet och kraftigt byggd, naken till midjan och hade endast några trasor att skyla sig med. Han vacklade och kunde knappt stå på benen, men han knöt näven mot de försvinnande jutska skeppen och spottade åt deras håll och gnisslade med tänderna och skrek någonting; därpå föll han omkull för en rullning men kom genast på fötter igen och slog sig för sitt bröst och sträckte armarna mot himlen och ropade med ändrad röst, men med ord som ingen förstod. Orm brukade säga på gamla dagar, när han berättade om sina minnen, att han aldrig hört en ilsknare tandagnisslan, inte heller en sorgmodigare och mera klingande röst än när främlingen ropade mot himlen.

\initial Han tedde sig sällsam för dem alla; de frågade honom mycket om vem han var och vad som hänt honom. Han förstod somligt av vad de frågade och kunde ge till svar brutna ord på nordiskt språk; och de tyckte sig förstå att han var jute och inte ville ro på lördagar och att detta var anledningen till hans hätskhet mot dem han nu rymt ifrån; men däri kunde de inte få någon mening, och några trodde att han var slagen av galenskap. De gåvo honom att äta och dricka; och han åt glupskt av bönor och fisk, men salt fläsk visade han ifrån sig med skräck. Krok sade att han kunde göra skäl för sig som roddare, och vid resans slut borde han kunna säljas för en vacker slant. Berse, fortsatte Krok, kunde kanske med sin klokhet försöka begripa honom och lura ut om han kunde ha något nyttigt att säga om de trakter han kom ifrån.

\initial De följande dagarna satt Berse mycket samman med främlingen, under det de samtalade så gott det ville gå. Berse var en lugn och tålmodig man, storätare och kunnig i skaldskap, som följt med till sjöss för att slippa undan en trätgirig hustru; han hade gott förstånd och mycken kunskap, och småningom lyckades han begripa allt mera av vad främlingen hade att säga. Detta meddelade han åt Krok och de andra.

– Han är inte galen, fast det kan synas så, sade Berse; inte heller är han en jute, och det syns ju också tydligt nog. Utan han säger att han är en jude. Detta är ett folk från öster som dödat den man som hålles för gud av de kristna. Detta dråp skedde för länge sedan; men de kristna hysa alltjämt stort hat till judarna för den sakens skull och döda dem gärna och vilja inte veta av förlikning eller dråpsböter. Därför bo de flesta judar hos den cordovanske kalifen, ty där hålles inte den dräpte mannen för gud.

\initial Berse tillfogade att han själv redan tidigare hört något liknande berättas, och flera bland männen sade sig också ha hört rykten i den vägen; Orm sade att han hört att den döde spikats på trä, liksom Lodbroks söner i gamla dagar gjort med den högste prästen i England. Men hur han kunnat hållas för gud efter det att judarna dödat honom kunde ingen förstå, ty en riktig gud kunde inte dödas av människor. Därpå fortsatte Berse med vad han förstått av främlingens tal:

– Han har varit slav hos jutarna ett år, och där led han mycket ont, emedan han inte kunde ro på lördagar; ty judarnas gud vredgas på varje jude som gör något på den dagen. Men detta kunde jutarna inte förstå, fast han många gånger ville förklara det för dem; de slogo honom och läto honom svälta när han inte ville ro. Hos dem lärde han det lilla han kan av vårt språk; men när han talar om dem, förbannar han dem på sitt eget tungomål, emedan han inte känner till nog ord på vårt. Han säger att han gråtit mycket hos dem och anropat sin gud om hjälp; han förstod att hans rop hörts när våra skepp kommo nära, och han tog med sig överbord en man som slagit honom mycket. Han bad sin gud att vara som en sköld för honom själv och låta den andre förgås; därför träffade intet spjut, och han fick kraft att dyka under vårt skepp; och sådan styrka har hans guds namn att han inte vill nämna det för mig, hur mycket jag än försöker få honom därtill. Detta är vad han säger om jutarna och sitt undslippande från dem; men han har mera att säga om annat, som han tror kan bli oss till nytta. Däribland är dock mycket som jag inte kan tydligt förstå.

\initial Krok sade att han hade svårt för att tro att en gud skulle göra sig besvär med att hjälpa en sådan arm trashank, hur mycket han än ropade, men att mannen onekligen handlat raskt och skickligt; och männen ville ha reda på varför denna märklige främling visade ifrån sig fläsk med vedervilja men likväl var glupsk på sämre föda. Berse svarade att det såg ut att vara med fläsket som med rodd på lördagar: judarnas gud vredgades vid att se en jude äta fläsk; men varför han råkade i vredesmod för en sådan sak, hade Berse inte kunnat bli klok på. Man kunde kanske förmoda, sade han, att guden själv tyckte så mycket om sådan föda att han inte unnade den åt sitt folk; och detta funno männen vara en trolig förklaring och prisade sig lyckliga att ha gudar som inte lade sig i sådana ting.

\initial Alla voro nu nyfikna på vad juden kunde ha mera att säga, som kunde bli dem till nytta; och till sist kom Berse så långt att han begrep det mesta:

– Han säger att han är en rik man i sitt eget land, hos den cordovanske kejsaren; han heter Salaman och är silversmed, och han säger att han också är stor skald. Han togs till fånga av en kristen herre från norr, som gjorde ett rövartåg till hans trakter. Denne lät honom skaffa en stor lösepenning och sålde honom därpå till en slavhandlare, emedan de kristna inte gärna hålla sitt ord till judar, för den dräpte gudens skull. Av slavhandlaren såldes han till handelsmän på sjön, och från dessa togs han av jutarna, och det var hans olycka att genast bli satt till åran en lördag. Nu hatar han förvisso dessa jutar med ett stort hat; men det är likväl ringa mot det hat han känner mot den kristne herren som bedrog honom. Denne herre är mycket rik och bor en dags väg från havet; och han säger att han gärna vill visa oss dit, på det vi må plundra den herren på allt han äger och bränna hans hus och sticka ut hans ögon och släppa honom naken bland ris och sten. Han säger att där finns rikedom för oss alla.

\initial Alla tyckte att detta var de bästa nyheter de hört på länge; och Salaman, som suttit bredvid Berse under dennes berättelse och noga följt med allt han kunde förstå, sprang skrikande upp och såg ut att vara mycket glad och kastade sig framstupa framför Krok och stack en flik av sitt skägg i munnen och tuggade på det; därpå grep han Kroks ena fot och satte den på sin nacke, under det han talade ivrigt och utan att någon kunde förstå honom. När han lugnat sig något, började han leta bland de ord han kunde; han sade att han ville tjäna Krok och hans män troget tills de vunnit dessa rikedomar och han själv fått sin hämnd; men han ville ha säkert löfte att han själv måtte få sticka ut ögonen på den kristne herren. Både Krok och Berse sade att detta var en rimlig begäran.

\initial På alla tre skeppen blev nu mycket tal om detta, och männen voro vid det bästa lynne; de sade att främlingen kanske inte hade mycken lycka för egen del, om man skulle döma efter vad han fått vara med om, men att han kanske hade så mycket mera åt dem, och Toke tyckte att han aldrig gjort ett bättre fiske. De voro vänliga mot juden och letade samman några klädesplagg åt honom och gåvo honom öl att dricka, fast de inte hade mycket kvar. Landet dit han ville föra dem hette Leon, och man visste någorlunda var det fanns: på höger hand mellan frankernas land och den cordovanske kalifens; kanske fem dagars god segling söderut från Bretagnes udde, som de nu siktat. De offrade åter till havsfolket och fingo god vind och styrde ut över öppna havet.

%
% F J Ä R D E   K A P I T L E T
%

\kapitel{FJÄRDE KAPITLET}{Hur Kroks män nådde Ramiros rike och besökte ett lönande ställe.}

\initial Orm brukade säga på gamla dagar, när han berättade om sina öden, att han inte hade mycket att klaga på medan han var med Krok, fast han så ovilligt kommit med på den resan. Slaget i skallen hade han ont av endast en kort tid; och med männen kom han gott överens, så att snart ingen tänkte på att han egentligen var deras fånge. De mindes med välbehag de goda får de fått hos honom och även på annat sätt var han dem till lags. Han kunde lika många skaldestycken som Berse och hade av sin moder lärt sig att framsäga dem med skalders ton; och han förstod också att på trovärdigt sätt berätta lögnhistorier, ehuru han själv erkände att Toke var hans överman i den konsten. Han skattades därför som en god och skicklig kamrat, som kunde komma med nöjsamt tidsfördriv under långa dagar med jämn vind, när alla voro fria från årorna.

\initial Några ombord klagade över att Krok stack åstad från Bretagne utan att först ha försökt skaffa ny färskmat; ty den föda man hade ombord började nu kännas gammal. Fläsket var härsket, stockfisken möglig, mjölet surt, brödet maskigt och vattnet skämt; men Krok och de erfarna männen höllo detta för att vara kost av bästa sort, som ingen sjöfarare kunde klaga på. Orm åt sina bitar med god aptit, men berättade därvid ofta om de läckerheter han varit van vid hemma. Berse sade att det syntes honom vara en vis och gudomlig ordning, att man med trivsel och god smak åt sådan föda till sjöss, som man hemma i land varken skulle kunna bjuda trälar eller hundar, utan endast svin; ty om det inte var så inrättat, skulle långfärder till havs ha ställt sig alltför svåra.

\initial Toke sade att för honom kändes det svårast att ölet nu tagit slut. Han var ingen kräsen man, sade han, och han trodde sig kunna äta de flesta ting om det behövdes, till och med sina sälskinnsskor, men endast om han finge gott öl därtill. För honom, sade han, ville det inte gå att tänka sig ett liv utan öl, varken på vatten eller i land; och han låg åt juden med många frågor angående ölet i det land de skulle komma till, men kunde inte få klart besked. Han berättade om stora gillen och dryckeslag, där han varit med, och sörjde över att han inte då passat på att dricka ännu mera.

\initial Andra natten på havet fingo de stark vind och hög sjö och voro glada att himlen förblev klar, ty de styrde nu efter stjärnorna. Krok började bli rädd att komma ut i det gränslösa havet; men de sjökunnigaste männen sade att hur man än seglade på sydlig kurs, skulle man alltid finna land på vänster hand, utom endast vid Njörvasund, där segelleden gick in till Rom som låg mitt i världen. De män, som seglade från Norge till Island, sade Berse, hade det svårare ställt; ty kommo de förbi Island, hade de intet annat land att vänta, utan endast det tomma havet utan slut.

\initial Juden var stjärnkunnig och sade sig vara skicklig i att finna rätt kurs; men han blev till föga hjälp med detta, ty hans stjärnor hade främmande namn och han själv blev sjösjuk. Det blev Orm också; och han och Salaman hängde över relingen bredvid varandra i stort elände och trodde att de skulle förgås. Juden skrek mycket på sitt eget tungomål, när han inte spydde, och Orm sade till honom att hålla tyst även om han hade ont; men han svarade att han ropade till sin gud, som fanns i stormvinden. Orm grep honom då i nacken och sade att hur sjuk han själv än kände sig, skulle han vara man att få honom överbord om han ropade en enda gång till, ty för sin del tyckte han att det blåste tillräckligt redan, utan att guden behövde kallas närmare.

\initial Salaman tystnade; och fram mot morgonen lugnade vädret av, och det blev bättre för dem båda. Salaman var mycket grön i ansiktet, men han smålog vänligt mot Orm och såg inte ut att hysa agg mot honom och pekade ut över havet mot soluppgången. Han letade bland de ord han kände till och sade att detta var morgonens röda vingar ytterst i havet och att hans gud var där. Orm svarade att det syntes honom bäst att ha honom på det avståndet.

\initial Fram på morgonen skönjde de berg framför sig i fjärran. De kommo intill kusten och hade besvär med att finna en skyddad vik för skeppen; juden sade att denna trakt var okänd för honom. De gingo i land och kommo genast i strid med ställets innebyggare, som bodde tätt här; dessa flydde snart, och Kroks män genomletade deras hyddor och återvände med getter och annan föda samt ett par fångar. Eldar gjordes upp; och alla voro glada att ha nått tryggt i land och åter få smaka stekt kött. Toke letade mycket efter öl, men fann endast ett par skinnbälgar fulla med vin, som var så strävt och surt att han sade sig känna buken skrumpna när han svalde det. Därför orkade han inte själv dricka allt, utan gav bort det som blev över och satt sedan för sig själv hela kvällen och sjöng sorgset, med tårar i skägget. Berse sade att han inte finge störas, ty han var en farlig man när han druckit så mycket att han grät.

\initial Salaman talade med fångarna; därpå sade han att de befunno sig i den kastilianske grevens land och att det ställe han ville föra dem till låg långt åt väster. Krok sade att de fingo vänta på annan vind för att komma åt det hållet, och att de kunde vila sig och äta gott under tiden; men det kunde bli besvärligt, ansåg han, om de bleve överfallna av manstarka flockar här, medan vinden låg på land, eller om fientliga skepp stängde in dem i viken där de lågo. Salaman förklarade nu, så gott han kunde, att faran för sådant vore ringa; ty greven av Kastilien hade knappt något enda skepp i sjön, och det skulle taga sin tid för honom att få en styrka samlad som kunde göra dem skada. Förr, sade han, hade greven av Kastilien varit mäktig; men han var nu mycket betryckt av kalifen och måste betala skatt till honom; ty utom kejsar Otto i Tyskland och kejsar Basilius i Konstantinopel funnes nu ingen härskare i världen så mäktig som kalifen i Cordova. Häråt skrattade männen mycket och sade att juden talade så gott han förstod, men att han inte syntes ha mycken reda på sig i dessa ting. Hade han inte hört talas om kung Harald av Danmark? frågade de. Och visste han inte att kung Harald var den mäktigaste av alla?

\initial Orm var ännu matt efter sin sjösjuka och hade ingen stor matlust och trodde att han skulle bli allvarligt sjuk, ty han hyste ofta oro för sin hälsa. Han somnade snart invid en eld och sov gott; men fram på natten, när allt blivit tyst i lägret, kom Toke och väckte honom. Han grät och sade att Orm var den ende vän han hade, och nu ville han sjunga för honom en visa som han just kommit att tänka på; den handlade om två björnungar, sade han, och han hade lärt den som litet barn av sin moder, och det var den vackraste visa han visste. Därmed satte han sig bredvid Orm och torkade sina tårar och började sjunga. Orm hade den egenheten att det föll sig svårt för honom att vara vänlig när han väcktes ur god sömn; men han sade ingenting utan vände sig över på andra sidan och försökte somna igen.

\initial Toke kom inte ihåg mycket av sin visa och blev åter ledsen; han sade att han suttit ensam hela kvällen och att ingen kommit för att hålla honom sällskap. Att Orm inte tittat till honom i hans sorg hade smärtat honom mest; ty han hade alltid hållit Orm för sin vän, ända från första stund, men förstod nu att han var en lymmel och kältring, liksom alla andra skåningar; och när en valp som han bar sig illa åt, var ett kok stryk det enda rätta.

\initial Därmed reste han sig för att se sig om efter en käpp; och Orm var nu klarvaken och satte sig upp. När Toke såg detta, måttade han en spark åt honom, men i detsamma tog Orm en brand ur elden och slängde i ansiktet på honom. Toke vek undan mitt i sparken och föll på rygg, men han var fort uppe igen och var nu vit i ansiktet och helt ifrån sig. Orm kom också kvickt på fötter. Det var klart månsken; men det flimrade rött för ögonen på Orm i hans raseri när han rusade på Toke, som försökte draga sitt svärd; Orm hade lagt sitt ifrån sig och inte hunnit få tag i det. Toke var en stor och stark man, bred över bringan och med väldiga händer; Orm hade inte ännu nått sin fulla styrka, men var redan stark nog för de flesta. Han fick ena armen runt nacken på Toke och grep om hans högra handlove med den andra, så att han inte skulle få ut svärdet; men Toke fick gott tag i hans kläder och rätade upp sig med ett ryck och slängde honom med benen i vädret över sin axel. Men Orm släppte inte helt sitt grepp, fast det kändes som om han höll på att knäckas; han svängde runt och fick ena knät mot ryggen på Toke och kastade sig bakåt och fick honom över sig. Därpå tog han i med all sin styrka och vältrade runt, så att Toke kom under med ansiktet mot marken. Många hade nu vaknat; och Berse kom springande med rep och sade att detta var vad man kunde vänta när Toke fått så mycket i sig. Han blev nu stadigt bunden till händer och fötter, fast han häftigt spjärnade emot. Han lugnade sig snart, och efter en stund ropade han till Orm, att han nu kom ihåg resten av visan; han började sjunga, men Berse hällde vatten på honom, och han föll därpå i sömn.

\initial När han vaknade nästa morgon, klagade han över att vara bunden och mindes ingenting; han fick höra vad som skett och kände sig ångerfull gentemot Orm och sade att det var hans olycka att ställa till förtret när han druckit; ty öl gjorde honom i sanning till en annan man, och vin kanske lika mycket. Han ville veta om Orm hyste agg mot honom för vad som hänt. Orm sade att han inte kände agg och att han även i fortsättningen gärna skulle vara med på något litet slagsmål, när Toke kände sig sinnad; men en sak borde han lova, nämligen att låta bli att sjunga; ty sång av en nattskärra, eller av en gammal kråka på ett uthustak, vore långt vackrare än hans nattvisor. Åt detta skrattade Toke och lovade att söka bättra sig därvidlag; ty han var en godmodig man när inte öl eller vin ändrade honom.

\initial Alla tyckte att Orm skött sig över förväntan i denna sak, så ung som han var; ty det brukade bli svåra märken på de flesta som råkade i Tokes händer när han kommit så långt som till tårar; och Orm steg därmed mycket både i sin egen uppskattning och i de andras. De började kalla honom Röde Orm efter detta, inte endast för hans röda hårs skull utan även emedan han visat sig vara en man som kunde sätta skarpt mot skarpt och som inte borde retas i onödan.

\initial Efter några dagar kom god vind, och skeppen lade ut. De höllo sig långt ute från kusten, för att undvika farliga strömdrag, och styrde västerut längs Ramiros rike och rundade udden längst i väster. De rodde nu söderut längs en brant och sönderskuren kust, och därpå fram genom en skärgård som männen funno vara lik den hemma i Blekinge, tills de nådde en flodmynning som juden höll utkik efter. Där styrde de in med högvattnet och rodde uppför floden tills de hejdades av forsar. De gingo i land och höllo rådslag, och Salaman fick beskriva den väg de hade kvar. Han sade att raska män skulle på mindre än en dag kunna taga sig fram härifrån till den man han ville hämnas på, en av konung Ramiros markgrevar vid namn Ordone: den störste rövare och illgärningsman, sade han, längs hela de kristnas gräns.

\initial Krok och Berse utfrågade honom noga om fästet, om dess styrka och belägenhet och om hur många män markgreven brukade ha hos sig där. Det låg i en så klippig och oländig trakt, sade Salaman, att kalifens här, som mest bestod av ryttare, aldrig kommit där i närheten. Därför var det ett gott tillhåll för en rövare och rymde stora rikedomar. Det var uppfört av ekstockar och skyddat av en jordvall med pålverk, och besättningen kunde vara högst tvåhundra man. Eftersom det låg så undangömt, trodde inte Salaman att vakthållningen var särskilt god, och ofta var mesta delen av besättningen åstad på plundringståg söderut.

\initial Krok sade att besättningens styrka oroade honom mindre än vallen och pålverket, vilka kunde göra det svårt nog att komma in i en hast. Några av männen sade att det borde vara enkelt att få eld på pålverket; men Berse sade att om alltsamman toge eld, skulle man ha föga glädje av rikedomarna. Det beslöts till sist att man skulle lita på lyckan och bestämma sig för bästa sätt vid framkomsten; fyrtio man skulle stanna kvar vid skeppen, och de andra skulle draga åstad när det börjat bli svalt på kvällen. Det lottades om vilka som skulle bli kvar, ty alla ville helst vara med där rikedomar voro att vinna.

\initial De sågo till sina vapen och sovo i en ekdunge under de hetaste timmarna; därpå stärkte de sig med föda, och när kvällen föll drog flocken åstad; tillsammans voro de etthundratrettiosex man. Krok gick främst, med juden och Berse, och därpå följde man efter man; somliga hade brynjor, andra lädertröjor, de flesta hade svärd och spjut, men några yxor, och alla buro sköld och hjälm. Orm gick med Toke, som sade att det var gott att få mjuka upp benen efter allt sittandet på roddarbänken.

\initial De gingo fram genom en vildmark där ingen människoboning syntes till; ty dessa gränstrakter mellan de kristna och andalusierna lågo sedan länge öde. De följde flodens norra strand och vadade över många bäckar; mörkret blev tätt, och de höllo rast och väntade på månuppgången. Därpå togo de av åt norr, upp genom en dalgång, och kommo nu raskt framåt över öppen mark; och Salaman befanns vara en god vägvisare, ty före gryningen nådde de fram till fästets närhet. Där lågo de stilla bland några rissnår och vilade en stund och spanade framåt för att urskilja vad de kunde i det bleka månskenet. De blevo modfällda vid åsynen av palissaderna, ty dessa voro av grova stockar och av mera än dubbel manshöjd; och porten var överbyggd och såg mycket stadig ut.

\initial Krok sade att det kanske inte bleve lätt att få eld på sådant timmer, och att han helst skulle vilja taga fästet utan eld; men det funnes kanske ingen annan utväg: man finge bära fram risknippor och stapla upp mot palissaden och tända på och hoppas att inte allt skulle brinna. Han frågade Berse om denne hade något bättre att föreslå; men Berse rev sig i huvudet och suckade och sade att han inte kunde komma på något bättre, fast han tyckte illa om att använda eld. Inte heller Salaman visste bättre råd, utan sade att han finge vara nöjd med att se den trolöse brinna, fast han hoppats på bättre hämnd än så.

\initial Toke kom nu krypande fram till Krok och Berse och undrade vad man väntade på: han kände sig törstig, och ju förr man stormade fästet, desto förr kunde han få något att dricka. Krok sade att svårigheten var att komma in. Toke sade att om han finge fem spjut, trodde han sig kunna visa att han dugde till annat än att ro och dricka öl. De ville ha reda på hur han tänkte göra; men han sade endast att han tänkte göra väg för dem in i fästet, om allt ginge väl, och att spjutens ägare finge vara beredda att skäfta om sina lån vid återfåendet. Berse, som kände honom sedan gammalt, sade att man borde ge honom spjuten. Så skedde, och Toke högg av skaften ovanför järnskoningen, så att han fick kvar alnslånga handtag till bladen. Därpå sade han att han var färdig; och han och Krok började försiktigt krypa fram mot vallen, dolda av ris och stenar, med utvalda män efter sig. Ett par tuppar hördes gala inne i fästet, men eljest var allting tyst och stilla.

\initial Ett stycke bredvid porten kröpo de upp på vallen. Toke reste sig intill pålverket; en god aln ovan marken stack han ett spjutblad in mellan tvenne stockar och tryckte till med all kraft för att få det att sitta stadigt. Högre upp, i springan närmast intill, fäste han ett andra blad; och när han till sist utan buller fått dem stadiga, steg han försiktigt upp på skaftstumparna och satte fast ett tredje blad högre upp i nästa springa. Men det var omöjligt för honom att fästa det tredje stadigt, så som han stod och utan buller; och Krok, som nu såg hur hans plan var, tecknade åt honom att komma ner och sade att det inte ginge utan hammare längre, även om folk skulle bli störda i sin sömn. Därmed fick han de återstående spjutbladen och tog Tokes plats på de båda fotsteg som redan sutto stadigt; han drev in det med ett par slag av sin yxhammare och satte därefter på samma sätt fast det fjärde och det femte snett högre upp. Han steg upp på dem efterhand som de satts fast och nådde så att komma upp till pålverkets krön.

\initial I detsamma blev det skrik och larm inne i fästet, med starkt blåsande i lurar; men andra män följde efter på Tokes trappa, så hastigt de kunde klättra, och kommo över efter Krok. Längs pålverkets innersida löpte en träbro för bågskyttar; Krok och de som följde honom kommo ned på den och höggo ned några yrvakna män som kommo springande med spjut och bågar. De fingo nu pilar på sig från marken, och ett par blevo träffade; men Krok och de andra sprungo längs träbron bort till porten och hoppade ned där för att fort få den öppnad, så att hela flocken kunde komma in. Här blev det nu hård strid, ty många av försvararna hade redan hunnit hit, och flera kommo löpande i varje ögonblick. En av de tjugo män som följt Krok hade blivit hängande på palissaden med en pil i ögat, och tre andra hade fallit för pilar när de sprungo längs träbron; men de som nått marken slöto sig tätt tillsamman och höjde härrop och kommo med spjut och svärd in i portgången, där det var mörkt och där trängseln blev stor, ty de hade nu fiender både framför sig och bakom.

\initial De fingo svar på sitt härrop utifrån, ty männen därute hade sprungit fram mot vallen, när de sågo försöket lyckas; många började hugga mot porten med yxor, under det andra klättrade in på Tokes trappa och kommo väl till pass för att hjälpa dem som voro i portgången. Striden där var oviss, med vänner och fiender blandade om varandra; Krok fällde flera män med sin yxa, men träffades själv över nacken med en stridsklubba av en stor man, som hade svart flätat skägg och såg ut att vara en hövding; hjälmen tog emot, men Krok vacklade och föll på knä. I en trängsel av sköldar och män, där spjut inte längre kunde brukas och där fötterna slunto i blod, kommo till sist Toke och Orm och ett par andra fram till porten och fingo reglarna skjutna ifrån, och de fiender i portgången som inte hunno undan blevo genast nedhuggna.

\initial Nu kom en stor skräck över de kristna, och de flydde med döden efter sig. Salaman, som var bland de första som kommo in genom porten, sprang fram som en besatt och snävade över fallna och hittade ett svärd på marken och svängde det över sitt huvud och skrek överljutt att alla skulle skynda sig fram till borgen; och Krok, som ännu var yr av slaget och inte kunde stå på benen, ropade detsamma där han låg innanför portgången. Många av männen sprungo in i hyddor, som lågo innanför vallen, för att släcka sin törst eller leta efter kvinnor; men de flesta följde efter de flyende fram till det stora borghuset i mitten, där porten var full av flyende. Förföljarna kommo in samman med dessa, innan den kunde stängas till; och inne i huset blev det nu åter strid, när de flyende sågo sig tvungna att sätta sig till motvärn. Den store mannen med det flätade skägget stred tappert och fällde två män som kommo mot honom; men han trängdes in i ett hörn och fick hugg på sig och föll omkull med svåra sår. Salaman, som nådde fram när han föll, kastade sig över honom och grep honom i skägget och spottade på honom och skrek ivrigt; men den andre såg inte ut att förstå mycket, utan riste på sig och klippte med ögonen och dog.

\initial Salaman brast ut i klagan över att han gått miste om sin fulla hämnd och inte själv fått bära hand på sin fiende; och de kristna som återstodo upphörde att försvara sig sedan deras anförare fallit. Några av dem skonades till livet, emedan de kunde vara till nytta; och segrarna togo rikligt för sig av mat och dryck, både öl och vin. Därpå genomletades fästet efter byte; och trätor uppstodo om de kvinnor som hittades undankrupna i skrymslen, ty männen hade nu lång tid varit utan kvinnor. Allt byte bars samman – mynt, smycken, vapen, kläder, bonader, pansaren, husgeråd, seldon, silverfat och mera –; och när det var samlat, befanns det vara mera än någon kunnat tänka; ty här, sade Salaman, fanns många års rov från andalusierna. Krok, som nu åter kunde stå på benen och fått en klut indränkt i vin lindad runt huvudet, gladdes vid synen men var rädd att det skulle bli svårt att ha allt på skeppen; men Berse trodde att allt skulle kunna tagas med.

– Ty ingen, sade han, klagar över tung last, när det är byte han har ombord.

\initial De förnöjde sig den dagen, i stor belåtenhet med vad de vunnit, och sovo sedan; på natten bröto de upp mot skeppen. Alla fångar voro hårt lastade, och männen själva hade mycket att bära. Några andalusiska fångar hade hittats i fästets källare; de gräto av lycka över sin befrielse men sågo eländiga ut och orkade inte bära mycket. De fingo sin frihet och följde med på återvägen, för att samman med Salaman fortsätta åt söder till sina egna trakter. Några åsnor hade tagits, och Krok sattes upp på en av dem och red i spetsen för tåget, med sina ben räckande till marken. Bakom honom leddes de andra, lastade med föda och öl; deras bördor lättades fort, ty männen ville ofta rasta och förfriska sig.

\initial Berse försökte skynda på dem för att komma fort till skeppen. Han var orolig för förföljelse, ty av folket i fästet hade några undkommit och kunde ha nått långt och kallat på hjälp; men de flesta männen voro glada och halvdruckna och brydde sig föga om vad han sade. Orm bar en packe siden, en bronsspegel och en stor skål av glas, som han hade mycket besvär med; Toke bar över axeln ett stort träskrin med vackra beslag, fyllt av många ting, och ledde med sig en flicka som fallit honom i smaken och som han ville behålla så länge som möjligt. Han skrattade mycket och sade till Orm att han hoppades att flickan vore markgrevens dotter, men därpå blev han sorgmodig vid tanken att det inte skulle finnas plats för henne på skeppet. Han var snubblig i fötterna av allt han druckit; men flickan såg ut att redan vara mån om honom och stödde honom när han snävade. Hon var välväxt och mycket ung; och Orm sade att han sällan sett en så vacker flicka som hon och att det kunde vara önskligt att ha sådan god kvinnolycka som Toke. Men Toke svarade att hur goda vänner de än voro, kunde han inte dela henne med Orm, emedan han tyckte alltför mycket om henne och ville behålla henne för sig själv om det läte sig göra.

\initial De kommo åter till skeppen, och det blev stor glädje bland dem som stannat kvar där när de sågo det rika bytet; ty detta skulle delas bland alla. Salaman fick mycken tack och rika skänker och drog därpå vidare med de befriade fångarna; han var ivrig att fort komma undan från de kristnas gränser. Toke, som drack hela tiden, började gråta när han fick höra att Salaman dragit bort, och sade att nu hade han ingen som kunde hjälpa honom att tala med flickan, och drog sitt svärd och ville springa efter honom. Men Orm och de andra lyckades få honom lugnad utan våldsamheter; och han somnade hos flickan, sedan han bundit henne fast vid sig så att hon inte skulle löpa bort eller bli stulen medan han sov.

\initial Nästa morgon började delningen av bytet, och det var ingen lätt sak. Alla skulle ha sin lika del; men Krok och Berse och rorsmannen och några till skulle ha tredubbelt mot de andra; och fast de förståndigaste sattes att dela upp allt rättvist, var det svårt att få alla belåtna. Berse sade, att eftersom det till stor del var Tokes förtjänst att fästet vunnits, borde också han ha tredubbel lott; och detta funno alla vara rätt. Men Toke själv sade att han skulle vara nöjd med enkel lott, om han i stället finge taga sin flicka ombord och ha henne i fred där.

– Ty jag vill gärna ha henne med hem, sade han, fast jag inte är säker på om hon är grevedotter. Jag trivs bra med henne redan, och ännu bättre blir det när vi kunna förstå varandra och hon lärt sig tala vårt språk.

\initial Berse sade att detta sista kanske inte bleve den fördel som Toke kunde tro, och därpå sade Krok att skeppen komme att bli så tyngda av det myckna bytet, trots att elva stupade män kunde frånräknas, att han knappt trodde att flickan kunde få plats ombord; snarare finge också en del av det minst värdefulla bytet lämnas kvar.

\initial Toke reste sig då och lyfte upp flickan på sin axel och bad alla att noga se hur vacker och välväxt hon var.

– Hon synes mig kunna väcka begärelse hos de flesta, sade han. Och om nu någon finns här, som åstundar henne tillräckligt mycket, vill jag slåss med honom nu genast så som han själv tycker bäst, med svärd eller yxa. Den som vinner behåller flickan; och den som dödas lättar då mera på skeppet än vad hon tynger; och på så sätt kan jag ändå få henne med mig hem.

\initial Flickan höll sig fast med ena handen i Tokes kindskägg och blev röd och sprattlade med benen och satte andra handen för ögonen, men tog bort den igen, och syntes glad att bli så mycket begapad; och alla tyckte att Tokes förslag var skickligt uttänkt. Men ingen ville slåss med honom, trots flickans skönhet, ty han var avhållen av alla och dessutom fruktad för sin styrka och sin händighet med vapen.

\initial Sedan allt rovet delats och stuvats ombord, bestämdes det att Toke skulle få taga sin flicka med på Kroks skepp, ehuru det var tungt lastat; ty sådan lön kunde han vara värd efter vad han uträttat vid fästet. Det hölls nu rådslag om hemfärden; det blev beslutat att gå upp längs kusterna, om så bleve tvunget för vädrets skull, men helst försöka komma upp till Irland, för att gå hem runt de skotska öarna; ty med så rik last skulle det bli alltför vågsamt att segla hem genom trånga farvatten, där möten kunde väntas.

\initial Sedan åto de och drucko vad de kunde, ty det fanns nu mat och dryck i riklighet, och något av födan måste lämnas kvar; och alla voro fulla av skämt och glädje och berättade för varandra vad de skulle skaffa sig för sina rikedomar efter hemkomsten. Därpå lade de ut och rodde nedför floden. Krok var nu tämligen återställd; men hövitsmannen på ett av de andra skeppen hade fallit vid fästet, och Berse tog befälet på hans skepp. Toke och Orm sutto vid sina åror på Kroks skepp som förr och hade det lätt nedför strömmen; Toke höll ett öga på flickan, som mest satt vid hans knän, och var mån om att ingen skulle komma henne nära i onödan.
